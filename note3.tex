%
% A new paper...
% Title: Convergence of the Gutt Star Product
% git-repository is gstar
%
% From now on, we proudly use the chairx style file for
% everything. All stuff in there is soo useful!
%


%
% Before we start, let's nag a bit about old latex constructs
% just to learn... This will be removed in the final version
%

\RequirePackage[l2tabu, orthodox]{nag}


%
% we always start with 11pt, draft mode on for easier editing and
% english as default language
%

\documentclass[
11pt,                          % standard font size
%draft,                         % draft or final?
english                        % standard language
]{article}


%
% some macro packages
%

\usepackage[english]{babel}    % with explicit language
\usepackage{amsmath}           % ams mathematical stuff
\usepackage[utf8]{inputenc}    % smart input of funny chars
\usepackage[T1]{fontenc}       % also for the font encoding
\usepackage{longtable}         % tables longer than one page
\usepackage{exscale}           % large summation signs in 11pt
\usepackage[final]{graphicx}   % to include pdf pictures
\usepackage[sort]{cite}        % nicer citations
\usepackage{array}             % nice tables
\usepackage{wasysym}           % smiley symbols
\usepackage[a4paper]{geometry} % geometry of page layout
%\usepackage{gitinfo2}          % include git info: Version 2
%\usepackage[multiuser]{fixme}  % correction notes, warnings etc.
\usepackage{xspace}            % better spacing after macros
\usepackage{tikz}              % for commutative diagrams and stuff
\usepackage{ifdraft}           % to determine whether draft mode
\usepackage{chairx}            % the Chair X style file
\usepackage[expansion=false    % no font expansion
           ]{microtype}        % only protrusion
\usepackage[nottoc]{tocbibind} % refs and index in the toc
\usepackage[backref=page,      % backrefs in the bibliography
           final=true,         % always treat as final
           pdfpagelabels       % use pdf page labels
           ]{hyperref}         % hyperrefs are cool!

%
% Some own macros...
%

%\usepackage{mnsymbol}

\newcommand{\ostar}{\star}
\newcommand{\coproduct}{\Delta}
\newcommand{\ocoproduct}{\Delta}

\newcommand{\Eta}{\mathrm{H}}
\newcommand{\Rho}{\operatorname{P}}
\newcommand{\bch}[2]{\mathrm{BCH}\left(#1, #2\right)}
\newcommand{\bchpart}[3]{\mathrm{BCH}_{#1}\left(#2, #3\right)}
\newcommand{\bchparts}[4]{\mathrm{BCH}_{#1, #2}\left(#3, #4\right)}
\newcommand{\bchtilde}[4]{\widetilde{\mathrm{BCH}}_{#1, #2}\left(#3; #4\right)}
\newcommand\ot[2]{\stackrel{\mathclap{#1}}{#2}}
%
%\mathrel{\overset{\makebox[0pt]
%	{\mbox{\normalfont\footnotesize\sffamily #1}}}{#2}}}


%
% pdf files for graphics in the following directory:
%

\graphicspath{{../tikz/}}


%
% tikz libraries to be loaded, feel free to add more...
%

\usetikzlibrary{matrix}
\usetikzlibrary{arrows}
\usetikzlibrary{patterns}
\usetikzlibrary{decorations.pathreplacing}


%
% page dimensions, scaling etc. Not final yet
%


\geometry{bindingoffset=0cm}
\geometry{hcentering=true}
\geometry{hscale=0.8}
\geometry{vscale=0.8}


%
% check whether draft or not: synctex is soo cool
%

%\ifdraft{\synctex=1}{}


%
% fix me settings
%

%\fxusetheme{color}


%
% Get the authors from external file
% This used in all documents of the paper project
%

%\input{../authors/authors}


%
% own local math macros follow here
%


%
% title page for Convergence of the Gutt Star Product
% authors are included from the authors file
% This has to be adapted in the final version
%

\title{Two identical notions}



%
% the text starts here
%

\begin{document}



\section{Topologically nilpotent Banach-Lie algebras}

In this part, we want to prove that topologically nilpotent Banach-Lie algebras are 
uniformly topologically nilpotent. The proof is almost the same as in [DiM\"u], it 
just needs a slight change.
\begin{proposition}
	Topologically nilpotent Banach-Lie algebras are uniformly topologically 
	nilpotent.
\end{proposition}
\begin{proof}
	Let $\lie{g}$ be a Banach-Lie algebra, then the limit $\mathcal{N} = 
	\lim_{n \rightarrow \infty} \mathcal{N}(n)$ exists. We will show 
	that there exists a sequence $(\xi_1, \xi_2, \ldots) \subset \mathbb{B}_1(0)$, 
	such that
	\begin{equation*}
		\limsup_{n \rightarrow \infty}
		\norm{\ad_{\xi_1} \circ \cdots \circ \ad_{\xi_n} }^{1/n}
		=
		\mathcal{N},
	\end{equation*}
	which implies the proposition, since the left hand side converges to zero by 
	the definition of topological nilpotency. It is clear that
	\begin{equation*}
		\limsup_{n \rightarrow \infty}
		\norm{\ad_{\xi_1} \circ \cdots \circ \ad_{\xi_n} }^{1/n}
		\leq
		\mathcal{N},
	\end{equation*}
	for every such sequence, so we find a sequence such that
	\begin{equation*}
		\limsup_{n \rightarrow \infty}
		\norm{\ad_{\xi_1} \circ \cdots \circ \ad_{\xi_n} }^{1/n}
		\geq
		\mathcal{N}
	\end{equation*}
	holds. Since this is clear for $\mathcal{N} = 0$, we suppose $\mathcal{N} > 0$.
	We define the space of sequences $X = \mathbb{B}_1(0)^{\mathbb{N}}$ with the
	metric
	\begin{equation*}
		\halbnorm{d}((\xi), (\eta))
		=
		\sum\limits_{i = 1}^\infty
		2^{-i} \frac{\norm{\xi_i - \eta_i}}{1 + \norm{\xi_i - \eta_i}}.
	\end{equation*}
	With this metric, $X$ is complete. Define furthermore for $\delta > 0$
	\begin{align*}
		X_{k, \delta}
		&=
		\left\{
			(\xi) \in X
		\ \left| \
			\norm{ \ad_{\xi_1} \circ \cdots \circ \ad_{\xi_n} }^{1/n}
			>
			(1 - \delta) \mathcal{N}
			, \
			\text{ for some }
			n \geq k	
		\right.
		\right\}
		\\
		Y_{k, \delta}
		&=
		\left\{
			(\xi) \in X
		\ \left| \
			\norm{ \ad_{\xi_1} \circ \cdots \circ \ad_{\xi_k} }^{1/k}
			>
			(1 - \delta) \mathcal{N}
		\right.
		\right\}.
	\end{align*}
	Then it is easy to see, that the $Y_{k, \delta}$ are open in $X$. Hence also 
	$X_{k, \delta}$ is open in $X$, since
	\begin{equation*}
		X_{k, \delta}
		=
		\bigcup\limits_{n \geq k}
		Y_{n, \delta}.
	\end{equation*}
	Our aim is to show, that each of these $X_{k, \delta}$ is dense in $X$, because 
	then, we could conclude that
	\begin{equation*}
		\bigcap\limits_{k = 1}^\infty
		X_{k, 1/k}
		\neq
		\emptyset
	\end{equation*}
	by the Baire category theorem for metric spaces. Clearly the set
	$\bigcup\limits_{r < 1} \mathbb{B}_r(0)$
	is dense in $X$ and we want to show that for any $k$ and any $\delta$, 
	$X_{k, \delta}$ is dense in $\bigcup\limits_{r < 1} \mathbb{B}_r(0)$. 
	So we need to show that for every $k \in \mathbb{N}$, $r < 1$, 
	$(\xi) \in \mathbb{B}_r(0)^{\mathbb{N}}$, $0 < \delta < 1$ and $0 < \varepsilon 
	< 1 - r$, there is a $(\eta) \in X_{k, \delta}$, such that
	\begin{equation*}
		\halbnorm{d}((\xi), (\eta))
		<
		\varepsilon.
	\end{equation*}
	So let's fix $k \in \mathbb{N}$, $\delta, r \in (0,1)$, $0 < \varepsilon < 1-r$
	and $(\xi) \in \mathbb{B}_r(0)^\mathbb{N}$. Let $m \in \mathbb{N}$ be big enough
	to have $2 \cdot 2^{-m} < \varepsilon$. Then, if for all $i \in \{1, \ldots, 
	m\}$ the estimate $\norm{\xi_i - \eta_i} < \frac{\varepsilon}{2}$ holds for two 
	sequences $(\xi), (\eta) \in X$, then $\halbnorm{d}((\xi), (\eta)) < 
	\varepsilon$. We fix such an $m \in \mathbb{N}$ and choose $\ell \geq 
	\max{k, m}$ such that
	\begin{equation*}
		\left(
			\frac{\varepsilon}{2^{m + 1}}
		\right)^{1 / \ell}
		\geq
		\sqrt{1 - \delta}.
	\end{equation*}
	This is certainly possible for $\ell$ big enough. We can also choose some
	$\vartheta_1, \ldots, \vartheta_\ell \in \mathbb{B}_1(0)$ such that
	\begin{equation*}
		\norm{
			\ad_{\vartheta_1} \circ \cdots \circ \ad_{\vartheta_\ell}
		}^{1/ \ell}
		\geq
		\mathcal{N}(l) \sqrt{1 - \delta}
		\geq
		\mathcal{N} \sqrt{1 - \delta}.
	\end{equation*}
	Just by the definition of $\mathcal{N}(\ell)$, such elements must exist.
	Now define $\varepsilon_t = \varepsilon \E^{2 \pi \I t/m} $ for $t = 0, 1, 
	\ldots, m$ and set $\varepsilon_0 = 0$. Set for $t$ in the same limits
	\begin{equation*}
		\theta_i^t
		=
		\xi_i + \frac{\varepsilon_t}{2} \vartheta_i
	\end{equation*}
	and
	\begin{equation*}
		\Theta_t
		=
		\ad_{\theta_1^t} \circ \cdots \circ \ad_{\theta_m^t}
		\circ
		\ad_{\vartheta_{m+1}} \circ \cdots \circ \ad_{\vartheta_\ell}.
	\end{equation*}
	Then we have
	\begin{equation*}
		\Theta_0
		=
		\ad_{\xi_1} \circ \cdots \circ \ad_{\xi_m}
		\circ
		\ad_{\vartheta_{m+1}} \circ \cdots \circ \ad_{\vartheta_\ell}.
	\end{equation*}
	and for algebraic reasons and because of the linearity of $\ad$
	\begin{equation*}
		\sum\limits_{j=1}^m
		(\Theta_j - \Theta_0)
		=
		\frac{m \varepsilon^m}{2^m}
		\ad_{\vartheta_1} \circ \cdots \circ \ad_{\vartheta_\ell}.
	\end{equation*}
	Hence we find
	\begin{equation*}
		\Big\Vert
			\sum\limits_{j=1}^m
			(\Theta_j - \Theta_0)
		\Big\Vert
		=
		\frac{m \varepsilon^m}{2^m}
		\norm{ \ad_{\vartheta_1} \circ \cdots \circ \ad_{\vartheta_\ell} }
		\geq
		\frac{m \varepsilon^m}{2^m}
		\mathcal{N}^\ell
		\sqrt{1-\delta}^\ell.
	\end{equation*}
	Thus, there is an $s \in \{0, 1, \ldots, m\}$ with
	\begin{equation*}
		\norm{\Theta_s}
		\geq
		\frac{\varepsilon^m}{2^{m + 1}}
		\mathcal{N}^\ell
		\sqrt{1- \delta}^\ell
		\geq
		\mathcal{N}^\ell (1- \delta)^\ell.
	\end{equation*}
	We can now construct the wanted sequence $(\eta)$ by setting
	\begin{equation*}
		\eta_i
		=
		\begin{cases}
			\theta_i^s
			&
			1 \leq i \leq m
			\\
			\vartheta_i
			&
			m+1 \leq i \leq \ell
			\\
			\xi_i
			&
			i \geq \ell + 1
		\end{cases}
	\end{equation*}
	and get $(\eta) \in X_{k, \delta}$, since
	\begin{equation*}
		\norm{ \ad_{\eta_1} \circ \cdots \circ \ad_{\eta_\ell}}^{1/\ell}
		=
		\norm{ \Theta_s }^{1/\ell}
		\geq
		(1 - \delta) \mathcal{N}
	\end{equation*}
	and $\ell \geq k$. The choice of $m$ ensures moreover $\halbnorm{d}((\xi), 
	(\eta)) < \varepsilon$ and hence the proof is finished.
\end{proof}

\end{document}

%%% Local Variables:
%%% mode: latex
%%% TeX-master: t
%%% End:
