%
% A new paper...
% Title: Convergence of the Gutt Star Product
% git-repository is gstar
%
% From now on, we proudly use the chairx style file for
% everything. All stuff in there is soo useful!
%


%
% Before we start, let's nag a bit about old latex constructs
% just to learn... This will be removed in the final version
%

\RequirePackage[l2tabu, orthodox]{nag}


%
% we always start with 11pt, draft mode on for easier editing and
% english as default language
%

\documentclass[
11pt,                          % standard font size
%draft,                         % draft or final?
english                        % standard language
]{article}


%
% some macro packages
%

\usepackage[english]{babel}    % with explicit language
\usepackage{amsmath}           % ams mathematical stuff
\usepackage[utf8]{inputenc}    % smart input of funny chars
\usepackage[T1]{fontenc}       % also for the font encoding
\usepackage{longtable}         % tables longer than one page
\usepackage{exscale}           % large summation signs in 11pt
\usepackage[final]{graphicx}   % to include pdf pictures
\usepackage[sort]{cite}        % nicer citations
\usepackage{array}             % nice tables
\usepackage{wasysym}           % smiley symbols
\usepackage[a4paper]{geometry} % geometry of page layout
%\usepackage{gitinfo2}          % include git info: Version 2
%\usepackage[multiuser]{fixme}  % correction notes, warnings etc.
\usepackage{xspace}            % better spacing after macros
\usepackage{tikz}              % for commutative diagrams and stuff
\usepackage{ifdraft}           % to determine whether draft mode
\usepackage{chairx}            % the Chair X style file
\usepackage[expansion=false    % no font expansion
           ]{microtype}        % only protrusion
\usepackage[nottoc]{tocbibind} % refs and index in the toc
\usepackage[backref=page,      % backrefs in the bibliography
           final=true,         % always treat as final
           pdfpagelabels       % use pdf page labels
           ]{hyperref}         % hyperrefs are cool!

%
% Some own macros...
%

%\usepackage{mnsymbol}

\newcommand{\ostar}{\star}
\newcommand{\coproduct}{\Delta}
\newcommand{\ocoproduct}{\Delta}

\newcommand{\Eta}{\mathrm{H}}
\newcommand{\Rho}{\operatorname{P}}
\newcommand{\bch}[2]{\mathrm{BCH}\left(#1, #2\right)}
\newcommand{\bchpart}[3]{\mathrm{BCH}_{#1}\left(#2, #3\right)}
\newcommand{\bchparts}[4]{\mathrm{BCH}_{#1, #2}\left(#3, #4\right)}
\newcommand{\bchtilde}[4]{\widetilde{\mathrm{BCH}}_{#1, #2}\left(#3; #4\right)}
\newcommand\ot[2]{\stackrel{\mathclap{#1}}{#2}}
%
%\mathrel{\overset{\makebox[0pt]
%	{\mbox{\normalfont\footnotesize\sffamily #1}}}{#2}}}


%
% pdf files for graphics in the following directory:
%

\graphicspath{{../tikz/}}


%
% tikz libraries to be loaded, feel free to add more...
%

\usetikzlibrary{matrix}
\usetikzlibrary{arrows}
\usetikzlibrary{patterns}
\usetikzlibrary{decorations.pathreplacing}


%
% page dimensions, scaling etc. Not final yet
%


\geometry{bindingoffset=0cm}
\geometry{hcentering=true}
\geometry{hscale=0.8}
\geometry{vscale=0.8}


%
% check whether draft or not: synctex is soo cool
%

%\ifdraft{\synctex=1}{}


%
% fix me settings
%

%\fxusetheme{color}


%
% Get the authors from external file
% This used in all documents of the paper project
%

%\input{../authors/authors}


%
% own local math macros follow here
%


%
% title page for Convergence of the Gutt Star Product
% authors are included from the authors file
% This has to be adapted in the final version
%

\title{Quasi-Nilpotency in Banach-Lie Algebras}

\author{
  \textbf{Paul Stapor}\thanks{\texttt{paul.stapor@stud-mail.uni-wuerzburg.de}},
  \addtocounter{footnote}{2}
  \\[0.5cm]
  \chairXaddress
}

%\date{Current Version of gstar: \gitAuthorIsoDate\\[0.2cm]
%  {\small
%    Last changes by \gitAuthorName{} on \gitAuthorDate \\
%    Git revision of gstar: \texttt{\gitAbbrevHash{}} \gitReferences
%  }
%}

\date{September 2015}


%
% the text starts here
%

\begin{document}

%
% title page
%

\maketitle

%
% abstract
%

\begin{abstract}
    Nilpotency is a well-studied property in finite-dimensional algebras, but the 
    transition to topological infinite-dimensional algebras is not easy, since there 
    are many inequivalent notions of so called ``quasi-nilpotency'', which all imply 
    true nilpotency in finite dimensions. In this work, we translate some notions 
    from the better known associative theory to Lie algebras, in particular 
    Banach-Lie algebras, clarify their relations among each other and give many 
    examples. We will see that these notions can be linked to possible topologies on 
    the universal enveloping algebra $\algebra{U}(\lie{g})$ of a Banach-Lie algebra 
    $\lie{g}$ and show that one of them is equivalent to the existence of 
    topologies which allow group-like elements in their completions. Finally, we 
    generalize some of these results to locally convex Lie algebras.
\end{abstract}

\newpage


%
% table of contents
%

\tableofcontents
\newpage


%
% Introduction
%

\section{Introduction}
\label{sec:Introduction}



\section{Topologies on Universal Enveloping Algebras}

\subsection{The $\Tensor_R$-topology}

\subsection{Nilpotent locally convex Lie-Algebras}

Bimodules here...



\subsection{$\E$-like Topologies}
We are particularly interested in the question, when a Banach-Lie algebra $\lie{g}$ allows a locally convex topology on $\algebra{U}(\lie{g})$, which has such a big completion that exponential functions and hence group-like elements are part of the completion. This would the case for every locally multiplicatively convex structure, for example and hence especially for a topology induced by a norm. Unfortunately, this would lead to the fact, that we do not find the original topology on $\lie{g}$ in $\algebra{U}(\lie{g})$.
\begin{remark}
	\label{Rem:BCHDesaster}
	Assume for every $\xi, \eta \in \lie{g}$ we would have $\exp(\xi)$ and 
	$\exp(\eta)$ in $\widehat{\algebra{U}(\lie{g})}$. Then of course their product 
	would be also part of the completion. But from 
	\cite{esposito.stapor.waldmann:2015a:pre} we see that in this case
	\begin{equation*}
		\exp(\xi) \cdot \exp(\eta)
		=
		\exp\left(
			\bch{\xi}{\eta}
		\right),
	\end{equation*}
	so for all $\xi, \eta \in \lie{g}$, we could give a sense to the element 
	$\exp\left( \bch{\xi}{\eta} \right)$, although we know that $\bch{\xi}{\eta}$ 
	does not need to exist at all. 
\end{remark}
We now propose a good criterion, which forbids this kind of effects.
\begin{definition}
	Let $\lie{g}$ be a Banach-Lie algebra and $\tau$ a metrizable topology on 
	$\algebra{U}(\lie{g})$. Then we say $\tau$ is an $\E$-like topology, if the 
	three following things hold:
	\begin{definitionlist}
		\item
		The multiplication in $(\algebra{U}(\lie{g}), \tau)$ is continuous.
		
		\item
		For every $\xi \in \lie{g}$, the exponential series $\sum_{n=0}^\infty 
		\frac{\xi^n}{n!}$ converges in the completion of $(\algebra{U}(\lie{g}), 
		\tau)$.
		
		\item
		The topology of $\lie{g}$ as a subspace of  the completion of $(\algebra{U}
		(\lie{g}), \tau)$ is equivalent to the norm topology of $\lie{g}$ as 
		Banach-Lie algebra.
	\end{definitionlist}
	We say that a Banach-Lie algebra is $\E$-like, when its universal enveloping 
	algebra has an $\E$-like topology.
\end{definition}
We now want to see that if a locally convex topology is $\E$-like, then effects 
like in Remark \ref{Rem:BCHDesaster} can not occur. 
\begin{proposition}
	Let $\lie{g}$ be a Banach-Lie algebra and $\algebra{U}(\lie{g})$ has an 
	$\E$-like topology. Then the Baker-Campbell-Hausdorff series converges globally 
	on $\lie{g}$.
\end{proposition}
\begin{proof}
	We use the $\E$-like topology on $\algebra{U}(\lie{g})$.
	Since $\widehat{\algebra{U}(\lie{g})}$ is a Fr\'echet space and $\lie{g}$ is a 
	closed subspace, the can look at the quotient 
	\begin{equation*}
		\algebra{U}'(\lie{g})
		=
		\frac{\widehat{\algebra{U}(\lie{g})}}{\lie{g}},
	\end{equation*}
	which will be a Fr\'echet space again, since $\lie{g}$ is closed and the 
	topology on $\widehat{\algebra{U}(\lie{g})}$ is second countable. We can see 
	$\algebra{U}'(\lie{g})$ as a subspace of $\widehat{\algebra{U}(\lie{g})}$ an 
	get two continuous projections
	\begin{equation*}
		\pi' \colon
		\widehat{\algebra{U}(\lie{g})}
		\longrightarrow
		\algebra{U}'(\lie{g})
		\quad \text{ and }
		\pi \colon
		\widehat{\algebra{U}(\lie{g})}
		\longrightarrow
		\lie{g}
	\end{equation*}
	fulfilling $\pi + \pi' = \id$. So in an $\E$-like topology, we have for $\xi, 
	\eta \in \lie{g}$ and $t,s \in \mathbb{K}$
	\begin{align*}
		\pi \left( \exp(t \xi) \cdot \exp(s \eta) \right)
		& =
		\pi
		\left(
			\lim_{N \rightarrow \infty}
			\left(
				\sum\limits_{n=0}^N
				\frac{t^n \xi^n}{n!}
			\right)
			\cdot
			\lim_{M \rightarrow \infty}
			\left(
				\sum\limits_{m=0}^M
				\frac{s^m \eta^m}{m!}
			\right)
		\right)
		\\
		& \ot{(a)}{=}
		\pi
		\left(
			\lim_{N \rightarrow \infty}
			\lim_{M \rightarrow \infty}
			\left(
				\sum\limits_{n=0}^N
				\frac{t^n \xi^n}{n!}
			\right)
			\cdot
			\left(
				\sum\limits_{m=0}^M
				\frac{s^m \eta^m}{m!}
			\right)
		\right)
		\\
		& \ot{(b)}{=}
		\lim_{N \rightarrow \infty}
		\lim_{M \rightarrow \infty}
		\pi
		\left(	
			\left(
				\sum\limits_{n=0}^N
				\frac{t^n \xi^n}{n!}
			\right)
			\cdot
			\left(
				\sum\limits_{m=0}^M
				\frac{s^m \eta^m}{m!}
			\right)
		\right)
		\\
		& \ot{(c)}{=}
		\lim_{N \rightarrow \infty}
		\lim_{M \rightarrow \infty}
		\sum\limits_{n=0}^{N}
		\sum\limits_{m=0}^{M}
		\bchparts{n}{m}{t \xi}{s \eta}
		\\
		& \ot{(d)}{=}
		\lim_{L \rightarrow \infty}
		\sum\limits_{\ell=0}^{L}
		\sum\limits_{a + b = \ell}
		\bchparts{a}{b}{t \xi}{s \eta}{n}{m}
		\\
		& =
		\lim_{L \rightarrow \infty}
		\sum\limits_{\ell=0}^{L}
		\bchpart{\ell}{t \xi}{s \eta}{n}{m},
	\end{align*}
	where we used the continuity of multiplication in $(a)$ and the one of $\pi$ in 
	$(b)$. In $(c)$ we just evaluated $\pi$ and used the fact that the 
	Baker-Campbell-Hausdorff series is a power series in $(d)$ and hence converges 
	absolutely. So it must converge globally on $\lie{g}$, when the latter is seen 
	as a subspace of $\widehat{\algebra{U}(\lie{g})}$. But since the topology is 
	$\E$-like and therefore equivalent to the one of $\lie{g}$ as Banach-Lie 
	algebra, $\lie{g}$ must be globally BCH.
\end{proof}
From \cite{woj} we see immediately, that being quasi-nilpotent is a necessary 
condition for being $\E$-like for a Banach-Lie algebra. For a finite-dimensional 
Lie algebra, this means that it must be nilpotent, since there quasi-nilpotency 
implies nilpotency. Furthermore, in \cite{stapor:2015a}, it is shown that being 
topologically uniformly nilpotent Banach-Lie algebras are $\E$-like. There and in 
\cite{esposito.stapor.waldmann:2015a:pre}, the so called $\Sym_{1^-}$ was shown to 
be $\E$-like for nilpotent, locally convex Lie algebras.




\section{Quasi-Nilpotency in Banach-Lie Algebras}
\label{sec:QuasiNilpotency}

\subsection{Classes of Quasi-Nilpotent Banach-Lie Algebras}
First, we want to classify, how we can ``measure'' the nilpotency of a Banach-Lie 
algebra. In an associative Banach algebra $\algebra{A}$, the usual way is looking 
at the spectral radii of the elements $a \in \algebra{A}$ or at the joint spectral 
radii of bounded subsets of $\algebra{A}$. Unfortunately, for a Banach-Lie algebra, 
a good notion of the spectral radius of elements is not so obvious. In 
\cite{galindo.palacios:}, Galindo and Palacios constructed something that they 
called the ``Lie-spectral radius''. We want to go a different way. Let $\lie{g}$ be 
a Banach-Lie algebra, then we know that $\lie{g}$ itself is not an associative 
Banach algebra, but the bounded linear operators on it $\Bounded(\lie{g})$ surely 
form such an algebra using the spectral norm. Since the map $\ad \colon \lie{g} 
\longrightarrow \Bounded(\lie{g})$ is a continuous homomorphism of Lie algebras, we 
know that its image forms a closed (and hence complete) subspace of 
$\Bounded(\lie{g})$ by the closed graph theorem. The subalgebra generated by 
$\Im(\ad)$ is now a subalgebra of $\Bounded(\lie{g})$ and hence a normed (but not 
necessarily complete) associative algebra. However, we can now use the notions from 
the associative theory by applying them to this subalgebra of $\Bounded(\lie{g})$.
\begin{definition}
	\label{Def:Nilpotencies1}
	Let $\lie{g}$ be a Banach-Lie algebra in which the Lie bracket fulfils the 
	estimate
	\begin{equation*}
		\norm{ [\xi, \eta] }
		\leq
		\norm{\xi}
		\norm{\eta}.
	\end{equation*}
	Denote by $\mathbb{B}_1(0)$ all elements $\xi \in \lie{g}$ with 
	$\norm{\xi} = 1$. We say that
	\begin{definitionlist}
		\item
		$\lie{g}$ is topologically nil (or radical, or quasi-nilpotent, or 
		(quasi-nilpotent) of type $(i)$), if every $\xi \in \lie{g}$ is 
		quasi-nilpotent, i.e.
		\begin{equation*}
			\lim_{n \longrightarrow \infty}
			\norm{\ad_{\xi}^n}^{\frac{1}{n}}
			=
			0.
		\end{equation*}
		
		\item
		$\lie{g}$ is uniformly topologically nil or (quasi-nilpotent) of type 
		$(ii)$, if
		\begin{equation*}
			\lim_{n \longrightarrow \infty}
			\mathcal{N}_1(n)
			=
			0
		\end{equation*}
		for
		\begin{equation}
			\mathcal{N}_1(n)
			=
			\sup \left\{ 
			\left.
				\norm{ \ad_{\xi}^n}^{\frac{1}{n}} 
			\right|
				\xi \in \mathbb{B}_1(0)
			\right\}.
		\end{equation}
		
		\item
		$\lie{g}$ is topologically nilpotent or (quasi-nilpotent) of type 
		$(iii)$, if for every sequence
		$(\xi_n)_{n \in \mathbb{N}} \subset \mathbb{B}_1(0)$ we have
		\begin{equation*}
			\lim_{n \longrightarrow \infty}
			\norm{ 
				\ad_{\xi_1} \circ \ldots \circ \ad_{\xi_n}
			}^{\frac{1}{n}}
			=
			0.
		\end{equation*}
		
		\item
		$\lie{g}$ is uniformly topologically nilpotent or (quasi-nilpotent) of 
		type $(iv)$, if
		\begin{equation*}
			\lim_{n \longrightarrow \infty}
			\mathcal{N}(n)
			=
			0
		\end{equation*}
		for
		\begin{equation}
			\mathcal{N}(n)
			=
			\sup \left\{ 
			\left.
				\norm{ 
					\ad_{\xi_1} \circ \ldots \circ \ad_{\xi_n}
				}^{\frac{1}{n}} 
			\right|
				\xi_1, \ldots, \xi_n \in \mathbb{B}_1(0)
			\right\}.
		\end{equation}
	\end{definitionlist}
\end{definition}
This definition is directly taken from \cite{muller}, just using the little adaptations we explained before. We will need one more notion, which will be 
turn out to be very useful later.
\begin{definition}[Symmetrically topologically nil Banach-Lie algebras]
	\label{Def:Nilpotencies2}
	Let $\lie{g}$ be a Banach-Lie algebra like in Definition 
	\ref{Def:Nilpotencies1}, then we say that $\lie{g}$ is symmetrically
	topologically nil or (quasi-nilpotent) of type $(ii')$, if
	\begin{equation*}
		\lim_{n \longrightarrow \infty}
		\mathcal{S}(n)
		=
		0
	\end{equation*}
	for
	\begin{equation}
		\mathcal{S}(n)
		=
		\sup \left\{ 
		\left.
			\left\Vert
				\frac{1}{n!}
				\sum\limits_{\sigma \in S_n}
				\ad_{\xi_{\sigma(1)}} \circ \ldots \circ \ad_{\xi_{\sigma(n)}}
			\right\Vert^{\frac{1}{n}} 
		\right|
			\xi_1, \ldots, \xi_n \in \mathbb{B}_1(0)
		\right\}.
	\end{equation}
\end{definition}
It is immediate to see the following implications:
\begin{equation*}
	(iv) \Longrightarrow 
	(ii') \Longrightarrow 
	(ii) \Longrightarrow 
	(i)
	\quad \text{ and } \quad
	(iv) \Longrightarrow
	(iii) \Longrightarrow 
	(i),
\end{equation*}
whereas the relations between $(iii)$, $(ii)$ and $(ii')$ are not clear. In the 
associative case, we find $(iii) \Longrightarrow (iv)$. It is also fair to ask 
whether $(ii) \Longrightarrow (ii')$.

\subsection{Examples}


\section{Topologically nilpotent Banach-Lie algebras}


\section{Uniformly quasi-nilpotent Banach-Lie algebras are $\E$-like}

We have everything at the hand to state or main Theorem now:
\begin{theorem}
	\label{Thm:Main}
	A Banach-Lie algebra is $\E$-like if and only if it is uniformly 
	quasi-nilpotent.
\end{theorem}
Since it is rather long and technical, we will split up the proof into two parts.


\subsection{$\E$-like Banach Lie Algebras are of Type $II$}
It is rather easy to show, that being of type $(ii)$ is really necessary to be 
$\E$-like.
\begin{proposition}
	Let $\lie{g}$ be a Banach-Lie algebra, which is not of type $(ii)$. Then it is 
	not $\E$-like.
\end{proposition}
\begin{proof}
	The idea is to construct two sequences which converge to zero, but whose 
	product does not converge: Being not of type $(ii)$ means, that there is an
	$\varepsilon > 0$, such that $\lim_{n \rightarrow \infty} \mathcal{N}_1(n) > 
	\varepsilon$. This means that for every $n \in \mathbb{N}$, there is a sequence 
	$\left(\xi_{n, k} \right)_{k \in \mathbb{N}} \subset \mathbb{B}_1(0)$, 
	such that
	\begin{equation*}
		\lim_{k \longrightarrow \infty}
		\left\Vert
			(\ad_{\xi_{n, k}})^n
		\right\Vert^{\frac{1}{n}}
		=
		\mathcal{N}_1(n).
	\end{equation*}
	So for every $n, k \in \mathbb{N}$ we have
	\begin{equation*}
		\left\Vert
			(\ad_{\xi_{n, k}})^n
		\right\Vert^{\frac{1}{n}}
		> \varepsilon.
	\end{equation*}
	This means that we find for every element $\xi_{n,n}$ a sequence 
	$(\alpha_{n, \ell})_{\ell \in \mathbb{N}}$ with $\norm{\alpha_{n, \ell}} = 1$ 
	for all $n, \ell \in \mathbb{N}$ and
	\begin{equation*}
		\lim_{\ell \longrightarrow \infty}
		(\ad_{\xi_{n,n}})^n(\alpha_{n,\ell})
		> 
		\varepsilon^n.
	\end{equation*}
	We define a new sequence $(\eta_n)_{n \in \mathbb{N}}$ with 
	$\eta_n = \alpha_{n, \ell}$ where $\ell$ is chosen big enough to 
	fulfil $(\ad_{\xi_{n,n}})^n(\alpha_{n,\ell}) > \varepsilon^n$.
	Now let's assume we have an $\E$-like topology on $\algebra{U}(\lie{g})$, 
	defined by a countable set of seminorms and we denote by $\algebra{P}$ the set 
	of all continuous seminorms. Then we have for every $\xi \in \lie{g}$ and every 
	$p \in \algebra{P}$
	\begin{equation*}
		p \left( \exp(\xi) \right)
		=
		p \left(
			\sum\limits_{n=0}^\infty
			\frac{\xi^n}{n!}
		\right)
		\leq
		\sum\limits_{n=0}^\infty
		\frac{p\left( \xi^n \right)}{n!}
		<
		\infty,
	\end{equation*}
	since the power series converges absolutely. Since the sequence $(\xi_{n,n})_{n 
	\in \mathbb{N}}$ is bounded in $\lie{g}$, it is also bounded in $\algebra{U}
	(\lie{g})$. Hence for every $t > 0$, the sequence $(\exp(t \xi_{n,n}))_{n \in 
	\mathbb{N}}$ is bounded by the continuity of the exponential function and the 
	sequence
	\begin{equation*}
		\left(
			\frac{t^n}{n!}
			p\left(
				\xi_{n,n}^n
			\right)
		\right)_{n \in \mathbb{N}}
	\end{equation*}
	converges to zero (in $\lie{g}$ and in $\algebra{U}(\lie{g})$). The series 
	$(\frac{1}{n} \eta_n)_{n \in \mathbb{n}}$ also converges to zero, again in 
	$\lie{g}$ and $\algebra{U}(\lie{g})$.
	Now we want to show that their product does \emph{not} converge to zero
	to get a contradiction to the continuity of the multiplication. Therefore
	we need again the projection map $\pi_1 \colon \algebra{U}(\lie{g}) 
	\longrightarrow \lie{g}$ which is linear and continuous and should hence not 
	spoil the convergence. For every fixed $t > 0$, we have
	\begin{align*}
		p \left(
			\pi_1\left(
				\frac{t^n \xi_{n,n}^n}{n!}
				\cdot
				\eta_n
			\right)
		\right)
		&=
		p\left(
			\pi_1 \left(
				\frac{1}{n!}
				\sum\limits_{k=0}^n
				\binom{n}{k}
				B_k^* t^n
				\xi_{n,n}^{\vee(k-n)}
				\vee
				(\ad_{\xi_{n,n}})^k(\eta_n)
			\right)
		\right)
		\\
		& =
		p\left(
			\frac{B_n^* t^n}{n!}
			(\ad_{\xi_{n,n}})^n(\eta_n)
		\right)
		\\
		& >
		\frac{|B_n|^* t^n \varepsilon^n}{n!}.
	\end{align*}
	But we know that this series does not converge to zero, if we choose for 
	example $t = \frac{3 \pi}{\varepsilon}$, which is allowed. Hence we have a 
	contradiction.
\end{proof}


\subsection{Type $II'$ quasi-nilpotent Banach-Lie algebras are $\E$-like}

For technical reasons, the reverse direction is split into two parts.

\subsubsection{Part I}
\begin{proposition}
	Let $\lie{g}$ be a uniformly quasi-nilpotent Banach-Lie algebra and assume that 
	there is a $p \geq 1$, such that $\lim_{n \rightarrow \infty} 
	\frac{\mathcal{N}_1(n)}{\sqrt[p]{n!}} = 0$. Then there is an $\E$-like topology 
	on $\algebra{U}(\lie{g})$.
\end{proposition}
\begin{proof}
	The proof is constructive. We look at the situation in the symmetric tensor 
	algebra and see that the product is continuous in $\Sym_{1^-}^{\bullet}
	(\lie{g})$ by the following argumentation. Let's denote the norm on $\lie{g}$ 
	for simplicity by $p$. Since $\lim_{n \rightarrow \infty} 
	\frac{\mathcal{N}_1(n)}{\sqrt[p]{n}} = 0$, we know there is a $p \geq 1$ and a 
	$c > 0$, such that $\mathcal{S}(n) \leq \frac{c}{\sqrt[p]{n}}$. Take $\xi_1, 
	\ldots, \xi_k, \eta \in \lie{g}$, then we find for some $\varepsilon > 0$ with 
	$R > 1 - \frac{1}{p} + \varepsilon$
	\begin{align*}
		p_R \big(
		&
			\xi_1 \tensor \cdots \tensor \xi_k
			\star
			\eta
		\big)
		\\
		&=
		\sum\limits_{n = 0}^k
		\frac{|B_n^*|}{k!} \binom{k}{n}
		(k + 1 - n)!^R
		p^{k+1-n}
		\left(
			\sum	\limits_{\sigma \in S_k}
			\xi_{\sigma(1)} \cdots \xi_{\sigma(k-n)}
			\cdot
			\left( 
				\ad_{\xi_{\sigma(k-n+1)}} 
				\circ \cdots \circ
				\ad_{\xi_{\sigma(k)}}
			\right)
			(\eta)
		\right)
		\\
		& \leq
		(k + 1)^R
		\sum\limits_{n = 0}^k
		\frac{|B_n^*|}{n!}
		\frac{1}{(k-n)!^{1 - R}}
		p^{k+1-n}
		\Bigg(
			\sum\limits_{
				I = \{i_1, \ldots, i_n\} \subset \{1, \ldots, k\}
			}
			(n-k)!
			\xi_1 \cdots \widehat{\xi_I} \cdots \xi_k
		\\
		& \quad \cdot
			\sum	\limits_{\sigma \in S_n}
			\left( 
				\ad_{\xi_{\sigma(1)}} 
				\circ \cdots \circ
				\ad_{\xi_{\sigma(n)}}
			\right)
			(\eta)
		\Bigg)
		\\
		& \leq
		(k + 1)^R
		\sum\limits_{n = 0}^k
		\frac{|B_n^*|}{n!}
		\frac{(k-n)!}{(k-n)!^{1 - R}}
		\sum\limits_{
			I = \{i_1, \ldots, i_n\} \subset \{1, \ldots, k\}
		}
		p(\xi_1) \cdots \widehat{p(\xi_I)} \cdots p(\xi_k)
		\\
		& \quad \cdot
		p \left(
			\sum	\limits_{\sigma \in S_n}
			\left( 
				\ad_{\xi_{\sigma(1)}} 
				\circ \cdots \circ
				\ad_{\xi_{\sigma(n)}}
			\right)
			(\eta)
		\right)
		\\
		& \leq
		(k + 1)^R
		\sum\limits_{n = 0}^k
		\frac{|B_n^*|}{n!}
		\frac{(k-n)!}{(k-n)!^{1 - R}}
		\binom{k}{n}
		\frac{c^n n!}{n!^{\frac{1}{p}}}
		p(\xi_1) \cdots p(\xi_k) p(\eta)
		\\
		& \leq
		(k + 1)^R
		\sum\limits_{n = 0}^k
		\frac{c^n |B_n^*|}{n!^{1 + \varepsilon}}
		\binom{k}{n}^{1-R}
		k!^R
		p(\xi_1) \cdots p(\xi_k) p(\eta)
		\\
		& \leq
		(k + 1)^R
		\underbrace{
			2^{k (1-R)}
		}_{
			=\kappa_1 k!^{\frac{1-R}{2}}
		}
		\underbrace{
			\sum\limits_{n = 0}^k
			\frac{c^n |B_n^*|}{n!^{1 + \varepsilon}}
		}_{
			= \kappa_2
		}
		k!^R
		p^k(\xi_1 \tensor \cdots \tensor \xi_k) p(\eta)
		\\
		& \leq
		(k-1)^R
		\kappa
		p_{R + \frac{1-R}{2}}
		(\xi_1 \tensor \cdots \tensor \xi_k)
		p(\eta).
	\end{align*}
	In the last step, we took $\kappa = \kappa_1 \kappa_2$. We can extend this 
	estimate to any arbitrary tensor of degree at most $k$ on the left hand side.
	Now we do the next step and take $\xi_1, \ldots, \xi_k, \eta_1, \ldots, 
	\eta_\ell \in \lie{g}$. We can iterate this estimate to get
	\begin{align*}
		p_R \big(
			\xi_1 \tensor \cdots \tensor \xi_k
		&
			\star
			\eta_1 \tensor \cdots \tensor \eta_\ell
		\big)
		\\
		& \leq
		(k + \ell)^R (k + \ell - 1)^{1 - \frac{1 - R}{2}}
		\cdots (k+1)^{{1 - \frac{1 - R}{2^{\ell - 1}}}}
		\kappa^\ell
		p_{1 - \frac{1-R}{2^\ell}}
		\left( \xi_1 \tensor \cdots \tensor \xi_k \right)
		p(\eta_1) \cdots p(\eta_\ell)
		\\
		& \leq
		\left( \frac{(k+\ell)!}{\ell!} \right)^{\frac{1 + R}{2}}
		\kappa^\ell
		p_1 \left( \xi_1 \tensor \cdots \tensor \xi_k \right)
		p^\ell \left( \eta_1 \tensor \cdots \tensor \eta_\ell \right)
		\\
		& \leq
		2^{k + \ell}
		\ell!^{\frac{1 + R}{2}}
		\kappa^\ell
		p_1 \left( \xi_1 \tensor \cdots \tensor \xi_k \right)
		p^\ell \left( \eta_1 \tensor \cdots \tensor \eta_\ell \right)
		\\
		&=
		(2p)_1 \left( \xi_1 \tensor \cdots \tensor \xi_k \right)
		(2 \kappa p)_{\frac{1 + R}{2}} 
		\left( \eta_1 \tensor \cdots \tensor \eta_\ell \right).
	\end{align*}
	Finally we can assume without loss of generality that $\ell \geq k$, since
	we could do all estimates also for the right hand side in the same way.
	By setting $R' = \frac{3}{4}(1-R)$, we find
	\begin{equation*}
		p_R \big(
			\xi_1 \tensor \cdots \tensor \xi_k
			\star
			\eta_1 \tensor \cdots \tensor \eta_\ell
		\big)
		\leq
		(2p)_{R'}
		\left( \xi_1 \tensor \cdots \tensor \xi_k \right)
		(2 \kappa p)_{R'}
		\left( \eta_1 \tensor \cdots \tensor \eta_\ell \right)
	\end{equation*}
	and the proposition is proven.
\end{proof}


\subsubsection{Part II}
Now we need to cover the last possible case: A uniformly quasi-nilpotent Banach-Lie 
algebra, which has a characteristic sequence, which decreases slower than any power 
of $n!$. An example for something like this would be $\mathcal{N}_1(n) = 
(\log(n) + 1)^{-n}$ for $n \in \mathbb{N}$.
\begin{proposition}
	Let $\lie{g}$ be a uniformly quasi-nilpotent Banach-Lie algebra and assume that 
	for all $p \geq 1$, we have $\lim_{n \rightarrow \infty} 
	\frac{\mathcal{N}_1(n)}{\sqrt[p]{n!}} = 0$. Then there is an $\E$-like topology 
	on $\algebra{U}(\lie{g})$.
\end{proposition}
\begin{proof}
	Now the idea is to use increasing sequences as counterweights and to modify the 
	$\Tensor_{1^-}$-topology in this way. Therefore, we need 
	to define a series $(\widetilde{\alpha}_n)_{n \in \mathbb{N}}$ with
	\begin{equation*}
		\widetilde{\alpha}
		=
		\log \left(
			\mathcal{N}_1(n)^{-1}
		\right).
	\end{equation*}
	Since $\mathcal{N}_1(n)$ is monotonously decreasing as $\norm{[\xi, \eta]} \leq 
	\norm{\xi} \norm{\eta}$, $\widetilde{\alpha}_n$ will be monotonously increasing.
	It is easy to see, that we can define a new series $(\alpha_n)_{n \in 
	\mathbb{N}}$, which fulfils $\alpha_n \leq \widetilde{\alpha}_n$ for all $n \in 
	\mathbb{N}$ and which is ``convex'', that means for all $n, m \in \mathbb{N}$, 
	$n \leq m$ and $t \in \mathbb{N}$ with $n \leq t \leq m$, we have
	\begin{equation*}
		\alpha_t 
		\leq 
		\alpha_n \left( 1 - \frac{t - n}{m - n} \right) + 
		\alpha_m \frac{t - n}{m - n}.
	\end{equation*}
	We can define a convex, smooth and monotonously increasing function
	\begin{equation*}
		f \colon
		\mathbb{R}_0^+
		\longrightarrow
		\mathbb{R}^+
		, \quad
		\forall_{n \in \mathbb{N}}
		\colon
		f(n) 
		=
		\alpha_n,
	\end{equation*}
	which must fulfil in addition
	\begin{equation*}
		\forall_{p \geq 1}
		\colon
		\lim_{x \rightarrow \infty}
		\frac{p f(x)}{x \log(x)}
		=
		0
		\quad \text{ and } \quad
		\lim_{x \rightarrow \infty}
		\frac{f(x)}{x}
		=
		\infty
	\end{equation*}
	since $\mathcal{N}_1(n)$ decreases more slowly than any power of the factorials, 
	but $\lie{g}$ is still uniformly quasi-nilpotent.
	
	We will need an analogue of the estimate $\binom{n}{m} \leq 2^n$ for the series 
	$\alpha$. This will be achieved by the following lemma.
	\begin{lemma}
		Let $(\alpha_n)_{n \mathbb{N}}$ be a monotonously increasing, convex 
		sequence, such that
		\begin{equation*}
			\lim_{n \rightarrow \infty}
			\frac{\alpha_n}{n \log(n)}
			=
			0
			\quad \text{ and } \quad
			\lim_{n \rightarrow \infty}
			\frac{\alpha_n}{n}
			=
			\infty.
		\end{equation*}
		Then there are constants $b, c \geq 0$, such that for every $n,m \in 
		\mathbb{N}$ with $m \leq n$ the following inequality holds:
		\begin{equation}
		\label{Lemma:LogBinomialEstimate}
			\alpha_n - \alpha_m - \alpha_{n-m}
			\leq
			b + c n.
		\end{equation}
	\end{lemma}
	\begin{subproof}
		This is a logarithmic version of the estimate for the binomial coefficient.
		First note that we can assign to the sequence $\alpha$ a smooth, 
		monotonously increasing and convex function $f$ as we did before.
		Note that $\lim_{x \rightarrow \infty} f'(x) = \infty$, since $f$ is convex 
		and $\lim_{x \rightarrow \infty} \frac{f(x)}{x} = \infty$. Now we can use
		the theorem of de l'H\^{o}spital to get
		\begin{equation*}
			0 
			=
			\lim_{x \rightarrow \infty}
			\frac{f(x)}{x \log(x)}
			=
			\lim_{x \rightarrow \infty}
			\frac{f'(x)}{\log(x) + 1}
			=
			\lim_{x \rightarrow \infty}
			x f''(x)
		\end{equation*}
		and hence the is a $c \geq 0$ with $f''(x) \leq \frac{c}{x}$. 
		Now assume $z = x + y$ for positive real numbers $x,y,z$. By Taylor's 
		theorem, we find
		\begin{align*}
			\tag{I}
			f(z)
			&=
			f(x) + y f'(x)
			+ \int\limits_{x}^{z} (z - t) f''(t) dt
			\\
			\tag{II}
			&=
			f(y) + x f'(y)
			+ \int\limits_{y}^{z} (z - t) f''(t) dt.
		\end{align*}
		For reasons of convexity, the function
		\begin{equation*}
			g(z,x)
			=
			g(z) - g(x) - g(z-x)
		\end{equation*}
		is maximal for fixed $z \in \mathbb{R}^+$ if $2x = z$. So we have proven
		Equation \eqref{Lemma:LogBinomialEstimate}, if we can show 
		$f(2x) - 2f(x) \leq cx + d$. Using (I) for $x = y$, we find
		\begin{equation*}
			f(2x)
			=
			f(x) - x f'(x) +
			\int\limits_x^{2x} (2x-t) f''(t) dt
			\quad
			\Longleftrightarrow
			\quad
			f(2x) - 2 f(x)
			=
			x f'(x) - f(x) +
			\int\limits_x^{2x} (2x-t) f''(t) dt.
		\end{equation*}
		Now we use a backward Taylor development to find
		\begin{align*}
			&&
			f(0)
			&=
			f(x) - xf'(x) + \int_x^0 (0-t) f''(t) dt
			\\
			& \Longleftrightarrow &
			xf'(x) - f(x)
			&=
			\int_0^x t f''(t) dt - f(0)
			\\&&
			& \leq
			\int_0^x t f''(t) dt
			\\&&
			& \leq
			\int_0^x t \frac{c}{t} dt
			\\&&
			&=
			cx.
		\end{align*}
		We then get
		\begin{align*}
			f(2x) - 2 f(x)
			& \leq
			cx + \int\limits_x^{2x} (2x-t) f''(t) dt
			\\
			& \leq
			cx + 
			\int\limits_x^{2x} (2x-t) f\frac{c}{t} dt
			\\
			&=
			cx + cx (\log(4) - 1)
			\\
			&=
			c \log(4) x
		\end{align*}
	\end{subproof}
	Now we can start doing the estimation. Let's define by $(\omega_n)$ a convex, 
	monotonously increasing series, which is smaller equal than the inverse of 
	$\max{\mathcal{N}_1(n), 2}$, but still growing faster than any exponential. This 
	is exactly, what we did before. Define a series $(\alpha)$ by $\alpha_n = 
	\sqrt{\omega_n}$. We will use $\alpha^{\frac{1}{2^m}}$ as counterweights for $m 
	\in \mathbb{N}_0$, in order to get a Fr\'echet topology.
	By doing analogous computations to the first part, we arrive at
	\begin{align*}
		p_\alpha \big(
		&
			\xi_1 \tensor \cdots \tensor \xi_k
			\star_z
			\eta
		\big)
		=
		(k+1) 
		p^k(\xi_1 \tensor \cdots \tensor \xi_k)
		p(\eta)
		\sum\limits_{n=0}^k
		\frac{|B_n^*| |z|^n}{n!}
		\frac{1}{\alpha_{k+1-n} \omega_n}
		\\
		& \leq
		(k+1) 
		p^k(\xi_1 \tensor \cdots \tensor \xi_k)
		p(\eta)
		\frac{1}{\alpha_{k+1}}
		\sum\limits_{n=0}^k
		\frac{|B_n^*| |z|^n}{n! \alpha_n}
		\frac{\alpha_{k+1}}{\alpha_{k+1-n} \alpha_n}
		\\
		& \leq
		(k+1) 
		p^k(\xi_1 \tensor \cdots \tensor \xi_k)
		p(\eta)
		\frac{1}{\alpha_{k+1}}
		\left(
			\sum\limits_{n=0}^k
			\frac{|B_n^*| |z|^n}{n! \alpha_n}
		\right)
		\left(
			\sum\limits_{n=0}^k
			\frac{\alpha_{k+1}}{\alpha_{k+1-n} \alpha_n}
		\right)
		\\
		& \leq
		(k+1) 
		p^k(\xi_1 \tensor \cdots \tensor \xi_k)
		p(\eta)
		\frac{1}{\alpha_{k+1}}
		\kappa_1
		c^{k+1}
		\\
		& \leq
		(k+1) 
		p_{\alpha'}(\xi_1 \tensor \cdots \tensor \xi_k)
		p(\eta)
		\frac{\sqrt{\alpha_k}}{\alpha_{k+1}^{3/4}}
		\kappa_1
		\kappa_2
		\\
		&=
		\kappa
		(k+1)
		\frac{1}{\sqrt[4]{\alpha_{k+1}}}
		\frac{\sqrt{\alpha_k}}{\sqrt{\alpha_{k+1}}}
		p_{\alpha'}(\xi_1 \tensor \cdots \tensor \xi_k)
		p(\eta).
	\end{align*}
	We use the notation $\alpha' = \sqrt{\alpha}$ and $\alpha^{(n)} = 
	\alpha^{\frac{1}{2^n}}$ to keep things readable. We find the estimate on all 
	tensors of degree at most $k$ by the infimum argument and can iterate to higher 
	tensors on the right hand side now:
	\begin{align*}
		p_\alpha \big(
		&
			\xi_1 \tensor \cdots \tensor \xi_k
			\star_z
			\eta_1 \tensor \cdots \tensor \eta_\ell
		\big)
		\\
		& \leq
		\kappa^\ell
		\frac{(k + \ell)!}{k!}
		\frac{\alpha_{k + \ell - 1}'}{\alpha_{k + \ell}' \alpha_{k+\ell}''}
		\frac{\alpha_{k + \ell - 2}''}
		{\alpha_{k + \ell - 1}'' \alpha_{k+\ell-1}^{(3)}}
		\cdots
		\frac{\alpha_{k}^{(\ell)}}
		{\alpha_{k + 1}^{(\ell)} \alpha_{k+1}^{(\ell + 1)}}
		p_{\alpha^{(\ell)}} (\xi_1 \tensor \cdots \tensor \xi_k)
		p^\ell ( \eta_1 \tensor \cdots \tensor \eta_\ell )
		\\
		& \leq
		\kappa^\ell
		2^{k + \ell}
		\ell!
		\frac{1}{\alpha_{k + \ell}'}
		k!
		p^k (\xi_1 \tensor \cdots \tensor \xi_k)
		p^\ell ( \eta_1 \tensor \cdots \tensor \eta_\ell )
		\\
		& \leq
		\kappa^\ell
		2^{k + \ell}
		\frac{k! \ell!}{\alpha_k' \alpha_\ell '}
		p^k (\xi_1 \tensor \cdots \tensor \xi_k)
		p^\ell ( \eta_1 \tensor \cdots \tensor \eta_\ell )
		\\
		&=
		(2p)_{\alpha'} (\xi_1 \tensor \cdots \tensor \xi_k)
		(2 \kappa p)_{\alpha'} ( \eta_1 \tensor \cdots \tensor \eta_\ell ).
	\end{align*}
	This, together with the infimum argument, finishes the proof, since it is clear 
	from the way we can describe the completion of $\left( \Sym^{\bullet}(\lie{g}), 
	\star_z \right)$, that exponential functions are part of it. The continuity of 
	the grading maps is also clear from the construction. Hence the topology is 
	$\E$-like.
\end{proof}



\section{An Outlook to possible Generalizations and open Questions}


\end{document}

%%% Local Variables:
%%% mode: latex
%%% TeX-master: t
%%% End:
