%
% A new paper...
% Title: Convergence of the Gutt Star Product
% git-repository is gstar
%
% From now on, we proudly use the chairx style file for
% everything. All stuff in there is soo useful!
%


%
% Before we start, let's nag a bit about old latex constructs
% just to learn... This will be removed in the final version
%

\RequirePackage[l2tabu, orthodox]{nag}


%
% we always start with 11pt, draft mode on for easier editing and
% english as default language
%

\documentclass[
11pt,                          % standard font size
%draft,                         % draft or final?
english                        % standard language
]{article}


%
% some macro packages
%

\usepackage[english]{babel}    % with explicit language
\usepackage{amsmath}           % ams mathematical stuff
\usepackage[utf8]{inputenc}    % smart input of funny chars
\usepackage[T1]{fontenc}       % also for the font encoding
\usepackage{longtable}         % tables longer than one page
\usepackage{exscale}           % large summation signs in 11pt
\usepackage[final]{graphicx}   % to include pdf pictures
\usepackage[sort]{cite}        % nicer citations
\usepackage{array}             % nice tables
\usepackage{wasysym}           % smiley symbols
\usepackage[a4paper]{geometry} % geometry of page layout
%\usepackage{gitinfo2}          % include git info: Version 2
%\usepackage[multiuser]{fixme}  % correction notes, warnings etc.
\usepackage{xspace}            % better spacing after macros
\usepackage{tikz}              % for commutative diagrams and stuff
\usepackage{ifdraft}           % to determine whether draft mode
\usepackage{chairx}            % the Chair X style file
\usepackage[expansion=false    % no font expansion
           ]{microtype}        % only protrusion
\usepackage[nottoc]{tocbibind} % refs and index in the toc
\usepackage[backref=page,      % backrefs in the bibliography
           final=true,         % always treat as final
           pdfpagelabels       % use pdf page labels
           ]{hyperref}         % hyperrefs are cool!

%
% Some own macros...
%

%\usepackage{mnsymbol}

\newcommand{\ostar}{\star}
\newcommand{\coproduct}{\Delta}
\newcommand{\ocoproduct}{\Delta}

\newcommand{\Eta}{\mathrm{H}}
\newcommand{\Rho}{\operatorname{P}}
\newcommand{\bch}[2]{\mathrm{BCH}\left(#1, #2\right)}
\newcommand{\bchpart}[3]{\mathrm{BCH}_{#1}\left(#2, #3\right)}
\newcommand{\bchparts}[4]{\mathrm{BCH}_{#1, #2}\left(#3, #4\right)}
\newcommand{\bchtilde}[4]{\widetilde{\mathrm{BCH}}_{#1, #2}\left(#3; #4\right)}
\newcommand\ot[2]{\stackrel{\mathclap{#1}}{#2}}
%
%\mathrel{\overset{\makebox[0pt]
%	{\mbox{\normalfont\footnotesize\sffamily #1}}}{#2}}}


%
% pdf files for graphics in the following directory:
%

\graphicspath{{../tikz/}}


%
% tikz libraries to be loaded, feel free to add more...
%

\usetikzlibrary{matrix}
\usetikzlibrary{arrows}
\usetikzlibrary{patterns}
\usetikzlibrary{decorations.pathreplacing}


%
% page dimensions, scaling etc. Not final yet
%


\geometry{bindingoffset=0cm}
\geometry{hcentering=true}
\geometry{hscale=0.8}
\geometry{vscale=0.8}


%
% check whether draft or not: synctex is soo cool
%

%\ifdraft{\synctex=1}{}


%
% fix me settings
%

%\fxusetheme{color}


%
% Get the authors from external file
% This used in all documents of the paper project
%

%\input{../authors/authors}


%
% own local math macros follow here
%


%
% title page for Convergence of the Gutt Star Product
% authors are included from the authors file
% This has to be adapted in the final version
%

\title{Quasi-Nilpotency in Banach-Lie Algebras}

\author{
  \textbf{Paul Stapor}\thanks{\texttt{paul.stapor@stud-mail.uni-wuerzburg.de}},
  \addtocounter{footnote}{2}
  \\[0.5cm]
  \chairXaddress
}

%\date{Current Version of gstar: \gitAuthorIsoDate\\[0.2cm]
%  {\small
%    Last changes by \gitAuthorName{} on \gitAuthorDate \\
%    Git revision of gstar: \texttt{\gitAbbrevHash{}} \gitReferences
%  }
%}

\date{September 2015}


%
% the text starts here
%

\begin{document}

%
% title page
%

\maketitle

%
% abstract
%

\begin{abstract}
    Nilpotency is a well-studied property in finite-dimensional algebras, but the 
    transition to topological infinite-dimensional algebras is not easy, since there 
    are many inequivalent notions of so called ``quasi-nilpotency'', which all imply 
    true nilpotency in finite dimensions. In this work, we translate some notions 
    from the better known associative theory to Lie algebras, in particular 
    Banach-Lie algebras, clarify their relations among each other and give many 
    examples. We will see that these notions can be linked to possible topologies on 
    the universal enveloping algebra $\algebra{U}(\lie{g})$ of a Banach-Lie algebra 
    $\lie{g}$ and show that one of them is equivalent to the existence of 
    topologies which allow group-like elements in their completions. Finally, we 
    generalize some of these results to locally convex Lie algebras.
\end{abstract}

\newpage


%
% table of contents
%

\tableofcontents
\newpage


%
% Introduction
%

\section{Introduction}
\label{sec:Introduction}



\section{A Topology on the Universal Enveloping Algebras}

\subsection{Preliminaries and the $\Tensor_R$-Topology}
As already mentioned, one possibility to see the universal enveloping algebra 
$\algebra{U}(\lie{g})$ of a Lie algebra $\lie{g}$ is the Symmetric tensor algebra 
$\Sym^{\bullet}(\lie{g}) = \bigoplus_{n=0}^\infty \Sym^n(\lie{g})$ via the 
Poincar\'e-Birkhoff-Witt theorem. One uses the isomorphism
\begin{equation}
	\label{eq:pbw-isomorphism}
	\sigma
	=
	\sum\limits_{n=0}^\infty
	\sigma_n
	\quad \text{ with }
	\sigma_n
	\colon
	\Sym^n(\lie{g})
	\longrightarrow
	\algebra{U}(\lie{g})
	, \
	\xi_1 \vee \cdots \vee \xi_n
	\longmapsto
	\frac{1}{n!}
	\sum\limits_{\tau \in S_n}
	\xi_{\tau(1)} \cdots \xi_{\tau(n)}
\end{equation}
to pull back the product from $\algebra{U}(\lie{g})$ to $\Sym^{\bullet}(\lie{g})$. In 
detail, the construction is done using a parameter $z \in \mathbb{K}$ where 
$\mathbb{K} = \mathbb{R}$ or $\mathbb{C}$ and the canonical projections $\pi_n \colon 
\Sym^{\bullet}(\lie{g}) \longrightarrow \Sym^n(\lie{g})$ from the grading:
\begin{equation}
	\label{eq:starproduct}
	\star_z
	\colon
	\Sym^k(\lie{g})
	\times
	\Sym^\ell(\lie{g})
	\longrightarrow
	\Sym^{\bullet}(\lie{g})
	, \quad
	(a, b)
	\longmapsto
	\sum\limits_{n=0}^{k + \ell - 1}
	z^n
	\left( \pi_{k + \ell - n} \circ \sigma^{-1} \right)
	\left(
		\sigma(a)
		\cdot
		\sigma(b)
	\right).
\end{equation}
Then it is extended bilinearly to the whole symmetric algebra. This associative 
product also sometimes called Gutt star product.
The parameter $z$ is motivated from deformation theory and allows to treat the 
symmetric and the universal enveloping algebra at the same time, since the former is 
obtained putting $z = 0$ and the latter setting $z = 1$. In this way, also the 
properties of the product can be studied by letting vary the parameter $z$. We will 
keep this parameter in our equations for the rest of this paper, since it allows a 
more general point of view. The same approach was used in 
\cite{esposito.stapor.waldmann:2015a:pre}. Since one aim of this paper is to extend 
this approach and to prove a link between possible topologies on $\algebra{U}
(\lie{g})$ and the properties of the Lie bracket of $\lie{g}$, we will first quickly 
introduce these concepts.

First we will need some explicit formulas for the product $\star_z$. Since we will 
mostly deal with certain quasi-nilpotent Banach-Lie algebras, it will be enough to 
control the product of a factorizing tensor $\xi_1 \vee \cdots \vee \xi_k$ with a 
vector $\eta$. Explicit formulas for this are known since a long time, e.g. 
\cite[Prop. 1]{gutt:1983a}, \cite[2.8.12 (c)]{dixmier:1977a} or \cite[Prop. 2.6]
{esposito.stapor.waldmann:2015a:pre}:
\begin{align}
    \label{eq:gstarformulaR}
    \xi_1 \vee \cdots \vee \xi_k \star_z \eta
    &=
    \sum\limits_{j=0}^k
    \frac{1}{k!} \binom{k}{j}
    z^j B_j^*
    \sum\limits_{\sigma \in S_k}
    \xi_{\sigma(1)} \cdots \xi_{\sigma(k - j)}
    [\xi_{\sigma(k - j + 1)},
    [ \ldots [\xi_{\sigma(k)}, \eta] \ldots ]
    ]
    \\
    \label{eq:gstarformulaL}
    \eta \star_z \xi_1 \vee \cdots \vee \xi_k
    &=
    \sum\limits_{j=0}^k
    \frac{1}{k!} \binom{k}{j}
    z^j B_j
    \sum\limits_{\sigma \in S_k}
    \xi_{\sigma(1)} \cdots \xi_{\sigma(k - j)}
    [\xi_{\sigma(k - j + 1)},
    [ \ldots [\xi_{\sigma(k)}, \eta] \ldots ]
    ]
    .    
\end{align}
The constants $B_j$ are the Bernoulli numbers, which are defined by
\begin{equation}
	\frac{x}{\E^x - 1}
	=
	\sum\limits_{j=0}^\infty
	\frac{B_j}{j!}
	x^j
\end{equation}
and $B_j^* = (-1)^j B_j$. There are also formulas for more complicated products, but 
we will not introduce them here, since \eqref{eq:gstarformulaL} and 
\eqref{eq:gstarformulaR} will suffice for our purposes.


Now, the idea is to topologize the tensor algebra $\Tensor^{\bullet}(\lie{g})$ and to 
obtain $\Sym^{\bullet}(\lie{g})$ as a closed subspace. Therefore we need a locally 
convex Lie algebra $\lie{g}$ and we call the set of all continuous seminorms 
$\algebra{P}$. For every $\halbnorm{p} \in \algebra{P}$, the projective tensor 
product $\halbnorm{p}^n = \halbnorm{p} \tensor[\pi] \cdots \tensor[\pi] \halbnorm{p}$ 
($n$-times) defines a seminorm on $\Tensor^n(\lie{g})$. We can use the grading of the 
tensor algebra to piece those seminorms together using weights. Therefore we choose a 
parameter $R \geq 0$ and define the $\Tensor_R$-topology.
\begin{definition}[$\Tensor_R$-Topology]
	Let $R \geq 0$, then we define for a continuous seminorm 
	$\halbnorm{p} \in \algebra{P}$
	\begin{equation}
		\label{def:TR-seminorms}
		\halbnorm{p}_R
		=
		\sum\limits_{n=0}^\infty
		n!^R \halbnorm{p}^n
	\end{equation}
	on $\Tensor^{\bullet}(\lie{g})$. The topology on the tensor algebra which is made 
	up by all of those $\halbnorm{p}_R$ for a fixed $R$ will be called the 
	$\Tensor_R$-topology and the topological algebra will be called 
	$\Tensor_R^{\bullet}(\lie{g})$. We also define for $R > 0$ the projective limit
	\begin{equation}
		\label{def:TR-projective}
		\Tensor_{R^-}^{\bullet}(\lie{g})
		=
		\projlim\limits_{\varepsilon \rightarrow 0}
		\Tensor_{R - \varepsilon}^{\bullet}(\lie{g}).
	\end{equation}.
\end{definition}
It is an important result that for every locally convex vector space $V$, this 
construction is possible and turns $\Tensor_R^{\bullet}(V)$ into a locally convex 
algebra. The seminorms $\halbnorm{p}_R$ are submultiplicative, i.e. $\forall_{\xi, 
\eta \in \Tensor_R^{\bullet}(V)}$
\begin{equation}
	\label{eq:TR-submultiplicative}
	\halbnorm{p}_R(\xi \tensor \eta)
	\leq
	\halbnorm{p}_R(\xi)
	\halbnorm{p}_R(\eta),
\end{equation}
if and only if $R = 0$. A very detailed discussion of this topology can be found in 
\cite{waldmann:2014a}. Yet, we mention some basic facts: for $R' > R$, the 
topology of $\Tensor_{R'}^{\bullet}(\lie{g})$ is strictly finer than the one of 
$\Tensor_R^{\bullet}(\lie{g})$ and hence the completion 
$\widehat{\Tensor}_{R'}^{\bullet}(\lie{g})$ is smaller than for $R < R'$.
Moreover, the topology has due to projective tensor product very good feature: if we 
need to show a continuity estimate on arbitrary tensors $x,y \in \Tensor^{\bullet}
(\lie{g})$, it is enough to show it on factorizing tensors $\xi_1 \tensor \cdots 
\tensor \xi_k$ and $\eta_1 \tensor \cdots \tensor \eta_\ell$ for arbitrary $k, \ell 
\in \mathbb{N}$. Once this is done, one uses the so called ``infimum argument'', 
which is described in detail in 
\cite[Proposition 3.2]{esposito.stapor.waldmann:2015a:pre} or in 
\cite[Lemma 3.2]{waldmann:2014a}. 

We say the Lie bracket in $\lie{g}$ is jointly continuous, if for every $\halbnorm{p} \in \algebra{P}$, there is a $\halbnorm{q}$ such that for all $\xi, \eta \in \lie{g}$
\begin{equation}
	\label{eq:continuity-of-liebracket}
	\halbnorm{p}([\xi, \eta])
	\leq
	\halbnorm{q}(\xi) \cdot \halbnorm{q}(\eta).
\end{equation}
It was one of the main results in \cite{esposito.stapor.waldmann:2015a:pre}, that the product $\star_z$ is continuous on $\Sym_R^{\bullet}(\lie{g})$ if and only if 
$R \geq 1$, under the condition that $\lie{g}$ is an AE-Lie algebra.
\begin{definition}[AE-algebra]
	Let $\algebra{A}$ be a not necessarily associative locally convex algebra
	and $\halbnorm{p}$ a continuous seminorm on it. Then we call
	another continuous seminorm $\halbnorm{q}$ an asymptotic estimate for 
	$\halbnorm{p}$, if for every $n \in \mathbb{N}$ and all $\xi_1, \ldots, \xi_n
	\in \algebra{A}$, the estimate
	\begin{equation}
		\label{def:AE-property}
		\halbnorm{p}\left(
			w(\xi_1, \ldots, \xi_n)
		\right)
		\leq
		\halbnorm{q}(\xi_1) \cdots \halbnorm{q}(\xi_n),
	\end{equation}
	where the notation $w(\xi_1, \ldots, \xi_n)$ means, this should hold for any
	way of setting the brackets in the multiplication. A locally convex (Lie) 
	algebra, in which every continuous seminorm has an asymptotic estimate will be 
	called AE(-Lie) algebra.
\end{definition}
AE algebras are ``almost'' locally multiplicatively convex (for short: lmc) algebras, 
meaning they have nearly submultiplicative seminorms. An associative AE algebra has 
for example an entire calculus. It is not clear, if all AE algebras are indeed lmc. 
In this sense, the AE condition \eqref{def:AE-property} is very restrictive for a 
locally convex algebra, but still very weak in general, since every normed (and hence 
every finite dimensional) algebra is AE, for example.

If the Lie algebra $\lie{g}$ is nilpotent and locally convex, it is automatically AE. In this case, continuity of $\star_z$ can already be achieved in the projective limit $\Sym_{1^-}^{\bullet}(\lie{g})$. This is important, since the completion of this projective limit is bigger than $\widehat{\Sym}_1^{\bullet}(\lie{g})$: in the case of $1^-$, the exponential series of $\xi \in \lie{g}$ is in the completion for every $\xi$, in the case $R = 1$, there are no exponentials. One aim of this paper is to find a certain type of quasi-nilpotency, which allows such a projective limit and a continuous product.


\subsection{Nilpotent locally convex Lie-Algebras}
For nilpotent, locally convex Lie algebras, we can give a new result: we know that 
for $R < 1$, $\Sym_R^{\bullet}(\lie{g})$ will not be a locally convex algebra, but it 
is still a vector space and forms a bimodule over $\Sym_{R'}^{\bullet}(\lie{g})$ for 
a sufficiently big $R' \geq 1$.
\begin{proposition}
	\label{prop:bimodules}
	Let $\lie{g}$ be a nilpotent, locally convex Lie algebra and $N \in \mathbb{N}$ 
	the smallest number, such that $\ad_{\lie{g}}^{N+1}(\lie{g}) = \{ 0 \}$. Let 
	moreover $z \in \mathbb{K}$ and $\halbnorm{p}$ be a submultiplicative seminorm.
	\begin{propositionlist}
		\item
		For all $\xi_1, \ldots, \xi_k, \eta \in \lie{g}$ and all $R \geq 0$ we have 
		the estimate
		\begin{equation}
			\halbnorm{p}_R
			\left(
				\xi_1 \tensor \cdots \tensor \xi_k \star_z \eta
			\right)
			\leq
			c (k+1)^R k^{N(1-R)}
			\halbnorm{p}_R(\xi_1 \tensor \cdots \tensor \xi_k)
			\halbnorm{p}(\eta)
		\end{equation}
		with the constant $c = \sum_{n = 0}^N \frac{|B_n^*| |z|^n}{n!}$.
		
		\item
		The vector space $\Sym_R^{\bullet}(\lie{g})$ is a bimodule over the algebra 
		$\Sym_{R + N(1-R)}^{\bullet}(\lie{g})$.
	\end{propositionlist}
\end{proposition}
\begin{proof}
	Let $\halbnorm{p}$ be a continuous seminorm and $R \geq 0$. We need to
	compute the estimate using \eqref{eq:gstarformulaR}:
	\begin{align*}
		\halbnorm{p}_R
		\big(
			\xi_1
		&	
			 \tensor \cdots \tensor \xi_k \star_z \eta
		\big)
		\\
		&=
	    \sum\limits_{j=0}^N
        \frac{1}{k!} \binom{k}{j}
        |z|^j |B_j^*|
        (k - n + 1)!^R
        \halbnorm{p}^{k - n + 1}
        \left(
            \sum\limits_{\sigma \in S_k}
            \xi_{\sigma(1)} \cdots \xi_{\sigma(k - j)}
            [\xi_{\sigma(k - j + 1)},
            [ \ldots [\xi_{\sigma(k)}, \eta] \ldots ]
            ]
        \right)
        \\
        & \leq
        (k + 1)^R
	    \sum\limits_{j=0}^N
        \frac{|z|^j |B_j^*|}{(k-j)!^{1-R} j!}
        k! 
        \halbnorm{p}(\xi_1) \cdots \halbnorm{p}(\xi_k)
        \halbnorm{p}(\eta)
        \\
        & =
        (k + 1)^R
	    \sum\limits_{j=0}^N
        \frac{|z|^j |B_j^*|}{j!}
        \frac{k!^{1 - R}}{(k-j)!^{1-R}}
        k!^R
        \halbnorm{p}(\xi_1 \tensor \cdots \tensor \xi_k)
        \halbnorm{p}(\eta)
        \\
        & \leq
        (k + 1)^R
        k^{N (1-R)}
	    \sum\limits_{j=0}^N
        \frac{|z|^j |B_j^*|}{j!}
        \halbnorm{p}_R(\xi_1 \tensor \cdots \tensor \xi_k)
        \halbnorm{p}(\eta).
	\end{align*}
	We should mention here, that the estimate for the left instead of the right-
	multiplication is exactly the same, since we only have to replace $B_j$ by 
	$B_j^*$ and we have $|B_j| = |B_j^*|$.
	
	For the second part, keep in mind that using the infimum argument, we have the 
	estimate for all tensors $x \in \bigoplus_{n=0}^k \Tensor^n(\lie{g})$. We can now 
	iterate this estimate, since
	\begin{equation*}
		\left( \xi_1 \tensor \cdots \tensor \xi_k \right)
		\star_z
		\left( \eta_1 \tensor \cdots \tensor \eta_\ell \right)
		=
		\frac{1}{\ell!}
		\sum\limits_{\tau \in S_\ell}
		\left( \xi_1 \tensor \cdots \tensor \xi_k \right)
		\star_z
		\left( \eta_{\tau(1)} \star_z \cdots \star_z \eta_{\tau(\ell)} \right).
	\end{equation*}
	This leads to
	\begin{align*}
		\halbnorm{p}_R
		\big(
			\xi_1
		&	
			 \tensor \cdots \tensor \xi_k 
			 \star_z 
			 \eta_1 \tensor \cdots \tensor \eta_\ell
		\big)
		\\
		& \leq
		\frac{(k + \ell)!^R}{k!^R}
		\left(
			\frac{(k + \ell - 1)!}{(k - 1)!}
		\right)^{N (1 - R)}
		c^\ell
		\halbnorm{p}_R( \xi_1 \tensor \cdots \tensor \xi_k )
		\halbnorm{p}( \eta_1 ) \cdots \halbnorm{p} ( \eta_\ell )
		\\
		& \leq
		2^{(k + \ell) R}
		\ell!^R
		2^{(k + \ell) N (1 - R)}
		\ell!^{N (1 - R)}
		c^\ell
		\halbnorm{p}_R( \xi_1 \tensor \cdots \tensor \xi_k )
		\halbnorm{p}( \eta_1 \tensor \cdots \tensor \eta_\ell )
		\\
		& \leq
		(2^{N + 1} \halbnorm{p})_R
		( \xi_1 \tensor \cdots \tensor \xi_k )
		(2^{N + 1} c \halbnorm{p})_{R + N (1 - R)}
		( \eta_1 \tensor \cdots \tensor \eta_\ell ).
	\end{align*}
	Using the infimum argument and knowing that this estimate works from both sides,
	we find the statement from the proposition.
\end{proof}



\subsection{Grouplike Topologies}
We are particularly interested in the question, when a Banach-Lie algebra $\lie{g}$ 
allows a locally convex topology on $\algebra{U}(\lie{g})$, which has such a big 
completion that exponential functions and hence group-like elements are part of it. This would be the case for every locally multiplicatively convex structure, for example and hence especially for a topology induced by a norm. Unfortunately, this would spoil some geometrically intuitive properties: the universal enveloping algebra may not be graded, but the symmetric tensor algebra is. It would be nice to have at least the first projection from this grading $\Sym^{\bullet}(\lie{g}) \longrightarrow \lie{g}$ to be continuous. This would guarantee that the topology of $\lie{g}$ would not be changed inside $\algebra{U}(\lie{g})$ and that $\lie{g}$ with its Banach topology is closed, even in the completion $\widehat{\algebra{U}}(\lie{g})$. 
\begin{remark}
	\label{rem:BCHDesaster}
	Assume for every $\xi, \eta \in \lie{g}$ we would have $\exp(\xi)$ and 
	$\exp(\eta)$ in $\widehat{\algebra{U}}(\lie{g})$. Then of course their product 
	would be also part of the completion. But from 
	\cite{esposito.stapor.waldmann:2015a:pre} we see that in this case
	\begin{equation*}
		\exp(\xi) \cdot \exp(\eta)
		=
		\exp\left(
			\bch{\xi}{\eta}
		\right),
	\end{equation*}
	so for all $\xi, \eta \in \lie{g}$, we could give a sense to the element 
	$\exp\left( \bch{\xi}{\eta} \right)$. If the projection $\pi_1$ onto $\lie{g}$ 
	would be continuous, we could even reconstruct an element $\bch{\xi}{\eta} \in 
	\lie{g}$, although we know that it does not need to exist at all. This argument 
	is made precise in Proposition \ref{prop:bchdesaster}.
	It is of course a neat question to ask, why we want this projection to be 
	continuous. To keep the topology of $\lie{g}$ inside $\widehat{\algebra{U}}
	(\lie{g})$, it would be enough to ask for an injective and continuous embedding 
	\begin{equation*}
		\iota
		\colon
		\lie{g}
		\longrightarrow
		\widehat{\algebra{U}}(\lie{g}),
	\end{equation*}
	which is in addition an continuous isomorphism in the range inside the target.
	This is of course true, but there is a special interest of having the projection 
	$\pi_1$ 	continuous which is maybe more visible if we go to the finite-dimensional 
	case. If $\lie{g}$ is finite-dimensional, there are always (many) continuous
	projections by the Hahn-Banach theorem, no matter if the Lie algebra is nilpotent 
	or not. Yet, our argument tells us that $\pi_1$, the only projection which is 
	canonically there, will \emph{not} be continuous, but infinitely many others
	which depend on choices. This is a situation, which we would like to avoid and 
	therefore, we give the next definition.
\end{remark}
\begin{definition}
	\label{prop:bchdesaster}
	Let $\lie{g}$ be a Banach-Lie algebra and $\tau$ a locally convex topology on 
	$\algebra{U}(\lie{g})$. Then we say $\tau$ is a group-like topology, if the 
	three following things hold:
	\begin{definitionlist}
		\item
		The multiplication in $(\algebra{U}(\lie{g}), \tau)$ is continuous.
		
		\item
		For every $\xi \in \lie{g}$, the exponential series $\sum_{n=0}^\infty 
		\frac{\xi^n}{n!}$ converges in the completion of $(\algebra{U}(\lie{g}), 
		\tau)$.
		
		\item
		The projection $\pi = \sigma \circ \pi_1 \circ \sigma^{-1}$ is continuous.
	\end{definitionlist}
\end{definition}
We now want to see that if a locally convex topology is $\E$-like, then effects 
like in Remark \ref{Rem:BCHDesaster} can not occur. 
\begin{proposition}
	Let $\lie{g}$ be a Banach-Lie algebra and $\algebra{U}(\lie{g})$ has an 
	grouplike topology. Then the Baker-Campbell-Hausdorff series converges globally 
	on $\lie{g}$.
\end{proposition}
\begin{proof}
	We use the $\E$-like topology on $\algebra{U}(\lie{g})$.
	Since $\widehat{\algebra{U}}(\lie{g})$ is a Fr\'echet space and $\lie{g}$ is a 
	closed subspace, the can look at the quotient 
	\begin{equation*}
		\algebra{U}'(\lie{g})
		=
		\frac{\widehat{\algebra{U}(\lie{g})}}{\lie{g}},
	\end{equation*}
	which will be a Fr\'echet space again, since $\lie{g}$ is closed and the 
	topology on $\widehat{\algebra{U}(\lie{g})}$ is second countable. We can see 
	$\algebra{U}'(\lie{g})$ as a subspace of $\widehat{\algebra{U}(\lie{g})}$ an 
	get two continuous projections
	\begin{equation*}
		\pi' \colon
		\widehat{\algebra{U}(\lie{g})}
		\longrightarrow
		\algebra{U}'(\lie{g})
		\quad \text{ and }
		\pi \colon
		\widehat{\algebra{U}(\lie{g})}
		\longrightarrow
		\lie{g}
	\end{equation*}
	fulfilling $\pi + \pi' = \id$. So in an $\E$-like topology, we have for $\xi, 
	\eta \in \lie{g}$ and $t,s \in \mathbb{K}$
	\begin{align*}
		\pi \left( \exp(t \xi) \cdot \exp(s \eta) \right)
		& =
		\pi
		\left(
			\lim_{N \rightarrow \infty}
			\left(
				\sum\limits_{n=0}^N
				\frac{t^n \xi^n}{n!}
			\right)
			\cdot
			\lim_{M \rightarrow \infty}
			\left(
				\sum\limits_{m=0}^M
				\frac{s^m \eta^m}{m!}
			\right)
		\right)
		\\
		& \ot{(a)}{=}
		\pi
		\left(
			\lim_{N \rightarrow \infty}
			\lim_{M \rightarrow \infty}
			\left(
				\sum\limits_{n=0}^N
				\frac{t^n \xi^n}{n!}
			\right)
			\cdot
			\left(
				\sum\limits_{m=0}^M
				\frac{s^m \eta^m}{m!}
			\right)
		\right)
		\\
		& \ot{(b)}{=}
		\lim_{N \rightarrow \infty}
		\lim_{M \rightarrow \infty}
		\pi
		\left(	
			\left(
				\sum\limits_{n=0}^N
				\frac{t^n \xi^n}{n!}
			\right)
			\cdot
			\left(
				\sum\limits_{m=0}^M
				\frac{s^m \eta^m}{m!}
			\right)
		\right)
		\\
		& \ot{(c)}{=}
		\lim_{N \rightarrow \infty}
		\lim_{M \rightarrow \infty}
		\sum\limits_{n=0}^{N}
		\sum\limits_{m=0}^{M}
		\bchparts{n}{m}{t \xi}{s \eta}
		\\
		& \ot{(d)}{=}
		\lim_{L \rightarrow \infty}
		\sum\limits_{\ell=0}^{L}
		\sum\limits_{a + b = \ell}
		\bchparts{a}{b}{t \xi}{s \eta}
		\\
		& =
		\lim_{L \rightarrow \infty}
		\sum\limits_{\ell=0}^{L}
		\bchpart{\ell}{t \xi}{s \eta},
	\end{align*}
	where we used the continuity of multiplication in $(a)$ and the one of $\pi$ in 
	$(b)$. In $(c)$ we just evaluated $\pi$ and used the fact that the 
	Baker-Campbell-Hausdorff series is a power series in $(d)$ and hence converges 
	absolutely. So it must converge globally on $\lie{g}$, when the latter is seen 
	as a subspace of $\widehat{\algebra{U}}(\lie{g})$. But since the topology is 
	$\E$-like and therefore equivalent to the one of $\lie{g}$ as Banach-Lie 
	algebra, $\lie{g}$ must be globally BCH.
\end{proof}
From \cite{woj} we see immediately, that being quasi-nilpotent is a necessary 
condition for being $\E$-like for a Banach-Lie algebra. For a finite-dimensional 
Lie algebra, this means that it must be nilpotent, since there quasi-nilpotency 
implies nilpotency. Furthermore, in \cite{stapor:2015a}, it is shown that being 
topologically uniformly nilpotent Banach-Lie algebras are $\E$-like. There and in 
\cite{esposito.stapor.waldmann:2015a:pre}, the so called $\Sym_{1^-}$ was shown to 
be $\E$-like for nilpotent, locally convex Lie algebras.




\section{Quasi-Nilpotency in Banach-Lie Algebras}
\label{sec:QuasiNilpotency}

\subsection{Classes of Quasi-Nilpotent Banach-Lie Algebras}
First, we want to classify, how we can ``measure'' the nilpotency of a Banach-Lie 
algebra. In an associative Banach algebra $\algebra{A}$, the usual way is looking 
at the spectral radii of the elements $a \in \algebra{A}$ or at the joint spectral 
radii of bounded subsets of $\algebra{A}$. Unfortunately, for a Banach-Lie algebra, 
a good notion of the spectral radius of elements is not so obvious. In 
\cite{galindo.palacios:}, Galindo and Palacios constructed something that they 
called the ``Lie-spectral radius''. We want to go a different way. Let $\lie{g}$ be 
a Banach-Lie algebra, then we know that $\lie{g}$ itself is not an associative 
Banach algebra, but the bounded linear operators on it $\Bounded(\lie{g})$ surely 
form such an algebra using the spectral norm. Since the map $\ad \colon \lie{g} 
\longrightarrow \Bounded(\lie{g})$ is a continuous homomorphism of Lie algebras, we 
know that its image forms a closed (and hence complete) subspace of 
$\Bounded(\lie{g})$ by the closed graph theorem. The subalgebra generated by 
$\Im(\ad)$ is now a subalgebra of $\Bounded(\lie{g})$ and hence a normed (but not 
necessarily complete) associative algebra. However, we can now use the notions from 
the associative theory by applying them to this subalgebra of $\Bounded(\lie{g})$.
\begin{definition}
	\label{Def:Nilpotencies1}
	Let $\lie{g}$ be a Banach-Lie algebra in which the Lie bracket fulfils the 
	estimate
	\begin{equation*}
		\norm{ [\xi, \eta] }
		\leq
		\norm{\xi}
		\norm{\eta}.
	\end{equation*}
	Denote by $\mathbb{B}_1(0)$ all elements $\xi \in \lie{g}$ with 
	$\norm{\xi} = 1$. We say that
	\begin{definitionlist}
		\item
		$\lie{g}$ is topologically nil (or radical, or quasi-nilpotent, or 
		(quasi-nilpotent) of type $(i)$), if every $\xi \in \lie{g}$ is 
		quasi-nilpotent, i.e.
		\begin{equation*}
			\lim_{n \longrightarrow \infty}
			\norm{\ad_{\xi}^n}^{\frac{1}{n}}
			=
			0.
		\end{equation*}
		
		\item
		$\lie{g}$ is uniformly topologically nil or (quasi-nilpotent) of type 
		$(ii)$, if
		\begin{equation*}
			\lim_{n \longrightarrow \infty}
			\mathcal{N}_1(n)
			=
			0
		\end{equation*}
		for
		\begin{equation}
			\mathcal{N}_1(n)
			=
			\sup \left\{ 
			\left.
				\norm{ \ad_{\xi}^n}^{\frac{1}{n}} 
			\right|
				\xi \in \mathbb{B}_1(0)
			\right\}.
		\end{equation}
		
		\item
		$\lie{g}$ is topologically nilpotent or (quasi-nilpotent) of type 
		$(iii)$, if for every sequence
		$(\xi_n)_{n \in \mathbb{N}} \subset \mathbb{B}_1(0)$ we have
		\begin{equation*}
			\lim_{n \longrightarrow \infty}
			\norm{ 
				\ad_{\xi_1} \circ \ldots \circ \ad_{\xi_n}
			}^{\frac{1}{n}}
			=
			0.
		\end{equation*}
		
		\item
		$\lie{g}$ is uniformly topologically nilpotent or (quasi-nilpotent) of 
		type $(iv)$, if
		\begin{equation*}
			\lim_{n \longrightarrow \infty}
			\mathcal{N}(n)
			=
			0
		\end{equation*}
		for
		\begin{equation}
			\mathcal{N}(n)
			=
			\sup \left\{ 
			\left.
				\norm{ 
					\ad_{\xi_1} \circ \ldots \circ \ad_{\xi_n}
				}^{\frac{1}{n}} 
			\right|
				\xi_1, \ldots, \xi_n \in \mathbb{B}_1(0)
			\right\}.
		\end{equation}
	\end{definitionlist}
\end{definition}
This definition is directly taken from \cite{muller}, just using the little adaptations we explained before. We will need need a notion, which is on the first sight a stronger version of $(ii.)$, but which is in fact equivalent. For the moment, we define it as follows.
\begin{itemize}
	\item[$ii'.)$]
	Let $\lie{g}$ be a Banach-Lie algebra like in Definition 
	\ref{Def:Nilpotencies1}, then we say that $\lie{g}$ is of type $(ii')$, if
	\begin{equation*}
		\lim_{n \longrightarrow \infty}
		\mathcal{S}(n)
		=
		0
	\end{equation*}
	for
	\begin{equation}
		\mathcal{S}(n)
		=
		\sup \left\{ 
		\left.
			\left\Vert
				\frac{1}{n!}
				\sum\limits_{\sigma \in S_n}
				\ad_{\xi_{\sigma(1)}} \circ \ldots \circ \ad_{\xi_{\sigma(n)}}
			\right\Vert^{\frac{1}{n}} 
		\right|
			\xi_1, \ldots, \xi_n \in \mathbb{B}_1(0)
		\right\}.
	\end{equation}
\end{itemize}
It is immediate to see the following implications:
\begin{equation*}
	(iv) \Longrightarrow 
	(ii') \Longrightarrow 
	(ii) \Longrightarrow 
	(i)
	\quad \text{ and } \quad
	(iv) \Longrightarrow
	(iii) \Longrightarrow 
	(i).
\end{equation*}
We will now show, that indeed $(ii) \Longleftrightarrow (ii')$, but we need a combinatorial lemma first.
\begin{lemma}
	\label{lemma:symmetric-to-power}
	Let $R$ be a ring of characteristic $0$, $n\in \mathbb{N}$ and $a_1, 
	\ldots, a_n \in R$, then we have the identity
	\begin{equation*}
		\sum\limits_{\sigma \in S_n}
		a_{\sigma(1)} \cdots a_{\sigma(n)}
		=
		\sum\limits_{k = 1}^n
		(-1)^{n-k}
		\sum\limits_{1 \leq i_1 < \cdots < i_k \leq n}
		(a_{i_1} + \cdots + a_{i_k})^n.
	\end{equation*}
\end{lemma}
\begin{proof}
	The easiest way is to prove this by a combinatorial argument.
\end{proof}
\begin{proposition}
	\label{prop:type-2-is-2prime}
	Banach-Lie algebras, which are quasi-nilpotent of type $(ii)$, are of type 
	$(ii')$.
\end{proposition}
\begin{proof}
	Let $\lie{g}$ be a type $(ii)$-Banach-Lie algebra and fix the series
	$\chi_n = \mathcal{N}_1(n)$. Hence for every $\xi \in \mathbb{B}_1(0)$, we have
	\begin{equation*}
		\norm{(\ad_\xi)^n}^{\frac{1}{n}}
		\leq
		\chi_n.
	\end{equation*}
	Note that 
	\begin{equation*}
		\tag{$*$}
		n \geq \sqrt[n]{n!} \geq \frac{n}{\E},
	\end{equation*}
	and hence for all $a_1, \ldots, a_k \in \mathbb{B}_1(0)$, we have
	\begin{equation*}
		\frac{1}{\E \sqrt[n]{n!}} (a_1 + \cdots + a_k)
		\in \mathbb{B}_1(0).
	\end{equation*}
	This gives
	\begin{align*}
		\left\Vert
			\frac{1}{n!}
			\sum\limits_{\sigma \in S_n}
			\ad_{\sigma(1)}
			\circ \cdots \circ
			\ad_{\sigma(n)}
		\right\Vert
		&=
		\frac{1}{n!}
		\left\Vert
			\sum\limits_{k = 1}^n
			(-1)^{n-k}
			\sum\limits_{1 \leq i_1 < \cdots < i_k \leq n}
			(\ad_{\xi_{i_1}} + \cdots + \ad_{\xi_{i_n}})^n
		\right\Vert
		\\
		& \leq
		\frac{1}{n!}
		\sum\limits_{k = 1}^n
		\sum\limits_{1 \leq i_1 < \cdots < i_k \leq n}
		\left\Vert
			(\ad_{\xi_{i_1} + \cdots + \xi_{i_n}})^n
		\right\Vert
		\\
		& \leq
		\frac{1}{n!}
		\sum\limits_{k = 1}^n
		\binom{n}{k}
		k^n \chi_n^n
		\\
		& \leq
		\frac{n^n}{n!}
		\sum\limits_{k = 1}^n
		\binom{n}{k}
		\chi_n^n
		\\
		& \leq
		(2 \E \chi_n)^n.
	\end{align*}
	Hence we find
	\begin{equation*}
		\mathcal{S}(n)
		\leq
		\sqrt[n]{(2 \E \chi_n)^n}
		=
		2 \E \chi_n,
	\end{equation*}
	which converges to zero if and only if $\chi_n$ does.
\end{proof}



\subsection{Examples}



\section{Topologically nilpotent Banach-Lie Algebras}
Now we want to show that that Banach-Lie algebras of type $(iii)$ are already of type 
$(iv)$. the proof is alomost the same as in the associative case and mostly taken 
from \cite{dixon.mueller:}. For reasons of completeness, we want give it here anyway, 
since a slight change has to be made.
\begin{proposition}
	Topologically nilpotent Banach-Lie algebras are uniformly topologically 
	nilpotent.
\end{proposition}
\begin{proof}
	Let $\lie{g}$ be a Banach-Lie algebra, then the limit $\mathcal{N} = 
	\lim_{n \rightarrow \infty} \mathcal{N}(n)$ exists. We will show 
	that there exists a sequence $(\xi_1, \xi_2, \ldots) \subset \mathbb{B}_1(0)$, 
	such that
	\begin{equation*}
		\limsup_{n \rightarrow \infty}
		\norm{\ad_{\xi_1} \circ \cdots \circ \ad_{\xi_n} }^{1/n}
		=
		\mathcal{N},
	\end{equation*}
	which implies the proposition, since the left hand side converges to zero by 
	the definition of topological nilpotency. It is clear that
	\begin{equation*}
		\limsup_{n \rightarrow \infty}
		\norm{\ad_{\xi_1} \circ \cdots \circ \ad_{\xi_n} }^{1/n}
		\leq
		\mathcal{N},
	\end{equation*}
	for every such sequence, so we find a sequence such that
	\begin{equation*}
		\limsup_{n \rightarrow \infty}
		\norm{\ad_{\xi_1} \circ \cdots \circ \ad_{\xi_n} }^{1/n}
		\geq
		\mathcal{N}
	\end{equation*}
	holds. Since this is clear for $\mathcal{N} = 0$, we suppose $\mathcal{N} > 0$.
	We define the space of sequences $X = \mathbb{B}_1(0)^{\mathbb{N}}$ with the
	metric
	\begin{equation*}
		\halbnorm{d}((\xi), (\eta))
		=
		\sum\limits_{i = 1}^\infty
		2^{-i} \frac{\norm{\xi_i - \eta_i}}{1 + \norm{\xi_i - \eta_i}}.
	\end{equation*}
	With this metric, $X$ is complete. Define furthermore for $\delta > 0$
	\begin{align*}
		X_{k, \delta}
		&=
		\left\{
			(\xi) \in X
		\ \left| \
			\norm{ \ad_{\xi_1} \circ \cdots \circ \ad_{\xi_n} }^{1/n}
			>
			(1 - \delta) \mathcal{N}
			, \
			\text{ for some }
			n \geq k	
		\right.
		\right\}
		\\
		Y_{k, \delta}
		&=
		\left\{
			(\xi) \in X
		\ \left| \
			\norm{ \ad_{\xi_1} \circ \cdots \circ \ad_{\xi_k} }^{1/k}
			>
			(1 - \delta) \mathcal{N}
		\right.
		\right\}.
	\end{align*}
	Then it is easy to see, that the $Y_{k, \delta}$ are open in $X$. Hence also 
	$X_{k, \delta}$ is open in $X$, since
	\begin{equation*}
		X_{k, \delta}
		=
		\bigcup\limits_{n \geq k}
		Y_{n, \delta}.
	\end{equation*}
	Our aim is to show, that each of these $X_{k, \delta}$ is dense in $X$, because 
	then, we could conclude that
	\begin{equation*}
		\bigcap\limits_{k = 1}^\infty
		X_{k, 1/k}
		\neq
		\emptyset
	\end{equation*}
	by the Baire category theorem for metric spaces. Clearly the set
	$\bigcup\limits_{r < 1} \mathbb{B}_r(0)$
	is dense in $X$ and we want to show that for any $k$ and any $\delta$, 
	$X_{k, \delta}$ is dense in $\bigcup\limits_{r < 1} \mathbb{B}_r(0)$. 
	So we need to show that for every $k \in \mathbb{N}$, $r < 1$, 
	$(\xi) \in \mathbb{B}_r(0)^{\mathbb{N}}$, $0 < \delta < 1$ and $0 < \varepsilon 
	< 1 - r$, there is a $(\eta) \in X_{k, \delta}$, such that
	\begin{equation*}
		\halbnorm{d}((\xi), (\eta))
		<
		\varepsilon.
	\end{equation*}
	So let's fix $k \in \mathbb{N}$, $\delta, r \in (0,1)$, $0 < \varepsilon < 1-r$
	and $(\xi) \in \mathbb{B}_r(0)^\mathbb{N}$. Let $m \in \mathbb{N}$ be big enough
	to have $2 \cdot 2^{-m} < \varepsilon$. Then, if for all $i \in \{1, \ldots, 
	m\}$ the estimate $\norm{\xi_i - \eta_i} < \frac{\varepsilon}{2}$ holds for two 
	sequences $(\xi), (\eta) \in X$, then $\halbnorm{d}((\xi), (\eta)) < 
	\varepsilon$. We fix such an $m \in \mathbb{N}$ and choose $\ell \geq 
	\max{k, m}$ such that
	\begin{equation*}
		\left(
			\frac{\varepsilon}{2^{m + 1}}
		\right)^{1 / \ell}
		\geq
		\sqrt{1 - \delta}.
	\end{equation*}
	This is certainly possible for $\ell$ big enough. We can also choose some
	$\vartheta_1, \ldots, \vartheta_\ell \in \mathbb{B}_1(0)$ such that
	\begin{equation*}
		\norm{
			\ad_{\vartheta_1} \circ \cdots \circ \ad_{\vartheta_\ell}
		}^{1/ \ell}
		\geq
		\mathcal{N}(l) \sqrt{1 - \delta}
		\geq
		\mathcal{N} \sqrt{1 - \delta}.
	\end{equation*}
	Just by the definition of $\mathcal{N}(\ell)$, such elements must exist.
	Now define $\varepsilon_t = \varepsilon \E^{2 \pi \I t/m} $ for $t = 0, 1, 
	\ldots, m$ and set $\varepsilon_0 = 0$. Set for $t$ in the same limits
	\begin{equation*}
		\theta_i^t
		=
		\xi_i + \frac{\varepsilon_t}{2} \vartheta_i
	\end{equation*}
	and
	\begin{equation*}
		\Theta_t
		=
		\ad_{\theta_1^t} \circ \cdots \circ \ad_{\theta_m^t}
		\circ
		\ad_{\vartheta_{m+1}} \circ \cdots \circ \ad_{\vartheta_\ell}.
	\end{equation*}
	Then we have
	\begin{equation*}
		\Theta_0
		=
		\ad_{\xi_1} \circ \cdots \circ \ad_{\xi_m}
		\circ
		\ad_{\vartheta_{m+1}} \circ \cdots \circ \ad_{\vartheta_\ell}.
	\end{equation*}
	and for algebraic reasons and because of the linearity of $\ad$
	\begin{equation*}
		\sum\limits_{j=1}^m
		(\Theta_j - \Theta_0)
		=
		\frac{m \varepsilon^m}{2^m}
		\ad_{\vartheta_1} \circ \cdots \circ \ad_{\vartheta_\ell}.
	\end{equation*}
	Hence we find
	\begin{equation*}
		\Big\Vert
			\sum\limits_{j=1}^m
			(\Theta_j - \Theta_0)
		\Big\Vert
		=
		\frac{m \varepsilon^m}{2^m}
		\norm{ \ad_{\vartheta_1} \circ \cdots \circ \ad_{\vartheta_\ell} }
		\geq
		\frac{m \varepsilon^m}{2^m}
		\mathcal{N}^\ell
		\sqrt{1-\delta}^\ell.
	\end{equation*}
	Thus, there is an $s \in \{0, 1, \ldots, m\}$ with
	\begin{equation*}
		\norm{\Theta_s}
		\geq
		\frac{\varepsilon^m}{2^{m + 1}}
		\mathcal{N}^\ell
		\sqrt{1- \delta}^\ell
		\geq
		\mathcal{N}^\ell (1- \delta)^\ell.
	\end{equation*}
	We can now construct the wanted sequence $(\eta)$ by setting
	\begin{equation*}
		\eta_i
		=
		\begin{cases}
			\theta_i^s
			&
			1 \leq i \leq m
			\\
			\vartheta_i
			&
			m+1 \leq i \leq \ell
			\\
			\xi_i
			&
			i \geq \ell + 1
		\end{cases}
	\end{equation*}
	and get $(\eta) \in X_{k, \delta}$, since
	\begin{equation*}
		\norm{ \ad_{\eta_1} \circ \cdots \circ \ad_{\eta_\ell}}^{1/\ell}
		=
		\norm{ \Theta_s }^{1/\ell}
		\geq
		(1 - \delta) \mathcal{N}
	\end{equation*}
	and $\ell \geq k$. The choice of $m$ ensures moreover $\halbnorm{d}((\xi), 
	(\eta)) < \varepsilon$ and hence the proof is finished.
\end{proof}
This settles the possible types of nilpotency of Banach-Lie algebras to just three with the following implications:
\begin{equation}
	\label{nilpotencytypes}
	(iii) \Longrightarrow (ii) \Longrightarrow (i).
\end{equation}


\section{Uniformly quasi-nilpotent Banach-Lie algebras are $\E$-like}

We have everything at the hand to state or main Theorem now:
\begin{theorem}
	\label{Thm:Main}
	A Banach-Lie algebra is $\E$-like if and only if it is uniformly 
	quasi-nilpotent.
\end{theorem}
Since it is rather long and technical, we will split up the proof into two parts.


\subsection{$\E$-like Banach Lie Algebras are of Type $II$}
It is rather easy to show, that being of type $(ii)$ is really necessary to be 
$\E$-like.
\begin{proposition}
	Let $\lie{g}$ be a Banach-Lie algebra, which is not of type $(ii)$. Then it is 
	not $\E$-like.
\end{proposition}
\begin{proof}
	The idea is to construct two sequences which converge to zero, but whose 
	product does not converge: Being not of type $(ii)$ means, that there is an
	$\varepsilon > 0$, such that $\lim_{n \rightarrow \infty} \mathcal{N}_1(n) > 
	\varepsilon$. This means that for every $n \in \mathbb{N}$, there is a sequence 
	$\left(\xi_{n, k} \right)_{k \in \mathbb{N}} \subset \mathbb{B}_1(0)$, 
	such that
	\begin{equation*}
		\lim_{k \longrightarrow \infty}
		\left\Vert
			(\ad_{\xi_{n, k}})^n
		\right\Vert^{\frac{1}{n}}
		=
		\mathcal{N}_1(n).
	\end{equation*}
	So for every $n, k \in \mathbb{N}$ we have
	\begin{equation*}
		\left\Vert
			(\ad_{\xi_{n, k}})^n
		\right\Vert^{\frac{1}{n}}
		> \varepsilon.
	\end{equation*}
	This means that we find for every element $\xi_{n,n}$ a sequence 
	$(\alpha_{n, \ell})_{\ell \in \mathbb{N}}$ with $\norm{\alpha_{n, \ell}} = 1$ 
	for all $n, \ell \in \mathbb{N}$ and
	\begin{equation*}
		\lim_{\ell \longrightarrow \infty}
		(\ad_{\xi_{n,n}})^n(\alpha_{n,\ell})
		> 
		\varepsilon^n.
	\end{equation*}
	We define a new sequence $(\eta_n)_{n \in \mathbb{N}}$ with 
	$\eta_n = \alpha_{n, \ell}$ where $\ell$ is chosen big enough to 
	fulfil $(\ad_{\xi_{n,n}})^n(\alpha_{n,\ell}) > \varepsilon^n$.
	Now let's assume we have an $\E$-like topology on $\algebra{U}(\lie{g})$, 
	defined by a countable set of seminorms and we denote by $\algebra{P}$ the set 
	of all continuous seminorms. Then we have for every $\xi \in \lie{g}$ and every 
	$p \in \algebra{P}$
	\begin{equation*}
		p \left( \exp(\xi) \right)
		=
		p \left(
			\sum\limits_{n=0}^\infty
			\frac{\xi^n}{n!}
		\right)
		\leq
		\sum\limits_{n=0}^\infty
		\frac{p\left( \xi^n \right)}{n!}
		<
		\infty,
	\end{equation*}
	since the power series converges absolutely. Since the sequence $(\xi_{n,n})_{n 
	\in \mathbb{N}}$ is bounded in $\lie{g}$, it is also bounded in $\algebra{U}
	(\lie{g})$. Hence for every $t > 0$, the sequence $(\exp(t \xi_{n,n}))_{n \in 
	\mathbb{N}}$ is bounded by the continuity of the exponential function and the 
	sequence
	\begin{equation*}
		\left(
			\frac{t^n}{n!}
			p\left(
				\xi_{n,n}^n
			\right)
		\right)_{n \in \mathbb{N}}
	\end{equation*}
	converges to zero (in $\lie{g}$ and in $\algebra{U}(\lie{g})$). The series 
	$(\frac{1}{n} \eta_n)_{n \in \mathbb{n}}$ also converges to zero, again in 
	$\lie{g}$ and $\algebra{U}(\lie{g})$.
	Now we want to show that their product does \emph{not} converge to zero
	to get a contradiction to the continuity of the multiplication. Therefore
	we need again the projection map $\pi_1 \colon \algebra{U}(\lie{g}) 
	\longrightarrow \lie{g}$ which is linear and continuous and should hence not 
	spoil the convergence. For every fixed $t > 0$, we have
	\begin{align*}
		p \left(
			\pi_1\left(
				\frac{t^n \xi_{n,n}^n}{n!}
				\cdot
				\eta_n
			\right)
		\right)
		&=
		p\left(
			\pi_1 \left(
				\frac{1}{n!}
				\sum\limits_{k=0}^n
				\binom{n}{k}
				B_k^* t^n
				\xi_{n,n}^{\vee(k-n)}
				\vee
				(\ad_{\xi_{n,n}})^k(\eta_n)
			\right)
		\right)
		\\
		& =
		p\left(
			\frac{B_n^* t^n}{n!}
			(\ad_{\xi_{n,n}})^n(\eta_n)
		\right)
		\\
		& >
		\frac{|B_n|^* t^n \varepsilon^n}{n!}.
	\end{align*}
	But we know that this series does not converge to zero, if we choose for 
	example $t = \frac{3 \pi}{\varepsilon}$, which is allowed. Hence we have a 
	contradiction.
\end{proof}


\subsection{Type $II'$ quasi-nilpotent Banach-Lie algebras are $\E$-like}

For technical reasons, the reverse direction is split into two parts.

\subsubsection{Part I}
\begin{proposition}
	Let $\lie{g}$ be a uniformly quasi-nilpotent Banach-Lie algebra and assume that 
	there is a $p \geq 1$, such that $\lim_{n \rightarrow \infty} 
	\frac{\mathcal{N}_1(n)}{\sqrt[p]{n!}} = 0$. Then there is an $\E$-like topology 
	on $\algebra{U}(\lie{g})$.
\end{proposition}
\begin{proof}
	The proof is constructive. We look at the situation in the symmetric tensor 
	algebra and see that the product is continuous in $\Sym_{1^-}^{\bullet}
	(\lie{g})$ by the following argumentation. Let's denote the norm on $\lie{g}$ 
	for simplicity by $p$. Since $\lim_{n \rightarrow \infty} 
	\frac{\mathcal{N}_1(n)}{\sqrt[p]{n}} = 0$, we know there is a $p \geq 1$ and a 
	$c > 0$, such that $\mathcal{S}(n) \leq \frac{c}{\sqrt[p]{n}}$. Take $\xi_1, 
	\ldots, \xi_k, \eta \in \lie{g}$, then we find for some $\varepsilon > 0$ with 
	$R > 1 - \frac{1}{p} + \varepsilon$
	\begin{align*}
		p_R \big(
		&
			\xi_1 \tensor \cdots \tensor \xi_k
			\star
			\eta
		\big)
		\\
		&=
		\sum\limits_{n = 0}^k
		\frac{|B_n^*|}{k!} \binom{k}{n}
		(k + 1 - n)!^R
		p^{k+1-n}
		\left(
			\sum	\limits_{\sigma \in S_k}
			\xi_{\sigma(1)} \cdots \xi_{\sigma(k-n)}
			\cdot
			\left( 
				\ad_{\xi_{\sigma(k-n+1)}} 
				\circ \cdots \circ
				\ad_{\xi_{\sigma(k)}}
			\right)
			(\eta)
		\right)
		\\
		& \leq
		(k + 1)^R
		\sum\limits_{n = 0}^k
		\frac{|B_n^*|}{n!}
		\frac{1}{(k-n)!^{1 - R}}
		p^{k+1-n}
		\Bigg(
			\sum\limits_{
				I = \{i_1, \ldots, i_n\} \subset \{1, \ldots, k\}
			}
			(n-k)!
			\xi_1 \cdots \widehat{\xi_I} \cdots \xi_k
		\\
		& \quad \cdot
			\sum	\limits_{\sigma \in S_n}
			\left( 
				\ad_{\xi_{\sigma(1)}} 
				\circ \cdots \circ
				\ad_{\xi_{\sigma(n)}}
			\right)
			(\eta)
		\Bigg)
		\\
		& \leq
		(k + 1)^R
		\sum\limits_{n = 0}^k
		\frac{|B_n^*|}{n!}
		\frac{(k-n)!}{(k-n)!^{1 - R}}
		\sum\limits_{
			I = \{i_1, \ldots, i_n\} \subset \{1, \ldots, k\}
		}
		p(\xi_1) \cdots \widehat{p(\xi_I)} \cdots p(\xi_k)
		\\
		& \quad \cdot
		p \left(
			\sum	\limits_{\sigma \in S_n}
			\left( 
				\ad_{\xi_{\sigma(1)}} 
				\circ \cdots \circ
				\ad_{\xi_{\sigma(n)}}
			\right)
			(\eta)
		\right)
		\\
		& \leq
		(k + 1)^R
		\sum\limits_{n = 0}^k
		\frac{|B_n^*|}{n!}
		\frac{(k-n)!}{(k-n)!^{1 - R}}
		\binom{k}{n}
		\frac{c^n n!}{n!^{\frac{1}{p}}}
		p(\xi_1) \cdots p(\xi_k) p(\eta)
		\\
		& \leq
		(k + 1)^R
		\sum\limits_{n = 0}^k
		\frac{c^n |B_n^*|}{n!^{1 + \varepsilon}}
		\binom{k}{n}^{1-R}
		k!^R
		p(\xi_1) \cdots p(\xi_k) p(\eta)
		\\
		& \leq
		(k + 1)^R
		\underbrace{
			2^{k (1-R)}
		}_{
			=\kappa_1 k!^{\frac{1-R}{2}}
		}
		\underbrace{
			\sum\limits_{n = 0}^k
			\frac{c^n |B_n^*|}{n!^{1 + \varepsilon}}
		}_{
			= \kappa_2
		}
		k!^R
		p^k(\xi_1 \tensor \cdots \tensor \xi_k) p(\eta)
		\\
		& \leq
		(k-1)^R
		\kappa
		p_{R + \frac{1-R}{2}}
		(\xi_1 \tensor \cdots \tensor \xi_k)
		p(\eta).
	\end{align*}
	In the last step, we took $\kappa = \kappa_1 \kappa_2$. We can extend this 
	estimate to any arbitrary tensor of degree at most $k$ on the left hand side.
	Now we do the next step and take $\xi_1, \ldots, \xi_k, \eta_1, \ldots, 
	\eta_\ell \in \lie{g}$. We can iterate this estimate to get
	\begin{align*}
		p_R \big(
			\xi_1 \tensor \cdots \tensor \xi_k
		&
			\star
			\eta_1 \tensor \cdots \tensor \eta_\ell
		\big)
		\\
		& \leq
		(k + \ell)^R (k + \ell - 1)^{1 - \frac{1 - R}{2}}
		\cdots (k+1)^{{1 - \frac{1 - R}{2^{\ell - 1}}}}
		\kappa^\ell
		p_{1 - \frac{1-R}{2^\ell}}
		\left( \xi_1 \tensor \cdots \tensor \xi_k \right)
		p(\eta_1) \cdots p(\eta_\ell)
		\\
		& \leq
		\left( \frac{(k+\ell)!}{\ell!} \right)^{\frac{1 + R}{2}}
		\kappa^\ell
		p_1 \left( \xi_1 \tensor \cdots \tensor \xi_k \right)
		p^\ell \left( \eta_1 \tensor \cdots \tensor \eta_\ell \right)
		\\
		& \leq
		2^{k + \ell}
		\ell!^{\frac{1 + R}{2}}
		\kappa^\ell
		p_1 \left( \xi_1 \tensor \cdots \tensor \xi_k \right)
		p^\ell \left( \eta_1 \tensor \cdots \tensor \eta_\ell \right)
		\\
		&=
		(2p)_1 \left( \xi_1 \tensor \cdots \tensor \xi_k \right)
		(2 \kappa p)_{\frac{1 + R}{2}} 
		\left( \eta_1 \tensor \cdots \tensor \eta_\ell \right).
	\end{align*}
	Finally we can assume without loss of generality that $\ell \geq k$, since
	we could do all estimates also for the right hand side in the same way.
	By setting $R' = \frac{3}{4}(1-R)$, we find
	\begin{equation*}
		p_R \big(
			\xi_1 \tensor \cdots \tensor \xi_k
			\star
			\eta_1 \tensor \cdots \tensor \eta_\ell
		\big)
		\leq
		(2p)_{R'}
		\left( \xi_1 \tensor \cdots \tensor \xi_k \right)
		(2 \kappa p)_{R'}
		\left( \eta_1 \tensor \cdots \tensor \eta_\ell \right)
	\end{equation*}
	and the proposition is proven.
\end{proof}


\subsubsection{Part II}
Now we need to cover the last possible case: A uniformly quasi-nilpotent Banach-Lie 
algebra, which has a characteristic sequence, which decreases slower than any power 
of $n!$. An example for something like this would be $\mathcal{N}_1(n) = 
(\log(n) + 1)^{-n}$ for $n \in \mathbb{N}$.
\begin{proposition}
	Let $\lie{g}$ be a uniformly quasi-nilpotent Banach-Lie algebra and assume that 
	for all $p \geq 1$, we have $\lim_{n \rightarrow \infty} 
	\frac{\mathcal{N}_1(n)}{\sqrt[p]{n!}} = 0$. Then there is an $\E$-like topology 
	on $\algebra{U}(\lie{g})$.
\end{proposition}
\begin{proof}
	Now the idea is to use increasing sequences as counterweights and to modify the 
	$\Tensor_{1^-}$-topology in this way. Therefore, we need 
	to define a series $(\widetilde{\alpha}_n)_{n \in \mathbb{N}}$ with
	\begin{equation*}
		\widetilde{\alpha}
		=
		\log \left(
			\mathcal{N}_1(n)^{-1}
		\right).
	\end{equation*}
	Since $\mathcal{N}_1(n)$ is monotonously decreasing as $\norm{[\xi, \eta]} \leq 
	\norm{\xi} \norm{\eta}$, $\widetilde{\alpha}_n$ will be monotonously increasing.
	It is easy to see, that we can define a new series $(\alpha_n)_{n \in 
	\mathbb{N}}$, which fulfils $\alpha_n \leq \widetilde{\alpha}_n$ for all $n \in 
	\mathbb{N}$ and which is ``convex'', that means for all $n, m \in \mathbb{N}$, 
	$n \leq m$ and $t \in \mathbb{N}$ with $n \leq t \leq m$, we have
	\begin{equation*}
		\alpha_t 
		\leq 
		\alpha_n \left( 1 - \frac{t - n}{m - n} \right) + 
		\alpha_m \frac{t - n}{m - n}.
	\end{equation*}
	We can define a convex, smooth and monotonously increasing function
	\begin{equation*}
		f \colon
		\mathbb{R}_0^+
		\longrightarrow
		\mathbb{R}^+
		, \quad
		\forall_{n \in \mathbb{N}}
		\colon
		f(n) 
		=
		\alpha_n,
	\end{equation*}
	which must fulfil in addition
	\begin{equation*}
		\forall_{p \geq 1}
		\colon
		\lim_{x \rightarrow \infty}
		\frac{p f(x)}{x \log(x)}
		=
		0
		\quad \text{ and } \quad
		\lim_{x \rightarrow \infty}
		\frac{f(x)}{x}
		=
		\infty
	\end{equation*}
	since $\mathcal{N}_1(n)$ decreases more slowly than any power of the factorials, 
	but $\lie{g}$ is still uniformly quasi-nilpotent.
	
	We will need an analogue of the estimate $\binom{n}{m} \leq 2^n$ for the series 
	$\alpha$. This will be achieved by the following lemma.
	\begin{lemma}
		Let $(\alpha_n)_{n \mathbb{N}}$ be a monotonously increasing, convex 
		sequence, such that
		\begin{equation*}
			\lim_{n \rightarrow \infty}
			\frac{\alpha_n}{n \log(n)}
			=
			0
			\quad \text{ and } \quad
			\lim_{n \rightarrow \infty}
			\frac{\alpha_n}{n}
			=
			\infty.
		\end{equation*}
		Then there are constants $b, c \geq 0$, such that for every $n,m \in 
		\mathbb{N}$ with $m \leq n$ the following inequality holds:
		\begin{equation}
		\label{Lemma:LogBinomialEstimate}
			\alpha_n - \alpha_m - \alpha_{n-m}
			\leq
			b + c n.
		\end{equation}
	\end{lemma}
	\begin{subproof}
		This is a logarithmic version of the estimate for the binomial coefficient.
		First note that we can assign to the sequence $\alpha$ a smooth, 
		monotonously increasing and convex function $f$ as we did before.
		Note that $\lim_{x \rightarrow \infty} f'(x) = \infty$, since $f$ is convex 
		and $\lim_{x \rightarrow \infty} \frac{f(x)}{x} = \infty$. Now we can use
		the theorem of de l'H\^{o}spital to get
		\begin{equation*}
			0 
			=
			\lim_{x \rightarrow \infty}
			\frac{f(x)}{x \log(x)}
			=
			\lim_{x \rightarrow \infty}
			\frac{f'(x)}{\log(x) + 1}
			=
			\lim_{x \rightarrow \infty}
			x f''(x)
		\end{equation*}
		and hence the is a $c \geq 0$ with $f''(x) \leq \frac{c}{x}$. 
		Now assume $z = x + y$ for positive real numbers $x,y,z$. By Taylor's 
		theorem, we find
		\begin{align*}
			\tag{I}
			f(z)
			&=
			f(x) + y f'(x)
			+ \int\limits_{x}^{z} (z - t) f''(t) dt
			\\
			\tag{II}
			&=
			f(y) + x f'(y)
			+ \int\limits_{y}^{z} (z - t) f''(t) dt.
		\end{align*}
		For reasons of convexity, the function
		\begin{equation*}
			g(z,x)
			=
			g(z) - g(x) - g(z-x)
		\end{equation*}
		is maximal for fixed $z \in \mathbb{R}^+$ if $2x = z$. So we have proven
		Equation \eqref{Lemma:LogBinomialEstimate}, if we can show 
		$f(2x) - 2f(x) \leq cx + d$. Using (I) for $x = y$, we find
		\begin{equation*}
			f(2x)
			=
			f(x) - x f'(x) +
			\int\limits_x^{2x} (2x-t) f''(t) dt
			\quad
			\Longleftrightarrow
			\quad
			f(2x) - 2 f(x)
			=
			x f'(x) - f(x) +
			\int\limits_x^{2x} (2x-t) f''(t) dt.
		\end{equation*}
		Now we use a backward Taylor development to find
		\begin{align*}
			&&
			f(0)
			&=
			f(x) - xf'(x) + \int_x^0 (0-t) f''(t) dt
			\\
			& \Longleftrightarrow &
			xf'(x) - f(x)
			&=
			\int_0^x t f''(t) dt - f(0)
			\\&&
			& \leq
			\int_0^x t f''(t) dt
			\\&&
			& \leq
			\int_0^x t \frac{c}{t} dt
			\\&&
			&=
			cx.
		\end{align*}
		We then get
		\begin{align*}
			f(2x) - 2 f(x)
			& \leq
			cx + \int\limits_x^{2x} (2x-t) f''(t) dt
			\\
			& \leq
			cx + 
			\int\limits_x^{2x} (2x-t) f\frac{c}{t} dt
			\\
			&=
			cx + cx (\log(4) - 1)
			\\
			&=
			c \log(4) x
		\end{align*}
	\end{subproof}
	Now we can start doing the estimation. Let's define by $(\omega_n)$ a convex, 
	monotonously increasing series, which is smaller equal than the inverse of 
	$\max{\mathcal{N}_1(n), 2}$, but still growing faster than any exponential. This 
	is exactly, what we did before. Define a series $(\alpha)$ by $\alpha_n = 
	\sqrt{\omega_n}$. We will use $\alpha^{\frac{1}{2^m}}$ as counterweights for $m 
	\in \mathbb{N}_0$, in order to get a Fr\'echet topology.
	By doing analogous computations to the first part, we arrive at
	\begin{align*}
		p_\alpha \big(
		&
			\xi_1 \tensor \cdots \tensor \xi_k
			\star_z
			\eta
		\big)
		=
		(k+1) 
		p^k(\xi_1 \tensor \cdots \tensor \xi_k)
		p(\eta)
		\sum\limits_{n=0}^k
		\frac{|B_n^*| |z|^n}{n!}
		\frac{1}{\alpha_{k+1-n} \omega_n}
		\\
		& \leq
		(k+1) 
		p^k(\xi_1 \tensor \cdots \tensor \xi_k)
		p(\eta)
		\frac{1}{\alpha_{k+1}}
		\sum\limits_{n=0}^k
		\frac{|B_n^*| |z|^n}{n! \alpha_n}
		\frac{\alpha_{k+1}}{\alpha_{k+1-n} \alpha_n}
		\\
		& \leq
		(k+1) 
		p^k(\xi_1 \tensor \cdots \tensor \xi_k)
		p(\eta)
		\frac{1}{\alpha_{k+1}}
		\left(
			\sum\limits_{n=0}^k
			\frac{|B_n^*| |z|^n}{n! \alpha_n}
		\right)
		\left(
			\sum\limits_{n=0}^k
			\frac{\alpha_{k+1}}{\alpha_{k+1-n} \alpha_n}
		\right)
		\\
		& \leq
		(k+1) 
		p^k(\xi_1 \tensor \cdots \tensor \xi_k)
		p(\eta)
		\frac{1}{\alpha_{k+1}}
		\kappa_1
		c^{k+1}
		\\
		& \leq
		(k+1) 
		p_{\alpha'}(\xi_1 \tensor \cdots \tensor \xi_k)
		p(\eta)
		\frac{\sqrt{\alpha_k}}{\alpha_{k+1}^{3/4}}
		\kappa_1
		\kappa_2
		\\
		&=
		\kappa
		(k+1)
		\frac{1}{\sqrt[4]{\alpha_{k+1}}}
		\frac{\sqrt{\alpha_k}}{\sqrt{\alpha_{k+1}}}
		p_{\alpha'}(\xi_1 \tensor \cdots \tensor \xi_k)
		p(\eta).
	\end{align*}
	We use the notation $\alpha' = \sqrt{\alpha}$ and $\alpha^{(n)} = 
	\alpha^{\frac{1}{2^n}}$ to keep things readable. We find the estimate on all 
	tensors of degree at most $k$ by the infimum argument and can iterate to higher 
	tensors on the right hand side now:
	\begin{align*}
		p_\alpha \big(
		&
			\xi_1 \tensor \cdots \tensor \xi_k
			\star_z
			\eta_1 \tensor \cdots \tensor \eta_\ell
		\big)
		\\
		& \leq
		\kappa^\ell
		\frac{(k + \ell)!}{k!}
		\frac{\alpha_{k + \ell - 1}'}{\alpha_{k + \ell}' \alpha_{k+\ell}''}
		\frac{\alpha_{k + \ell - 2}''}
		{\alpha_{k + \ell - 1}'' \alpha_{k+\ell-1}^{(3)}}
		\cdots
		\frac{\alpha_{k}^{(\ell)}}
		{\alpha_{k + 1}^{(\ell)} \alpha_{k+1}^{(\ell + 1)}}
		p_{\alpha^{(\ell)}} (\xi_1 \tensor \cdots \tensor \xi_k)
		p^\ell ( \eta_1 \tensor \cdots \tensor \eta_\ell )
		\\
		& \leq
		\kappa^\ell
		2^{k + \ell}
		\ell!
		\frac{1}{\alpha_{k + \ell}'}
		k!
		p^k (\xi_1 \tensor \cdots \tensor \xi_k)
		p^\ell ( \eta_1 \tensor \cdots \tensor \eta_\ell )
		\\
		& \leq
		\kappa^\ell
		2^{k + \ell}
		\frac{k! \ell!}{\alpha_k' \alpha_\ell '}
		p^k (\xi_1 \tensor \cdots \tensor \xi_k)
		p^\ell ( \eta_1 \tensor \cdots \tensor \eta_\ell )
		\\
		&=
		(2p)_{\alpha'} (\xi_1 \tensor \cdots \tensor \xi_k)
		(2 \kappa p)_{\alpha'} ( \eta_1 \tensor \cdots \tensor \eta_\ell ).
	\end{align*}
	This, together with the infimum argument, finishes the proof, since it is clear 
	from the way we can describe the completion of $\left( \Sym^{\bullet}(\lie{g}), 
	\star_z \right)$, that exponential functions are part of it. The continuity of 
	the grading maps is also clear from the construction. Hence the topology is 
	$\E$-like.
\end{proof}
Finally, we can classify quasi-nilpotent Banach-Lie algebras in the following way:
\begin{equation}
	\label{classification}
	\begin{array}{lccl}
		& \text{nilpotent} &
		\\
		& \Downarrow &
		\\
		\text{top. nilpotent } &
		(iii) & \Longleftrightarrow &
		\text{Many strange shit here}
		\\
		& \Downarrow &
		\\
		\text{unif. quasi-nilpotent } &
		(ii) & \Longleftrightarrow &
		\text{Group-like topologies exist on }
		\algebra{U}(\lie{g})
		\\
		& \Downarrow &
		\\
		\text{quasi-nilpotent } &
		(i) & \Longleftrightarrow &
		\text{Baker-Campbell-Hausdorff series converges globally}
	\end{array}
\end{equation}


\section{An Outlook to possible Generalizations and open Questions}



\begin{thebibliography}{1}

\bibitem {bogfjellmo.dahmen.schmedig:2015a}
\textsc{Bogfjellmo, G., Dahmen, R., Schmedig, A.: }\newblock \emph{Character
  groups of Hopf algebras as infinite-dimensional Lie groups}.
\newblock Preprint  \textbf{arXiv: 1410.6468 [math.DG]} (2015).

\bibitem {boseck.czichowski.rudolph:1981a}
\textsc{Boseck, H., Czichowski, G., Rudolph, K.-P.: }\newblock \emph{Analysis
  on topological groups - General Lie Theory}, vol.~37 in \emph{Teubner-Texte
  zur Mathematik [Teubner Texts in Mathematics]}.
\newblock BSB B. G. Teubner Verlagsgesellschaft, 1981.

\bibitem {dixmier:1977a}
\textsc{Dixmier, J.: }\newblock \emph{Enveloping Algebras}.
\newblock \emph{North-Holland Mathematical Library} no. \textbf{14}.
\newblock North-Holland Publishing Co., Amsterdam, New York, Oxford, 1977.

\bibitem {drinfeld:1983a}
\textsc{Drinfel'd, V.~G.: }\newblock \emph{On constant quasiclassical solutions
  of the Yang-Baxter quantum equation}.
\newblock Sov. Math. Dokl.  \textbf{28} (1983), 667--671.

\bibitem {gloeckner.neeb:2012a}
\textsc{Gl{\"o}ckner, H., Neeb, K.-H.: }\newblock \emph{When unit groups of
  continuous inverse algebras are regular Lie groups}.
\newblock Studia Math.  \textbf{211} (2012), 95--109.

\bibitem {goodman:1971a}
\textsc{Goodman, R.: }\newblock \emph{Differential Operators of Infinite Order
  on a Lie Group, II}.
\newblock Indiana Math. J.  \textbf{21} (1971), 383--409.

\bibitem {gutt:1983a}
\textsc{Gutt, S.: }\newblock \emph{An Explicit $*$-Product on the Cotangent
  Bundle of a Lie Group}.
\newblock Lett. Math. Phys.  \textbf{7} (1983), 249--258.

\bibitem {michael:1952a}
\textsc{Michael, E.~A.: }\newblock \emph{Locally Multiplicatively-Convex
  Topological Algebras}.
\newblock \emph{Mem. Amer. Math. Soc.} no. \textbf{11}.
\newblock AMS, Providence, R. I., 1952.

\bibitem {mitiagin.rolewicz.zelazko:1962a}
\textsc{Mitiagin, B.~S., Rolewicz, S., {\.{Z}}elazko, W.: }\newblock
  \emph{Entire functions in $B_0$-algebras}.
\newblock Studia Math.  \textbf{21} (1962), 291--306.

\bibitem {mueller:1994a}
\textsc{M{\"u}ller, V.: }\newblock \emph{Nil, Nilpotent and PI-Algebras}.
\newblock Func. Anal. Op. Theory  \textbf{30} (1994).

\bibitem {pflaum.schottenloher:1998a}
\textsc{Pflaum, M.~J., Schottenloher, M.: }\newblock \emph{Holomorphic
  deformation of Hopf algebras and applications to quantum groups}.
\newblock J. Geom. Phys.  \textbf{28} (1998), 31--44.

\bibitem {rasevskii:1966a}
\textsc{Ra{\v{s}}evski{\u{i}}, P.~K.: }\newblock \emph{Associative
  hyper-envelopes of Lie algebras, their regular representations and ideals}.
\newblock Trans. Mosc. Math. Soc.  \textbf{15} (1966), 3--54.

\bibitem {waldmann:2014a}
\textsc{Waldmann, S.: }\newblock \emph{A nuclear Weyl algebra}.
\newblock J. Geom. Phys.  \textbf{81} (2014), 10--46.

\bibitem {wojtynski:1998a}
\textsc{Wojtynski, W.: }\newblock \emph{Quasi-nilpotent 
{B}anach-{L}ie-Algebras are {B}aker-{C}ampbell-{H}ausdorff}.
\newblock J. Func. Anal.  \textbf{153} (1998), 405--413.

\end{thebibliography}


% shulman.turovskii

% galindo.palacios

\end{document}

%%% Local Variables:
%%% mode: latex
%%% TeX-master: t
%%% End:
