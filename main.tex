%
% A new paper...
% Title: Convergence of the Gutt Star Product
% git-repository is gstar
%
% From now on, we proudly use the chairx style file for
% everything. All stuff in there is soo useful!
%


%
% Before we start, let's nag a bit about old latex constructs
% just to learn... This will be removed in the final version
%

\RequirePackage[l2tabu, orthodox]{nag}


%
% we always start with 11pt, draft mode on for easier editing and
% english as default language
%

\documentclass[
11pt,                          % standard font size
%draft,                         % draft or final?
english                        % standard language
]{article}


%
% some macro packages
%

\usepackage[english]{babel}    % with explicit language
\usepackage{amsmath}           % ams mathematical stuff
\usepackage[utf8]{inputenc}    % smart input of funny chars
\usepackage[T1]{fontenc}       % also for the font encoding
\usepackage{longtable}         % tables longer than one page
\usepackage{exscale}           % large summation signs in 11pt
\usepackage[final]{graphicx}   % to include pdf pictures
\usepackage[sort]{cite}        % nicer citations
\usepackage{array}             % nice tables
\usepackage{wasysym}           % smiley symbols
\usepackage[a4paper]{geometry} % geometry of page layout
%\usepackage{gitinfo2}          % include git info: Version 2
%\usepackage[multiuser]{fixme}  % correction notes, warnings etc.
\usepackage{xspace}            % better spacing after macros
\usepackage{tikz}              % for commutative diagrams and stuff
\usepackage{ifdraft}           % to determine whether draft mode
\usepackage{chairx}            % the Chair X style file
\usepackage[expansion=false    % no font expansion
           ]{microtype}        % only protrusion
\usepackage[nottoc]{tocbibind} % refs and index in the toc
\usepackage[backref=page,      % backrefs in the bibliography
           final=true,         % always treat as final
           pdfpagelabels       % use pdf page labels
           ]{hyperref}         % hyperrefs are cool!

%
% Some own macros...
%

%\usepackage{mnsymbol}

\newcommand{\ostar}{\star}
\newcommand{\coproduct}{\Delta}
\newcommand{\ocoproduct}{\Delta}

\newcommand{\Eta}{\mathrm{H}}
\newcommand{\Rho}{\operatorname{P}}
\newcommand{\bch}[2]{\mathrm{BCH}\left(#1, #2\right)}
\newcommand{\bchpart}[3]{\mathrm{BCH}_{#1}\left(#2, #3\right)}
\newcommand{\bchparts}[4]{\mathrm{BCH}_{#1, #2}\left(#3, #4\right)}
\newcommand{\bchtilde}[4]{\widetilde{\mathrm{BCH}}_{#1, #2}\left(#3; #4\right)}
\newcommand\ot[2]{\stackrel{\mathclap{#1}}{#2}}
%
%\mathrel{\overset{\makebox[0pt]
%	{\mbox{\normalfont\footnotesize\sffamily #1}}}{#2}}}


%
% pdf files for graphics in the following directory:
%

\graphicspath{{../tikz/}}


%
% tikz libraries to be loaded, feel free to add more...
%

\usetikzlibrary{matrix}
\usetikzlibrary{arrows}
\usetikzlibrary{patterns}
\usetikzlibrary{decorations.pathreplacing}


%
% page dimensions, scaling etc. Not final yet
%


\geometry{bindingoffset=0cm}
\geometry{hcentering=true}
\geometry{hscale=0.8}
\geometry{vscale=0.8}


%
% check whether draft or not: synctex is soo cool
%

%\ifdraft{\synctex=1}{}


%
% fix me settings
%

%\fxusetheme{color}


%
% Get the authors from external file
% This used in all documents of the paper project
%

%\input{../authors/authors}


%
% own local math macros follow here
%


%
% title page for Convergence of the Gutt Star Product
% authors are included from the authors file
% This has to be adapted in the final version
%

\title{(Quasi-)Nilpotent Lie Algebras and their Universal Envelopes}

\author{
  \textbf{Paul Stapor}\thanks{\texttt{paul.stapor@stud-mail.uni-wuerzburg.de}},
  \addtocounter{footnote}{2}
  \\[0.5cm]
  \chairXaddress
}

%\date{Current Version of gstar: \gitAuthorIsoDate\\[0.2cm]
%  {\small
%    Last changes by \gitAuthorName{} on \gitAuthorDate \\
%    Git revision of gstar: \texttt{\gitAbbrevHash{}} \gitReferences
%  }
%}

\date{September 2015}


%
% the text starts here
%

\begin{document}

%
% title page
%

\maketitle

%
% abstract
%

\begin{abstract}
    Nilpotency is a well-studied property in finite-dimensional algebras, but the 
    transition to topological infinite-dimensional algebras is not easy, since there 
    are many inequivalent notions of so called ``quasi-nilpotency'', which all imply 
    true nilpotency in finite dimensions. In this work, we translate some notions 
    from the better known associative theory to Lie algebras, in particular 
    Banach-Lie algebras, clarify their relations among each other and give many 
    examples. We will see that these notions can be linked to possible topologies on 
    the universal enveloping algebra $\algebra{U}(\lie{g})$ of a Banach-Lie algebra 
    $\lie{g}$ and show that one of them is equivalent to the existence of 
    topologies which allow group-like elements in their completions. Finally, we 
    generalize some of these results to locally convex Lie algebras.
\end{abstract}

\newpage

%
% table of contents
%

\tableofcontents

\newpage

%
% Introduction
%

\section{Introduction}
\label{sec:Introduction}

If one wants to study the universal enveloping algebra $\algebra{U}(\lie{g})$ of a 
given Lie algebra $\lie{g}$ in a topological setting, there are different 
possibilities to see it. The most geometric one are invariant differential operators 
on the corresponding Lie group $G$. Another interpretation are compactly supported 
distributions near the unit element of $G$. Both approaches are a priori bound to 
finite-dimensional Lie algebras. While the first is mostly used in the theory of 
quantum groups \cite{pflaum.schottenloher:1998a}, the second allows some more 
generalizations to infinite-dimensional Lie algebras, see e.g. 
\cite{beltita.nicolae:2015a}. There is yet another correspondence, which is 
completely independent of the dimensions of $\lie{g}$ and uses the 
Poincar\'e-Birkhoff-Witt Theorem: as a vector space, the symmetric tensor algebra 
$\Sym^{\bullet}(\lie{g})$ is isomorphic to $\algebra{U}(\lie{g})$ via the total 
symmetrization. So one can also pull-back the non-commutative product to the 
symmetric algebra and topologize this deformed algebra. This approach was first 
discovered by Rasevski{\u i} cite{rasevskii:1966a} and Goodman \cite{goodman:1971a}, 
but seems to have gone rather unnoticed. These first attempts were bound to 
finite-dimensional Lie algebras, while a younger work 
\cite{esposito.stapor.waldmann:2015a:pre}, from the point of view of deformation 
quantization, uses a similar idea to construct an explicit topology in a slightly 
different way on universal enveloping algebras of Banach- and certain locally convex 
Lie algebras.

A very important aim of this last article was to construct a very coarse locally 
convex topology on $\algebra{U}(\lie{g})$, in order to get a large completion 
$\widehat{\algebra{U}}(\lie{g})$. Since the elements in $\algebra{U}(\lie{g})$ can be 
interpreted as polynomials, the elements in $\widehat{\algebra{U}}(\lie{g})$ are 
certain entire functions. One important result was, that nilpotent locally convex Lie 
algebras allow bigger completions: here, one can construct a topology, which contains 
exponentials series $\exp(\xi)$ for every $\xi \in \lie{g}$, while for non-nilpotent 
Lie algebras, this is not possible, at least with this particular construction based 
on the symmetric algebra.

The aim of the present work is to investigate further, when such a coarser topology 
and hence a bigger completion is possible. Therefore, we first restrict to the easier 
case of Banach-Lie algebras. We show that for general (Banach-) Lie algebras, the 
$\Tensor_R$-topology from \cite{esposito.stapor.waldmann:2015a:pre} is as coarse as 
possible. Since moreover, in Banach-Lie algebras, there exists weaker forms of 
nilpotency, it is a neat question to ask whether topologies with bigger completions 
containing exponentials (which we want to call grouplike topologies on $\algebra{U}
(\lie{g})$), can exist for quasi-nilpotent Banach-Lie algebras. It is the main result 
of this work to show that there is a particular form of quasi-nilpotency, which is 
equivalent to the existence of such a topology on $\algebra{U}(\lie{g})$.

Although nilpotency is a very well-studied phenomenon, the situation in topological 
infinite-dimensional algebras is less clear. Here, there are different and 
inequivalent generalizations possible, which all imply real nilpotency in finite 
dimensions. The probably weakest forms, quasi-nilpotency, means that the spectral 
radius of every algebra element is zero. A big part of the work in this area deals 
with normed associative algebras and for certain algebras of compact operators, 
important results have been found, like in \cite{shulman.turovskii:2000a}. But, there 
is not yet a closed theory of quasi-nilpotency. Nevertheless, different notions of 
quasi-nilpotency are worked out in the associative case and links between them are 
known, see e.g. \cite{dixon:1991a, mueller:1994a, dixon.mueller:1992a}. Especially 
joint quasi-nilpotency of bounded subsets, so called topological nilpotency, is an 
interesting field of study. This idea originated from a work of Rota and Strang 
\cite{rota.strang:1960a} and was carried on (e.g. \cite{mueller:1997a}). A 
particularly nice paper from Galindo and Palacios \cite{galindo.palacios:2012a} also 
deals with topological nilpotency in non-associative normed algebras. But still, 
besides Wojty\'nski's theorem that a Banach-Lie algebra is quasi-nilpotent if and 
only if it has a globally convergent Baker-Campbell-Hausdorff series 
\cite{wojtynski:1998a}, further particular results or even a good notion of different 
types of quasi-nilpotency for Banach-Lie algebras do not seem to exist.

This work is organized as follows: In Section 2, we present the main techniques and 
the construction of the already mentioned $\Tensor_R$-topology on a universal 
enveloping algebra via the Poincar\'e-Birkhoff-Witt Theorem and use this possibility 
to present a new result for locally convex, nilpotent Lie algebras in this context. 
Then we specify what kind of topology we would like to achieve on $\algebra{U}
(\lie{g})$ and show, that these projective grouplike topologies are bound to the case 
of quasi-nilpotent Banach-Lie algebras. Section 3 deals with different types of 
quasi-nilpotency in Banach-Lie algebras and tries to transfer some of the notions 
from the associative to the Lie case. We also show that, as it is the case for 
associative Banach algebras, topological nilpotency and uniform topological 
nilpotency coincide for Banach-Lie algebras and finally give an example for each type 
of quasi-nilpotent structure. Section 4 contains the main result of this 
work: we prove that so called uniform quasi-nilpotency of a Banach-Lie algebra 
$\lie{g}$ is equivalent to the existence of a projective grouplike topology on 
$\algebra{U}(\lie{g})$ by an explicit construction. Finally, in section 5, we state 
some immediate corollaries from our results and show some possible ways and bounds 
for a generalization to locally convex Lie algebras.

Acknowledgements: I want to thank Stefan Waldmann, Chiara Esposito and Matthias 
Sch\"otz for fruitful discussion on the topic as well as Karl-Hermann Neeb for a 
helpful remark.


\section{A Topology on the Universal Enveloping Algebras}

\subsection{Preliminaries and the $\Tensor_R$-Topology}
As already mentioned, one possibility to see the universal enveloping algebra 
$\algebra{U}(\lie{g})$ of a Lie algebra $\lie{g}$ is the symmetric tensor algebra 
$\Sym^{\bullet}(\lie{g}) = \bigoplus_{n=0}^\infty \Sym^n(\lie{g})$ via the 
Poincar\'e-Birkhoff-Witt theorem. One uses the isomorphism
\begin{equation}
	\label{eq:pbw-isomorphism}
	\sigma
	=
	\sum\limits_{n=0}^\infty
	\sigma_n
	\quad \text{ with }
	\sigma_n
	\colon
	\Sym^n(\lie{g})
	\longrightarrow
	\algebra{U}(\lie{g})
	, \
	\xi_1 \vee \cdots \vee \xi_n
	\longmapsto
	\frac{1}{n!}
	\sum\limits_{\tau \in S_n}
	\xi_{\tau(1)} \cdots \xi_{\tau(n)}
\end{equation}
to pull back the product from $\algebra{U}(\lie{g})$ to $\Sym^{\bullet}(\lie{g})$. 
In detail, the construction is done using a parameter $z \in \mathbb{K}$ where 
$\mathbb{K} = \mathbb{R}$ or $\mathbb{C}$ and the canonical projections $\pi_n 
\colon \Sym^{\bullet}(\lie{g}) \longrightarrow \Sym^n(\lie{g})$ from the grading:
\begin{equation}
	\label{eq:starproduct}
	\star_z
	\colon
	\Sym^k(\lie{g})
	\times
	\Sym^\ell(\lie{g})
	\longrightarrow
	\Sym^{\bullet}(\lie{g})
	, \quad
	(a, b)
	\longmapsto
	\sum\limits_{n=0}^{k + \ell - 1}
	z^n
	\left( \pi_{k + \ell - n} \circ \sigma^{-1} \right)
	\left(
		\sigma(a)
		\cdot
		\sigma(b)
	\right).
\end{equation}
Then it is extended bilinearly to the whole symmetric algebra. This associative 
product is sometimes called Gutt star product, since Gutt was the first to describe 
it in the sense of deformation theory in \cite{gutt:1983a}.
The parameter $z$ is motivated from deformation theory and allows to treat the 
symmetric and the universal enveloping algebra at the same time, since the former is 
obtained putting $z = 0$ and the latter setting $z = 1$. In this way, also the 
properties of the product can be studied by letting vary the parameter $z$ and hence 
making $(\Sym^{\bullet}(\lie{g}), \star_z)$ ``more or less'' commutative. Since this 
allows to treat everything from an additional point of view, we will keep this so 
called deformation parameter in our equations. Since one aim of this paper is to 
extend this approach and to prove a link between possible topologies on $\algebra{U}
(\lie{g})$ and the properties of the Lie bracket of $\lie{g}$, we will briefly
introduce these concepts.

First we need some explicit formulas for the product $\star_z$. Since we will 
mostly deal with certain quasi-nilpotent Banach-Lie algebras, it will be enough to 
control the product of a factorizing symmetric tensor $\xi_1 \vee \cdots \vee \xi_k$ 
with a vector $\eta$. Explicit formulas for this are known since a long time, e.g. 
\cite[Prop. 1]{gutt:1983a}, \cite[2.8.12 (c)]{dixmier:1977a} or \cite[Prop. 2.6]
{esposito.stapor.waldmann:2015a:pre}:
\begin{align}
    \label{eq:gstarformulaR}
    \xi_1 \vee \cdots \vee \xi_k \star_z \eta
    &=
    \sum\limits_{j=0}^k
    \frac{1}{k!} \binom{k}{j}
    z^j B_j^*
    \sum\limits_{\sigma \in S_k}
    \xi_{\sigma(1)} \vee \cdots \vee \xi_{\sigma(k - j)} \vee
    [\xi_{\sigma(k - j + 1)},
    [ \ldots [\xi_{\sigma(k)}, \eta] \ldots ]
    ]
    \\
    \label{eq:gstarformulaL}
    \eta \star_z \xi_1 \vee \cdots \vee \xi_k
    &=
    \sum\limits_{j=0}^k
    \frac{1}{k!} \binom{k}{j}
    z^j B_j
    \sum\limits_{\sigma \in S_k}
    \xi_{\sigma(1)} \vee \cdots \vee \xi_{\sigma(k - j)} \vee
    [\xi_{\sigma(k - j + 1)},
    [ \ldots [\xi_{\sigma(k)}, \eta] \ldots ]
    ]
    .    
\end{align}
The constants $B_j$ are the Bernoulli numbers, which are defined by
\begin{equation}
	\frac{x}{\E^x - 1}
	=
	\sum\limits_{j=0}^\infty
	\frac{B_j}{j!}
	x^j
\end{equation}
and $B_j^* = (-1)^j B_j$. There are also formulas for more involved products, but 
we will not introduce them here, since \eqref{eq:gstarformulaL} and 
\eqref{eq:gstarformulaR} will suffice for our purposes. Finally, we extend this 
product to the whole tensor algebra by symmetrizing beforehand which will be very 
useful for technical reasons later:
\begin{equation}
	\label{eq:extended-gstar}
	\star_z
	\colon
	\Tensor^{\bullet}(\lie{g})
	\times
	\Tensor^{\bullet}(\lie{g})
	\longrightarrow
	\Sym^{\bullet}(\lie{g})
	, \quad
	\star_z
	=
	\star_z
	\circ
	\left(
		\Symmetrizer \times \Symmetrizer
	\right),
\end{equation}
where $\Symmetrizer \colon \Tensor^{\bullet}(\lie{g}) \longrightarrow \Sym^{\bullet}
(\lie{g})$ denotes the total symmetrization. Clearly, this really extends $\star_z$, 
since $\Symmetrizer$ is the identity on $\Sym^{\bullet}(\lie{g})$. 

Now we topologize the tensor algebra $\Tensor^{\bullet}(\lie{g})$ in a locally convex 
manner and obtain $\Sym^{\bullet}(\lie{g})$ as a closed subspace, since the total 
symmetrization will be a continuous map and $\Sym^{\bullet}(\lie{g}) = \ker (\id - 
\Symmetrizer)$. Therefore, we start with a locally convex Lie algebra $\lie{g}$ and 
call the set of all continuous seminorms $\algebra{P}$. For every $\halbnorm{p} \in 
\algebra{P}$, the projective tensor product $\halbnorm{p}^n = \halbnorm{p} 
\tensor[\pi] \cdots \tensor[\pi] \halbnorm{p}$ ($n$-times) defines a seminorm on 
$\Tensor^n(\lie{g})$ and fulfils $\halbnorm{p}^n(\xi_1 \tensor \cdots \tensor \xi_n) 
= \halbnorm{p}(\xi_1) \cdots \halbnorm{p}(\xi_n)$. We can use the graded structure of 
the tensor algebra to piece those seminorms together using weights. For every 
$R \geq 0$ we can define the $\Tensor_R$-topology.
\begin{definition}[$\Tensor_R$-Topology]
	\label{def:TR-topology}
	Let $R \geq 0$ and $\halbnorm{p} \in \algebra{P}$, then 
	\begin{equation}
		\label{def:TR-seminorms}
		\halbnorm{p}_R
		=
		\sum\limits_{n=0}^\infty
		n!^R \halbnorm{p}^n
	\end{equation}
	gives a continuous seminorm on $\Tensor^{\bullet}(\lie{g})$. 
	The topology on the tensor algebra, which is
	defined by all of those $\halbnorm{p}_R$ for a fixed $R$ will be called the 
	$\Tensor_R$-topology and the topological algebra will be called 
	$\Tensor_R^{\bullet}(\lie{g})$. We also define for $R > 0$ the projective limit
	\begin{equation}
		\label{def:TR-projective}
		\Tensor_{R^-}^{\bullet}(\lie{g})
		=
		\projlim\limits_{\varepsilon \rightarrow 0}
		\Tensor_{R - \varepsilon}^{\bullet}(\lie{g}).
	\end{equation}
\end{definition}
Note that this construction is possible for every locally convex vector space $V$ 
and turns $\Tensor_R^{\bullet}(V)$ (together with the tensor product) into a locally 
convex algebra. The seminorms $\halbnorm{p}_R$ are submultiplicative, i.e. 
$\forall_{x, y \in \Tensor_R^{\bullet}(V)}$
\begin{equation}
	\label{eq:TR-submultiplicative}
	\halbnorm{p}_R(x \tensor y)
	\leq
	\halbnorm{p}_R(x)
	\halbnorm{p}_R(y),
\end{equation}
if and only if $R = 0$. A very detailed discussion of this topology can be found in 
\cite{waldmann:2014a}. We mention some basic facts: for $R' > R$, the 
topology of $\Tensor_{R'}^{\bullet}(\lie{g})$ is strictly finer than the one of 
$\Tensor_R^{\bullet}(\lie{g})$ and hence the completion 
$\widehat{\Tensor}_{R'}^{\bullet}(\lie{g})$ is smaller than for $R < R'$.
Those completions can be described explicitly as subspaces of the cartesian product 
$\prod_{n=0}^\infty \Tensor^n(\lie{g})$ by
\begin{equation}
	\label{eq:TR-completion-explicit}
	\widehat{\Tensor}_R^{\bullet}(\lie{g})
	=
	\left\{
	\left.
		x
		=
		\sum\limits_{n = 0}^\infty
		x^n
		\in
		\prod_{n=0}^\infty \Tensor^n(\lie{g})
	\ \right| \ 
		\forall_{\halbnorm{p} \in \algebra{P}}
		\forall_{c > 0}
		\colon
		\sum\limits_{n = 0}^\infty
		n!^R c^n p^n \left( x^n \right)
		<
		\infty
	\right\}.
\end{equation}
The $c > 0$ in \eqref{eq:TR-completion-explicit} is actually superfluous, but we put 
it there to underline the exponential behaviour of $\halbnorm{p}_R$ when the seminorm 
$\halbnorm{p}$ gets rescaled by $c > 0$.

The $\Tensor_R$-topology has, due to projective tensor product, a 
very good feature: if we need to show a continuity estimate on arbitrary tensors 
$x,y \in \Tensor^{\bullet}(\lie{g})$, it is enough to show it on factorizing tensors 
$\xi_1 \tensor \cdots \tensor \xi_k$ and $\eta_1 \tensor \cdots \tensor \eta_\ell$ 
for arbitrary $k, \ell \in \mathbb{N}$. Once this is done, one uses the so called 
``infimum argument'' to get it an all tensors. This is a technical property of the 
projective tensor product and a detailed explanation can be found in
\cite[Proposition 3.2]{esposito.stapor.waldmann:2015a:pre}.

We say the Lie bracket in $\lie{g}$ is jointly continuous, if for every 
$\halbnorm{p} \in \algebra{P}$, there is a $\halbnorm{q}$ such that for all $\xi, 
\eta \in \lie{g}$
\begin{equation}
	\label{eq:continuity-of-liebracket}
	\halbnorm{p}([\xi, \eta])
	\leq
	\halbnorm{q}(\xi) \cdot \halbnorm{q}(\eta).
\end{equation}
It was one of the main results in \cite{esposito.stapor.waldmann:2015a:pre}, that 
the product $\star_z$ is continuous on $\Sym_R^{\bullet}(\lie{g})$ if and only if 
$R \geq 1$, under the condition that $\lie{g}$ is an AE-Lie algebra.
\begin{definition}[AE-algebra]
	\label{def:AE-algebra}
	Let $\algebra{A}$ be a not necessarily associative locally convex algebra
	and $\halbnorm{p}$ a continuous seminorm on it. Then we call
	another continuous seminorm $\halbnorm{q}$ an asymptotic estimate for 
	$\halbnorm{p}$, if for every $n \in \mathbb{N}$ and all $\xi_1, \ldots, \xi_n
	\in \algebra{A}$, the estimate
	\begin{equation}
		\label{eq:def:AE-property}
		\halbnorm{p}\left(
			w(\xi_1, \ldots, \xi_n)
		\right)
		\leq
		\halbnorm{q}(\xi_1) \cdots \halbnorm{q}(\xi_n),
	\end{equation}
	where the notation $w(\xi_1, \ldots, \xi_n)$ means, this should hold for a word 
	in the letters $\xi_1, \ldots, \xi_n$ and any
	way of setting the brackets in the multiplication. A locally convex (Lie) 
	algebra, in which every continuous seminorm has an asymptotic estimate will be 
	called an AE(-Lie) algebra.
\end{definition}
AE algebras are ``almost'' locally multiplicatively convex (for short: lmc) 
algebras, meaning they have nearly submultiplicative seminorms. An associative AE 
algebra has for example an entire calculus. In this sense, the AE condition 
\eqref{eq:def:AE-property} is very restrictive for a locally convex algebra, but 
still very weak in general, since every normed (and hence every finite-dimensional) 
algebra is AE, for example.

It is not clear, if all AE algebras are indeed lmc, but for associative, commutative 
Fr\'echet algebras, which are AE, this is the case 
\cite{mitiagin.rolewicz.zelazko:1962a}. It seems like there is not much known about 
this particular subclass of locally convex algebras. The first time, this expression 
appeared, was in a work by Boseck, Czichowski and Rudolph 
\cite{boseck.czichowski.rudolph:1981a}, but in a different sense. They use the same 
notion of asymptotic estimate seminorms, but an AE algebra in their definition has to 
fulfil additional, technical condition, which make those AE algebras lmc. We want to 
use the weaker notion of AE algebra we just defined above. Equivalent definitions to 
ours (using different names) where used in \cite{gloeckner.neeb:2012a} and 
\cite{bogfjellmo.dahmen.schmedig:2015a}, for example.


If the Lie algebra $\lie{g}$ is nilpotent and locally convex, it is automatically 
AE. In this case, continuity of $\star_z$ can already be achieved in the projective 
limit $\Sym_{1^-}^{\bullet}(\lie{g})$. This is important, since the completion of 
this projective limit is bigger than $\widehat{\Sym}_1^{\bullet}(\lie{g})$: for $R = 
1^-$, the exponential series of $\xi \in \lie{g}$ is in the completion for every 
$\xi \in \lie{g}$, whereas for $R = 1$, there are no exponentials. One aim of this 
paper is to find a certain type of quasi-nilpotency, which allows still such a 
projective limit and a continuous product.

Finally, we should mention that the construction of the $\Tensor_1$- and the 
$\Tensor_{1^-}$-topology is functorial: continuous morphisms between (nilpotent) 
AE-Lie algebras $\lie{g}$ and $\lie{h}$ lift to continuous morphisms between the 
algebras $\Sym_1^{\bullet}(\lie{g})$ and $\Sym_1^{\bullet}(\lie{h})$ or 
$\Sym_{1^-}^{\bullet}(\lie{g})$ and $\Sym_{1^-}^{\bullet}(\lie{h})$, respectively, 
and hence to continuous morphisms of the topologized universal enveloping algebras.
In this sense, the $\Tensor_R$-topologies behave ``reasonably well''.



\subsection{Nilpotent locally convex Lie-Algebras}
As a first result in this work, we want to show that for nilpotent, locally convex 
Lie algebras, we do not only find the algebras $\Sym_R^{\bullet}(\lie{g})$ for 
$R \geq 1^-$, but also a rich structure of bimodules. For $R < 1$, 
$\left( \Sym_R^{\bullet}(\lie{g}), \star_z \right)$ will not be a locally convex 
algebra any more, but it is still a vector space and a bimodule over 
$\Sym_{R'}^{\bullet}(\lie{g})$ for a sufficiently big $R' \geq 1$.
\begin{proposition}
	\label{prop:bimodules}
	Let $\lie{g}$ be a nilpotent, locally convex Lie algebra and $N \in \mathbb{N}$ 
	the smallest number, such that $\ad_{\lie{g}}^{N+1}(\lie{g}) = \{ 0 \}$. Let 
	moreover $z \in \mathbb{K}$ and $\halbnorm{p}$ be a submultiplicative seminorm.
	\begin{propositionlist}
		\item
		For all $\xi_1, \ldots, \xi_k, \eta \in \lie{g}$ and all $R \geq 0$ we have 
		the estimate
		\begin{equation}
			\halbnorm{p}_R
			\left(
				\xi_1 \tensor \cdots \tensor \xi_k \star_z \eta
			\right)
			\leq
			c (k+1)^R k^{N(1-R)}
			\halbnorm{p}_R(\xi_1 \tensor \cdots \tensor \xi_k)
			\halbnorm{p}(\eta)
		\end{equation}
		with the constant $c = \sum_{n = 0}^N \frac{|B_n^*| |z|^n}{n!}$.
		
		\item
		The vector space $\widehat{\Sym}_R^{\bullet}(\lie{g})$ is a bimodule over 
		the algebra $\widehat{\Sym}_{R + N(1-R)}^{\bullet}(\lie{g})$.
	\end{propositionlist}
\end{proposition}
\begin{proof}
	Let $\halbnorm{p}$ be a continuous seminorm and $R \geq 0$. We need to
	compute an estimate using \eqref{eq:gstarformulaR}:
	\begin{align*}
		&
		\halbnorm{p}_R
		\big(
			\xi_1 \tensor \cdots \tensor \xi_k \star_z \eta
		\big)
		\\
		&=
	    \sum\limits_{j=0}^N
        \frac{1}{k!} \binom{k}{j}
        |z|^j |B_j^*|
        (k - n + 1)!^R
        \halbnorm{p}^{k - n + 1}
        \left(
            \sum\limits_{\sigma \in S_k}
            \xi_{\sigma(1)} \vee \cdots \vee \xi_{\sigma(k - j)} \vee
            [\xi_{\sigma(k - j + 1)},
            [ \ldots [\xi_{\sigma(k)}, \eta] \ldots ]
            ]
        \right)
        \\
        & \leq
        (k + 1)^R
	    \sum\limits_{j=0}^N
        \frac{|z|^j |B_j^*|}{j! (k-j)!^{1-R}}
        k! 
        \halbnorm{p}(\xi_1) \cdots \halbnorm{p}(\xi_k)
        \halbnorm{p}(\eta)
        \\
        & =
        (k + 1)^R
	    \sum\limits_{j=0}^N
        \frac{|z|^j |B_j^*|}{j!}
        \frac{k!^{1 - R}}{(k-j)!^{1-R}}
        k!^R
        \halbnorm{p}(\xi_1 \tensor \cdots \tensor \xi_k)
        \halbnorm{p}(\eta)
        \\
        & \leq
        (k + 1)^R
        k^{N (1-R)}
	    \sum\limits_{j=0}^N
        \frac{|z|^j |B_j^*|}{j!}
        \halbnorm{p}_R(\xi_1 \tensor \cdots \tensor \xi_k)
        \halbnorm{p}(\eta).
	\end{align*}
	We should mention here, that the estimate for the left instead of the right
	multiplication is exactly the same, since we only have to replace $B_j$ by 
	$B_j^*$ and we have $|B_j| = |B_j^*|$.
	
	For the second part, keep in mind that using the infimum argument, we have the 
	estimate for all tensors $x \in \bigoplus_{n=0}^k \Tensor^n(\lie{g})$. We can 
	now iterate this estimate, since
	\begin{equation*}
		\left( \xi_1 \tensor \cdots \tensor \xi_k \right)
		\star_z
		\left( \eta_1 \tensor \cdots \tensor \eta_\ell \right)
		=
		\frac{1}{\ell!}
		\sum\limits_{\tau \in S_\ell}
		\left( \xi_1 \tensor \cdots \tensor \xi_k \right)
		\star_z
		\left( \eta_{\tau(1)} \star_z \cdots \star_z \eta_{\tau(\ell)} \right).
	\end{equation*}
	This leads to
	\begin{align*}
		\halbnorm{p}_R
		\big(
			(\xi_1
		&	
			 \tensor \cdots \tensor \xi_k )
			 \star_z 
			 (\eta_1 \tensor \cdots \tensor \eta_\ell)
		\big)
		\\
		& \leq
		\frac{(k + \ell)!^R}{k!^R}
		\left(
			\frac{(k + \ell - 1)!}{(k - 1)!}
		\right)^{N (1 - R)}
		c^\ell
		\halbnorm{p}_R( \xi_1 \tensor \cdots \tensor \xi_k )
		\halbnorm{p}( \eta_1 ) \cdots \halbnorm{p} ( \eta_\ell )
		\\
		& \leq
		2^{(k + \ell) R}
		\ell!^R
		2^{(k + \ell) N (1 - R)}
		\ell!^{N (1 - R)}
		c^\ell
		\halbnorm{p}_R( \xi_1 \tensor \cdots \tensor \xi_k )
		\halbnorm{p}( \eta_1 \tensor \cdots \tensor \eta_\ell )
		\\
		& \leq
		(2^{N + 1} \halbnorm{p})_R
		( \xi_1 \tensor \cdots \tensor \xi_k )
		(2^{N + 1} c \halbnorm{p})_{R + N (1 - R)}
		( \eta_1 \tensor \cdots \tensor \eta_\ell ).
	\end{align*}
	Using the infimum argument and knowing that this estimate works for both sides,
	we find the statement from the proposition.
\end{proof}
\begin{remark}
	The same result can be achieved for simply locally convex Lie algebras, omitting 
	submultiplicativity of the seminorms. It uses a more 
	involved formula for two factorizing tensors. Since the advantage is not so big,
	we omit this proof here. It can be found in \cite{stapor:2015a}.
\end{remark}
These module structures only exist for nilpotent Lie algebras. It is easy 
to construct a counter-example: take the solvable Lie algebra generated by $x, y$ 
with the relation $[x,y] = y$ and use a norm such that $\norm{x} = \norm{y} = 1$. 
For any $R < 1$ and any $\varepsilon > 0$ such that $R + \varepsilon < 1$, the 
sequence $(a) \in \Sym_R^{\bullet}(\lie{g})$ with $a_n = \frac{x^n}{n!^{R + 
\varepsilon}} \in \Sym^n(\lie{g})$ converges to zero. But the product $a_n \star_z 
y$ is divergent and can not be controlled by whatever $R' > 1$, since 
$\halbnorm{p}_{R'}(y) = \halbnorm{p}(y) = 1$ for all $R'$.





\subsection{Grouplike Topologies}
Obviously, the nilpotency of a Lie-algebra $\lie{g}$ allows certain features for 
topologies on $\algebra{U}(\lie{g})$ which a non-nilpotent Lie algebras does not 
allow, at least for the $\Tensor_R$-topology. 
Especially the size of the completion and the question if exponentials can be 
obtained or not is very interesting and we want to know, if there may be 
other topologies on $\algebra{U}(\lie{g})$, which allow bigger completions. To make 
this more concrete, we give the following definition.
\begin{definition}[Projective grouplike Topologies]
	\label{def:grouplike-topology}
	Let $\lie{g}$ be a Banach-Lie algebra and $\tau$ a locally convex topology on 
	$\algebra{U}(\lie{g})$. Then we say $\tau$ is a projective grouplike topology, 
	if the three following things hold:
	\begin{definitionlist}
		\item
		The multiplication in $(\algebra{U}(\lie{g}), \tau)$ is continuous.
		
		\item
		For every $\xi \in \lie{g}$, the exponential series $\sum_{n=0}^\infty 
		\frac{\xi^n}{n!}$ converges in the completion of $(\algebra{U}(\lie{g}), 
		\tau)$.
		
		\item
		The projection 
		\begin{equation}
			\label{eq:projection}
			\pi 
			= 
			\pi_1 \circ \sigma^{-1}
			\colon
			\algebra{U}(\lie{g})
			\longrightarrow
			\lie{g}
		\end{equation}
		and the embedding
		\begin{equation}
			\label{eq:embedding} 
			\iota \colon \lie{g} \longrightarrow \algebra{U}(\lie{g})
		\end{equation}	
		are continuous.
	\end{definitionlist}
	If point $(iii.)$ is not fulfilled, we call $\tau$ just grouplike.
\end{definition}
From any topology on the universal enveloping algebra, we would expect the first 
point. The second was motivated from the discussion before and is of particular 
interest concerning the integrability of the Lie algebra. The third point 
may look a bit arbitrary on the first sight, but it has its own motivation: if we 
start with a Banach-Lie algebra $\lie{g}$, we would certainly like it to carry the 
same topology as a subspace of $\widehat{\algebra{U}}(\lie{g})$ after embedding it 
and taking the completion, everything else seems pointless. But we do not need a 
\emph{continuous projection} onto $\lie{g}$. A continuous homeomorphism would 
suffice. However, $\pi$ is the best candidate for such a homeomorphism and if we
see $\algebra{U}(\lie{g})$ as $(\Sym^{\bullet}(\lie{g}), \star_z)$, it 
is even the only canonical candidate. This argument is even more intuitive in the 
finite-dimensional case: here, there are \emph{always} continuous projections onto 
$\lie{g}$ by the Hahn-Banach theorem. So we would like the most intuitive one to be 
continuous, too. This justifies point three.

We want to show that asking for the existence of a projective grouplike topology 
on $\algebra{U}(\lie{g})$ puts strong restrictions on $\lie{g}$ itself.
\begin{proposition}
	\label{prop:bch-desaster}
	Let $\lie{g}$ be a Banach-Lie algebra and $\algebra{U}(\lie{g})$ has a projective 
	grouplike topology. Then the Baker-Campbell-Hausdorff series converges globally 
	on $\lie{g}$.
\end{proposition}
\begin{proof}
	Use this projective grouplike topology on $\algebra{U}(\lie{g})$, $\xi, \eta \in 
	\lie{g}$ and $t,s \in \mathbb{K}$, then we have
	\begin{align*}
		\pi \left( \exp(t \xi) \cdot \exp(s \eta) \right)
		& =
		\pi
		\left(
			\lim_{N \rightarrow \infty}
			\left(
				\sum\limits_{n=0}^N
				\frac{t^n \xi^n}{n!}
			\right)
			\cdot
			\lim_{M \rightarrow \infty}
			\left(
				\sum\limits_{m=0}^M
				\frac{s^m \eta^m}{m!}
			\right)
		\right)
		\\
		& \ot{(a)}{=}
		\pi
		\left(
			\lim_{N \rightarrow \infty}
			\lim_{M \rightarrow \infty}
			\left(
				\sum\limits_{n=0}^N
				\frac{t^n \xi^n}{n!}
			\right)
			\cdot
			\left(
				\sum\limits_{m=0}^M
				\frac{s^m \eta^m}{m!}
			\right)
		\right)
		\\
		& \ot{(b)}{=}
		\lim_{N \rightarrow \infty}
		\lim_{M \rightarrow \infty}
		\pi
		\left(	
			\left(
				\sum\limits_{n=0}^N
				\frac{t^n \xi^n}{n!}
			\right)
			\cdot
			\left(
				\sum\limits_{m=0}^M
				\frac{s^m \eta^m}{m!}
			\right)
		\right)
		\\
		& \ot{(c)}{=}
		\lim_{N \rightarrow \infty}
		\lim_{M \rightarrow \infty}
		\sum\limits_{n=0}^{N}
		\sum\limits_{m=0}^{M}
		\bchparts{n}{m}{t \xi}{s \eta}
		\\
		& \ot{(d)}{=}
		\lim_{L \rightarrow \infty}
		\sum\limits_{\ell=0}^{L}
		\sum\limits_{a + b = \ell}
		\bchparts{a}{b}{t \xi}{s \eta}
		\\
		& =
		\lim_{L \rightarrow \infty}
		\sum\limits_{\ell=0}^{L}
		\bchpart{\ell}{t \xi}{s \eta},
	\end{align*}
	where we used the continuity of multiplication in $(a)$ and the one of $\pi$ in 
	$(b)$. In $(c)$ we just evaluated $\pi$ and used the fact that the 
	Baker-Campbell-Hausdorff series is a power series in $(d)$ and hence converges 
	absolutely. So it must converge globally on $\lie{g}$, when the latter is seen 
	as a subspace of $\widehat{\algebra{U}}(\lie{g})$. But since the topology as a 
	subspace is equivalent to the one of $\lie{g}$ itself, $\bch{\xi}{\eta}$ is 
	defined for all $\xi, \eta \in \lie{g}$.
\end{proof}
If we want a continuous projection onto $\lie{g}$, Proposition 
\ref{prop:bch-desaster} rules out the existence of locally multiplicatively convex or 
norm topologies on $\algebra{U}(\lie{g})$ by the result of Wojty{\`n}ski 
\cite{wojtynski:1998a}, at least for not quasi-nilpotent Banach-Lie algebras. We see 
that in order to have a projective grouplike topology, $\lie{g}$ must at least be 
quasi-nilpotent, but maybe this is not enough. To go on, we must clarify the possible 
notions of quasi-nilpotency in Banach-Lie algebras. 



\section{Quasi-Nilpotency in Banach-Lie Algebras}
\label{sec:QuasiNilpotency}

\subsection{Classes of Quasi-Nilpotent Banach-Lie Algebras}
First, we want to classify, how we can ``measure'' the nilpotency of a Banach-Lie 
algebra. In an associative Banach algebra $\algebra{A}$, the usual way is looking 
at the spectral radii of the elements $a \in \algebra{A}$ or at the joint spectral 
radii of bounded subsets of $\algebra{A}$. Unfortunately, for a Banach-Lie algebra, 
a good notion of the spectral radius of elements is not so obvious. In 
\cite{galindo.palacios:2012a}, Galindo and Palacios constructed a ``Lie-spectral 
radius'', but could not prove bigger theorems with this. We want to go a different 
way. Let $\lie{g}$ be a Banach-Lie algebra, then the bounded linear operators 
$\Bounded(\lie{g})$ form an associative Banach algebra with the spectral norm. The 
image $\ad \colon \lie{g} \longrightarrow \Bounded(\lie{g})$ generates a normed (but 
not necessarily complete) subalgebra of $\Bounded(\lie{g})$. We can now use the 
notions from the associative theory:
\begin{definition}
	\label{def:nilpotencies}
	Let $\lie{g}$ be a Banach-Lie algebra in which the Lie bracket fulfils the 
	estimate
	\begin{equation*}
		\norm{ [\xi, \eta] }
		\leq
		\norm{\xi}
		\norm{\eta}.
	\end{equation*}
	Denote by $\mathbb{B}_1(0)$ all elements $\xi \in \lie{g}$ with 
	$\norm{\xi} = 1$. We say that
	\begin{definitionlist}
		\item
		$\lie{g}$ is quasi-nilpotent (or topologically nil, or radical, or 
		of type $(i)$), if every $\xi \in \lie{g}$ is quasi-nilpotent, i.e.
		\begin{equation*}
			\lim_{n \rightarrow \infty}
			\norm{\ad_{\xi}^n}^{\frac{1}{n}}
			=
			0.
		\end{equation*}
		
		\item
		$\lie{g}$ is uniformly quasi-nilpotent (or uniformly topologically nil, or 
		of type $(ii)$), if
		\begin{equation*}
			\lim_{n \rightarrow \infty}
			\mathcal{N}_1(n)
			=
			0
		\end{equation*}
		for
		\begin{equation}
			\mathcal{N}_1(n)
			=
			\sup \left\{ 
			\left.
				\norm{ \ad_{\xi}^n}^{\frac{1}{n}} 
			\ \right| \
				\xi \in \mathbb{B}_1(0)
			\right\}.
		\end{equation}
		
		\item
		$\lie{g}$ is topologically nilpotent (or of type 
		$(iii)$), if for every sequence
		$(\xi_n)_{n \in \mathbb{N}} \subset \mathbb{B}_1(0)$ we have
		\begin{equation*}
			\lim_{n \rightarrow \infty}
			\norm{ 
				\ad_{\xi_1} \circ \ldots \circ \ad_{\xi_n}
			}^{\frac{1}{n}}
			=
			0.
		\end{equation*}
		
		\item
		$\lie{g}$ is uniformly topologically nilpotent (or of 
		type $(iv)$), if
		\begin{equation*}
			\lim_{n \rightarrow \infty}
			\mathcal{N}(n)
			=
			0
		\end{equation*}
		for
		\begin{equation}
			\mathcal{N}(n)
			=
			\sup \left\{ 
			\left.
				\norm{ 
					\ad_{\xi_1} \circ \ldots \circ \ad_{\xi_n}
				}^{\frac{1}{n}} 
			\ \right| \
				\xi_1, \ldots, \xi_n \in \mathbb{B}_1(0)
			\right\}.
		\end{equation}
	\end{definitionlist}
\end{definition}
This definition is the same as in \cite{mueller:1994a}, just using $\ad_\xi$ 
instead of $\xi$ itself. We will need reformulation of type $(ii.)$, which seems to 
be stronger on the first sight but which is in fact equivalent. For the moment, we 
define it as follows.
\begin{itemize}
	\item[$ii'.)$]
	Let $\lie{g}$ be a Banach-Lie algebra like in Definition 
	\ref{def:nilpotencies}, then we say that $\lie{g}$ is of type $(ii')$, if
	\begin{equation*}
		\lim_{n \longrightarrow \infty}
		\mathcal{S}(n)
		=
		0
	\end{equation*}
	for
	\begin{equation}
		\mathcal{S}(n)
		=
		\sup \Bigg\{ 
			\bigg\Vert
				\frac{1}{n!}
				\sum\limits_{\sigma \in S_n}
				\ad_{\xi_{\sigma(1)}} \circ \ldots \circ \ad_{\xi_{\sigma(n)}}
			\bigg\Vert^{\frac{1}{n}} 
		\ \Bigg| \ 
			\xi_1, \ldots, \xi_n \in \mathbb{B}_1(0)
		\Bigg\}.
	\end{equation}
\end{itemize}
It is immediate to see that the following implications hold:
\begin{equation*}
	(iv) \Longrightarrow 
	(ii') \Longrightarrow 
	(ii) \Longrightarrow 
	(i)
	\quad \text{ and } \quad
	(iv) \Longrightarrow
	(iii) \Longrightarrow 
	(i).
\end{equation*}
To show that $(ii) \Longrightarrow (ii')$, we need a little combinatorial lemma.
\begin{lemma}
	\label{lemma:symmetric-to-power}
	Let $R$ be a ring, $n\in \mathbb{N}$ and $a_1, 
	\ldots, a_n \in R$, then we have the identity
	\begin{equation*}
		\sum\limits_{\sigma \in S_n}
		a_{\sigma(1)} \cdots a_{\sigma(n)}
		=
		\sum\limits_{k = 1}^n
		(-1)^{n-k}
		\sum\limits_{1 \leq i_1 < \cdots < i_k \leq n}
		(a_{i_1} + \cdots + a_{i_k})^n.
	\end{equation*}
\end{lemma}
\begin{proof}
	The proof is elementary and can be done by a combinatorial argument, counting 
	the multiplicities of the possible products.
\end{proof}
\begin{proposition}
	\label{prop:type-2-is-2prime}
	Banach-Lie algebras, which are quasi-nilpotent of type $(ii)$, are of type 
	$(ii')$.
\end{proposition}
\begin{proof}
	Let $\lie{g}$ be a type $(ii)$-Banach-Lie algebra. For every $\xi \in 
	\mathbb{B}_1(0)$, we have
	\begin{equation*}
		\norm{(\ad_\xi)^n}^{\frac{1}{n}}
		\leq
		\mathcal{N}_1(n).
	\end{equation*}
	Note that 
	\begin{equation*}
		\tag{$*$}
		n \geq \sqrt[n]{n!} \geq \frac{n}{\E},
	\end{equation*}
	and hence for all $a_1, \ldots, a_k \in \mathbb{B}_1(0)$, we have
	\begin{equation*}
		\frac{1}{\E \sqrt[n]{n!}} (a_1 + \cdots + a_k)
		\in \mathbb{B}_1(0).
	\end{equation*}
	This gives
	\begin{align*}
		\left\Vert
			\frac{1}{n!}
			\sum\limits_{\sigma \in S_n}
			\ad_{\sigma(1)}
			\circ \cdots \circ
			\ad_{\sigma(n)}
		\right\Vert
		&=
		\frac{1}{n!}
		\left\Vert
			\sum\limits_{k = 1}^n
			(-1)^{n-k}
			\sum\limits_{1 \leq i_1 < \cdots < i_k \leq n}
			(\ad_{\xi_{i_1}} + \cdots + \ad_{\xi_{i_n}})^n
		\right\Vert
		\\
		& \leq
		\frac{1}{n!}
		\sum\limits_{k = 1}^n
		\sum\limits_{1 \leq i_1 < \cdots < i_k \leq n}
		\left\Vert
			(\ad_{\xi_{i_1} + \cdots + \xi_{i_n}})^n
		\right\Vert
		\\
		& \leq
		\frac{1}{n!}
		\sum\limits_{k = 1}^n
		\binom{n}{k}
		k^n (\mathcal{N}_1(n))^n
		\\
		& \leq
		\frac{n^n}{n!}
		\sum\limits_{k = 1}^n
		\binom{n}{k}
		(\mathcal{N}_1(n))^n
		\\
		& \leq
		(2 \E \mathcal{N}_1(n))^n.
	\end{align*}
	Hence we find
	\begin{equation*}
		\mathcal{S}(n)
		\leq
		\sqrt[n]{(2 \E \mathcal{N}_1(n))^n}
		=
		2 \E \mathcal{N}_1(n),
	\end{equation*}
	which converges to zero if $\mathcal{N}_1(n)$ itself does.
\end{proof}



\subsection{Topologically nilpotent Banach-Lie Algebras}
For associative algebras, Dixon and M\"uller proved that $(iii.) \Longrightarrow 
(vi.)$ \cite[Theorem 3]{dixon.mueller:1992a}. With a slight adaptation, their proof 
also works for Banach-Lie algebras.
\begin{proposition}
	\label{prop:top-nilpotency}
	Topologically nilpotent Banach-Lie algebras are uniformly topologically 
	nilpotent.
\end{proposition}
\begin{proof}
	We just want to sketch the proof and explain the changes.
	Let $\lie{g}$ be a Banach-Lie algebra, then one has to show that the limit 
	$\mathcal{N} = \lim_{n \rightarrow \infty} \mathcal{N}(n)$ exists and to 
	construct a sequence $(\xi) = (\xi_1, \xi_2, \ldots) \subset \mathbb{B}_1(0)$, 
	such that
	\begin{equation*}
		\limsup_{n \rightarrow \infty}
		\norm{\ad_{\xi_1} \circ \cdots \circ \ad_{\xi_n} }^{1/n}
		=
		\mathcal{N},
	\end{equation*}
	which implies the proposition, since the left hand side converges to zero by 
	the definition of topological nilpotency. Clearly,
	\begin{equation*}
		\limsup_{n \rightarrow \infty}
		\norm{\ad_{\xi_1} \circ \cdots \circ \ad_{\xi_n} }^{1/n}
		\leq
		\mathcal{N},
	\end{equation*}
	for every such sequence, so we must find $(\xi)$ with
	\begin{equation*}
		\limsup_{n \rightarrow \infty}
		\norm{\ad_{\xi_1} \circ \cdots \circ \ad_{\xi_n} }^{1/n}
		\geq
		\mathcal{N}.
	\end{equation*}
	holds. Since this is clear for $\mathcal{N} = 0$, we suppose $\mathcal{N} > 0$.
	Define complete metric space $X = \mathbb{B}_1(0)^{\mathbb{N}}$ with
	\begin{equation}
		\label{eq:metric-on-X}
		\halbnorm{d}((\xi), (\eta))
		=
		\sum\limits_{i = 1}^\infty
		2^{-i} \frac{\norm{\xi_i - \eta_i}}{1 + \norm{\xi_i - \eta_i}}.
	\end{equation}
	Now, the proof is completely analogous to the one for the associative case.
	The main change is in the definition on $X$: we need to use the unit ball in 
	$\lie{g}$ to construct it, and then to go on with $\ad_{\lie{g}}$ every time
	we need the associative algebra structure, which is possible since we have for 
	all $\xi \in \lie{g}$ the inequality $\norm{\xi} \geq \norm{\ad_{\xi}}$. 
	The main argument is the Baire category theorem: for $\delta > 0$, the sets
	\begin{equation*}
		X_{k, \delta}
		=
		\left\{
			(\xi) \in X
		\ \left| \
			\norm{ \ad_{\xi_1} \circ \cdots \circ \ad_{\xi_n} }^{1/n}
			>
			(1 - \delta) \mathcal{N}
			, \
			\text{ for some }
			n \geq k	
		\right.
		\right\}
	\end{equation*}
	are open in $X$. One shows, that they are dense in $X$ and gets
	\begin{equation*}
		\bigcap\limits_{k = 1}^\infty
		X_{k, 1/k}
		\neq
		\emptyset
	\end{equation*}
	by the Baire category theorem for metric spaces. To show this density,
	one shows that the $X_{k, \delta}$ are dense in the set
	\begin{equation*}
		W 
		=
		\bigcup\limits_{r < 1}
		\mathbb{B}_r(0)^{\mathbb{N}}
	\end{equation*}
	by an explicit construction. Since $W$ is obviously dense in $X$, this also 
	holds for the $X_{k, \delta}$.
\end{proof}
This settles the possible types of nilpotency of Banach-Lie algebras to just three 
with the following implications:
\begin{equation}
	\label{nilpotencytypes}
	(iii) \Longrightarrow (ii) \Longrightarrow (i).
\end{equation}



\subsection{Examples}

Since we know that different types of quasi-nilpotency exist, we want to give an 
example for each one of it. The examples are commutator Lie algebras of associative 
algebras. These in turn are is inspired by \cite[Theorem 1]{dixon.mueller:1992a}


\subsubsection{Example 1: A uniformly quasi-nilpotent Banach-Lie Algebra}

Take the free associative $\mathbb{C}$-algebra in the generators $\{x_n\}_{n \in 
\mathbb{N}}$ together with the relation $x_i x_j = \delta_{i+1}^j x_i x_j$. Let 
$p \in [1, \infty)$ and use the $\ell^p$-norm on this space to complete it. An 
arbitrary element in this algebra, which we call $\algebra{A}_p$, will have the form
\begin{equation*}
	a =
	\sum\limits_{i < j}
	a_{i, j} x_i x_{i + 1} \cdots x_{j - 1}
	, \
	a_{i,j} \in \mathbb{C}
	\quad \text{ with } \quad
	\norm{a}_p
	=
	\left(
		\sum\limits_{i < j}
		|a_{i, j}|^p
	\right)^{1/p}.
\end{equation*}
\begin{lemma}
	\label{lemma:Ex1}
	Let $p \in [1, \infty)$ and $a \in \algebra{A}_p$. 
	Then we have the following estimates.
	\begin{lemmalist}
		\item
		Powers of  fulfils the inequality
		\begin{equation}
			\label{eq:lemma:Ex1Powers}
			\norm{a^n}
			\leq
			\frac{1}{\sqrt[p]{n!}}
			\norm{a}_p^n.
		\end{equation}
		
		\item
		$\algebra{A}_p$ is not topologically nilpotent but topologically uniformly 
		nil as associative algebra and the characteristic series fulfils
		\begin{equation}
			\label{eq:lemma:Ex1Sequence}
			\frac{1}{\sqrt[p]{n}}
			\leq 
			\mathcal{N}_1(n) 
			\leq 
			\sqrt[p]{\frac{\E}{n}}.
		\end{equation}	
	\end{lemmalist}
\end{lemma}
\begin{proof}
	We just need to compute
	\begin{equation*}
		n! \norm{a^n}_p^p
		=
		n! \sum\limits_{i_0 < i_1 < \cdots < i_n}
		\left|
			a_{i_0, i_1}
		\right|^p
			\cdots
		\left|
			a_{i_{n-1}, i_n}
		\right|^p
		\leq
		\Bigg(
			\sum\limits_{i < j}
			\left|
				a_{i,j}
			\right|^p
		\Bigg)
		\cdots
		\Bigg(
			\sum\limits_{i < j}
			\left|
				a_{i,j}
			\right|^p
		\Bigg)
		=
		\left(
			\norm{a}_p^n
		\right)^p,
	\end{equation*}
	which leads to 
	\begin{equation*}
		\sqrt[p]{n!} \norm{a^n}_p
		\leq
		\norm{a}_p^n
	\end{equation*}
	and hence to the first estimate.
	
	For the second part, we get the estimate on the right hand side by taking 
	$\norm{a}_p = 1$ and using $\frac{1}{n!} \leq \frac{\E^n}{n^n}$ and taking the 
	$n$-th root. For the estimate on left hand side, set
	\begin{equation*}
		a_n 
		=
		\frac{1}{\sqrt[p]{n}}
		\sum\limits_{i=1}^n
		x_i.
	\end{equation*}
	Note that $\norm{a_n}_p = 1$ for all $n$ and we get
	\begin{equation*}
		\norm{a_n^n}_p
		=
		\left\Vert
			\left(
				\frac{1}{\sqrt[p]{n}}
				\sum\limits_{i=1}^n
				x_i
			\right)^n
		\right\Vert_p
		=
		\sqrt[p]{\frac{1}{n^n}}
		\norm{x_1 \cdots x_n}_p
		=
		\sqrt[p]{\frac{1}{n^n}},
	\end{equation*}
	so $\mathcal{N}_1(n) \geq \sqrt[p]{\frac{1}{n}}$.
	Hence $\algebra{A}_p$ is at least uniformly topologically nil. But since
	\begin{equation*}
		\norm{x_1 \cdots x_n}_p
		=
		\norm{x_1}_p \cdots \norm{x_n}_p
		=
		1
	\end{equation*}
	and $\norm{x_i} = 1$, we see that $\algebra{A}_p$ is 
	\emph{not} topologically nilpotent, hence the statement is true.
\end{proof}
Now we pass to the Lie algebra $\lie{g}_p$, which comes from this 
associative algebra. We see that this Lie algebra behaves very similarly to 
$\algebra{A}_p$ concerning its nilpotency.
\begin{lemma}
	The Banach-Lie algebra $\lie{g}_p$ has the following properties.
	\begin{lemmalist}
		\item
		The Lie bracket fulfils the inequality
		\begin{equation}
			\label{eq:lemma:Ex1LiePowers}
			\left\Vert
				\ad_a^n (b)
			\right\Vert_p
			\leq
			\frac{2^n \sqrt[p]{2^n}}{\sqrt[p]{n!}}
			\norm{a}_p^n \norm{b}_p.
		\end{equation}
		
		\item
		$\algebra{A}_p$ is not topologically nilpotent but topologically uniformly 
		nil as Lie algebra and the characteristic series fulfils
		\begin{equation}
			\label{eq:lemma:Ex1LieSequence}
			\frac{1}{\sqrt[p]{n}} 
			\leq 
			\mathcal{N}_1(n)
			\leq 
			\frac{2 \sqrt[p]{2 \E}}{\sqrt[p]{n}}.
		\end{equation}	
	\end{lemmalist}
\end{lemma}
\begin{proof}
	Again, we do an explicit computation using the results of the foregoing lemma:
	\begin{align*}
		\left\Vert
			\ad_a^n(b)
		\right\Vert_p
		&=
		\left\Vert
			\sum\limits_{k = 0}^n
			(-1)^k
			\binom{n}{k}
			a^k
			\cdot b \cdot
			a^{n - k}
		\right\Vert_p
		\\
		& \leq
		\sum\limits_{k = 0}^n
		\binom{n}{k}
		\norm{a^k}_p
		\cdot \norm{b}_p \cdot
		\norm{a^{n - k}}_p
		\\
		& \leq
		\sum\limits_{k = 0}^n
		\binom{n}{k}
		\frac{1}{\sqrt[p]{k!}}
		\frac{1}{\sqrt[p]{(n - k)!}}
		\norm{a}_p^n
		\cdot \norm{b}_p
		\\
		& \leq
		\sum\limits_{k = 0}^n
		\binom{n}{k}
		\sqrt[p]{\frac{2^n}{n!}}		
		\norm{a}_p^n
		\cdot \norm{b}_p
		\\
		& =
		\frac{2^n \sqrt[p]{2^n}}{\sqrt[p]{n!}}		
		\norm{a}_p^n
		\cdot \norm{b}_p.
	\end{align*}
	The second part is again just taking the $n$-th root and setting 
	$\norm{a}_p = 1$ for the right hand side. For the left hand side, choose
	for every $n$: $a = \frac{1}{\sqrt[p]{n}} \sum_{i=1}^n x_i$ and $b = x_{n+1}$.
\end{proof}



\subsubsection{Example 2: A quasi-nilpotent Banach-Lie Algebra}

We combine all algebras $\algebra{A}_p$ to one, by taking their direct sum and 
complete it using an $\ell^1$-norm: take the free $\mathbb{C}$-algebra in the 
generators $\{x_{m;n}\}_{m,n \in \mathbb{N}}$, now with the relation $x_{m;n} \cdot 
x_{m';n'} = x_{m;n} \cdot x_{m';n'} \delta_m^{m'} \delta_{n+1}^{n'}$. Now an 
arbitrary element of this algebra, which we want to call $\algebra{A}_\infty$, will 
have the form
\begin{equation*}
	a =
	\sum\limits_{m \in \mathbb{N}}
	\sum\limits_{i < j}
	a_{m; i, j} x_{m; i} x_{m; i + 1} \cdots x_{m; j - 1}
	, \
	a_{m; i,j} \in \mathbb{C}
	\quad \text{ with } \quad
	\norm{a}
	=
	\sum\limits_{m \in \mathbb{N}}
	\left(
		\sum\limits_{i < j}
		|a_{m; i, j}|^m
	\right)^{1/m}.
\end{equation*}
One can see from what preceded, that this algebra is still type topologically 
nilpotent. It is also not of type $(ii.)$. We can find a sequence $(a_n)_{n \in 
\mathbb{N}}$ with $\norm{ a_n } = 1$ such that the sequence $\norm{a_n^n}^{1/n}$ 
does not converge to zero. This is enough, since certainly $\mathcal{N}_1(n) \geq 
\norm{a_n^n}^{1/n}$. Set
\begin{equation*}
	a_n
	=
	\frac{1}{\sqrt[n]{n}}
	(x_{n;1} + \cdots + x_{n;n}),
\end{equation*}
then we get
\begin{equation*}
	\norm{a_n^n}
	=
	\left\Vert
		\frac{1}{\sqrt[n]{n}^n}
		x_{n; 1} \cdots x_{n; n}
	\right\Vert
	=
	\frac{1}{n}
	\norm{ x_{n; 1} \cdots x_{n; n} }
	=
	\frac{1}{n}
\end{equation*}
and hence
\begin{equation*}
	\sqrt[n]{ \norm{a_n^n} }
	=
	\sqrt[n]{\frac{1}{n}}
	\longrightarrow
	1.
\end{equation*}
We just need to show, that this algebra is really quasi-nilpotent. So let 
$a \in \algebra{A}_\infty$ and choose some $\varepsilon > 0$. Then we can write 
\begin{equation*}
	a 
	=
	\sum\limits_{m=1}^\infty
	a_m
	=
	\sum\limits_{m=1}^\infty
	\sum\limits_{i < j}
	a_{m; i, j} x_{m; i} \cdots x_{m; j - 1}	
\end{equation*}
and there must be an $M \in \mathbb{N}$, such that
\begin{equation*}
	\bigg\Vert
		\sum\limits_{m = M+1}^\infty
		a_m
	\bigg\Vert
	=
	\sum\limits_{m = M + 1}^\infty
	\norm{a_m}_m
	<
	\frac{\varepsilon}{2}.
\end{equation*}
Hence we find for $n \in \mathbb{N}$
\begin{equation*}
	\sqrt[n]{\norm{a^n}}
	=
	\sqrt[n]{
		\sum\limits_{m=1}^\infty
		\norm{a_m^n}_m
	}
	\leq
	\sum\limits_{m = 1}^M
	\sqrt[n]{
		\left\Vert a_m^n \right\Vert_m
	}
	+
	\sqrt[n]{ \left( \frac{\varepsilon}{2} \right)^n }
	\leq
	\frac{M \sqrt[M]{\E}}
	{\sqrt[M]{n}}
	+ \frac{\varepsilon}{2}.
\end{equation*}
It is clear, that for $n$ big enough, this expression is smaller than $\varepsilon$, 
since $M$ is a fixed integer. So $\algebra{A}_\infty$ is really quasi-nilpotent.


\subsubsection{Example 3: A topologically nilpotent Banach-Lie Algebra}
We can also use $\algebra{A}_1$ from Example 1, modify the norm and get first a 
topologically nilpotent associate Banach algebra and then a Banach-Lie algebra with 
the same property. Just take again the free $\\mathbb{K}$-algebra generated by the 
$x_i$, $i \in \mathbb{N}$ and use the norm
\begin{equation}
	a =
	\sum\limits_{i < j}
	a_{i, j} x_i x_{i + 1} \cdots x_{j - 1}
	, \
	a_{i,j} \in \mathbb{C}
	\quad \text{ with } \quad
	\norm{a}
	=
	\sum\limits_{n = 1}^\infty
	\frac{1}{(j - i)!}|a_{i,j}|
\end{equation}
to complete it. We will call this algebra $\algebra{A}_0$.
\begin{lemma}
	$\algebra{A}_0$ is an associative, topologically nilpotent 
	Banach algebra with $\mathcal{N}(n) \leq \frac{\E}{n}$.
\end{lemma}
\begin{proof}
	Let $a^{(1)}, \ldots, a^{(n)} \in \algebra{A}_0$ with $a^{(\ell)} = \sum_{i < j}
	a_{i,j}^{(\ell)} x_i \cdots x_{j-1}$ for $\ell = 1, \ldots, n$. Then we have
	\begin{align*}
		\left\Vert
			a^{(1)} \cdots a^{(n)}
		\right\Vert
		&=
		\sum\limits_{i_0 < \cdots < i_n}
		\frac{1}{(i_n - i_0)!}
		|a_{i_0, i_1}^{(1)}| \cdots |a_{i_{n-1}, i_n}^{(n)}|
		\\
		&=
		\sum\limits_{i_0 < \cdots < i_n}
		\frac{(i_n - i_{n-1})! \cdots (i_1 - i_0)!}{(i_n - i_0)!}
		\frac{|a_{i_0, i_1}^{(1)}|}{(i_1 - i_0)!} 
		\cdots 
		\frac{|a_{i_{n-1}, i_n}^{(n)}|}{(i_n - i_{n-1})!}
		\\
		& \stackrel{(*)}{\leq}
		\sum\limits_{i_0 < \cdots < i_n}
		\frac{1}{n!}
		\frac{|a_{i_0, i_1}^{(1)}|}{(i_1 - i_0)!} 
		\cdots 
		\frac{|a_{i_{n-1}, i_n}^{(n)}|}{(i_n - i_{n-1})!}
		\\
		& \leq
		\frac{1}{n!}
		\left(
			\sum\limits_{i < j}
			\frac{|a_{i, j}^{(1)}|}{(i - j)!} 
		\right)
		\cdots 
		\left(
			\sum\limits_{i < j}
			\frac{|a_{i, j}^{(n)}|}{(i - j)!} 
		\right)
		\\
		& =
		\frac{1}{n!}
		\left\Vert a^{(1)} \right\Vert
		\cdots
		\left\Vert a^{(n)} \right\Vert.
	\end{align*}
	In ($*$), we used that $i_1! \cdots i_n! n! \leq (i_1 + \cdots + i_n)!$,
	if $i_1, \ldots, i_n \geq 1$. With this result and $n^n \leq \E^n n!$,
	we find $\mathcal{N}(n) \leq \frac{\E}{n}$. Analogously to Example 1, it is 
	clear that the Banach-Lie algebra $\lie{g}_0$, which comes from $\algebra{A}_0$, 
	will also be topologically nilpotent, but certainly not nilpotent.
\end{proof}


\section{Uniformly quasi-nilpotent Banach-Lie Algebras}

We have everything at the hand to state our main theorem now:
\begin{theorem}
	\label{thm:main}
	A Banach-Lie algebra $\lie{g}$ has a grouplike topology on 
	$\algebra{U}(\lie{g})$ if and only if it is uniformly quasi-nilpotent.
\end{theorem}
Since it is rather long and technical, we will split up the proof into two parts.


\subsection{Uniform Quasi-nilpotency is necessary for projective grouplike 
Topologies}
\begin{proposition}
	\label{prop:mainthm_part_1}
	Let $\lie{g}$ be a Banach-Lie algebra, which is not uniformly 
	quasi-nilpotent. Then it $\algebra{U}(\lie{g})$ can not have a projective 
	grouplike topology.
\end{proposition}
\begin{proof}
	The idea is to construct two sequences which converge to zero, but whose 
	product does not converge: Being not of type $(ii)$ means, that there is an
	$\varepsilon > 0$, such that $\lim_{n \rightarrow \infty} \mathcal{N}_1(n) > 
	\varepsilon$. This means that for every $n \in \mathbb{N}$, there is a sequence 
	$\left(\xi_{n, k} \right)_{k \in \mathbb{N}} \subset \mathbb{B}_1(0)$, 
	such that
	\begin{equation*}
		\lim_{k \longrightarrow \infty}
		\left\Vert
			(\ad_{\xi_{n, k}})^n
		\right\Vert^{\frac{1}{n}}
		=
		\mathcal{N}_1(n).
	\end{equation*}
	So for every $n, k \in \mathbb{N}$ we have
	\begin{equation*}
		\left\Vert
			(\ad_{\xi_{n, k}})^n
		\right\Vert^{\frac{1}{n}}
		> \varepsilon.
	\end{equation*}
	This means that we find for every element $\xi_{n,n}$ a sequence 
	$(\alpha_{n, \ell})_{\ell \in \mathbb{N}}$ with $\norm{\alpha_{n, \ell}} = 1$ 
	for all $n, \ell \in \mathbb{N}$ and
	\begin{equation*}
		\lim_{\ell \rightarrow \infty}
		(\ad_{\xi_{n,n}})^n(\alpha_{n,\ell})
		> 
		\varepsilon^n.
	\end{equation*}
	We define a new sequence $(\eta_n)_{n \in \mathbb{N}}$ with 
	$\eta_n = \alpha_{n, \ell}$ where $\ell$ is chosen big enough to 
	fulfil 
	\begin{equation*}
		(\ad_{\xi_{n,n}})^n(\alpha_{n,\ell}) 
		>
		\varepsilon^n.
	\end{equation*}
	Now assume we have a projective grouplike topology on $\algebra{U}(\lie{g})$ 
	and we denote by $\algebra{P}$ the set of all continuous seminorms on it. 
	Then we have for a fixed $t \in \mathbb{R}$ and for every $\xi \in \lie{g}$ 
	and every $p \in \algebra{P}$
	\begin{equation*}
		p \left( \exp(t \xi) \right)
		=
		p \left(
			\sum\limits_{n=0}^\infty
			\frac{|t|^n \xi^n}{n!}
		\right)
		\leq
		\sum\limits_{n=0}^\infty
		\frac{p\left( \xi^n \right)}{n!}
		<
		\infty,
	\end{equation*}
	since the power series converges absolutely. Since the sequence 
	$(t \xi_{n,n})_{n \in \mathbb{N}}$ is bounded in $\lie{g}$, it is also bounded 
	in $\algebra{U}(\lie{g})$. Hence for every $t > 0$, the sequence
	\begin{equation*}
		\left(
			\frac{t^n}{n!}
			p\left(
				\xi_{n,n}^n
			\right)
		\right)_{n \in \mathbb{N}}
	\end{equation*}
	converges to zero (in $\lie{g}$ and in $\algebra{U}(\lie{g})$). The sequence 
	$(\frac{1}{n} \eta_n)_{n \in \mathbb{n}}$ also converges to zero, again in 
	$\lie{g}$ and $\algebra{U}(\lie{g})$.
	Now we want to show that their product does \emph{not} converge to zero
	to get a contradiction to the continuity of the multiplication. Therefore
	we need again the projection map $\pi_1 \colon \algebra{U}(\lie{g}) 
	\longrightarrow \lie{g}$ which is linear and continuous and should hence not 
	spoil the convergence. For every fixed $t > 0$, we have
	\begin{align*}
		p \left(
			\pi_1\left(
				\frac{t^n \xi_{n,n}^n}{n!}
				\cdot
				\frac{\eta_n}{n}
			\right)
		\right)
		&=
		p\left(
			\pi_1 \left(
				\frac{1}{n \cdot n!}
				\sum\limits_{k=0}^n
				\binom{n}{k}
				B_k^* t^n
				\xi_{n,n}^{\vee(k-n)}
				\vee
				(\ad_{\xi_{n,n}})^k(\eta_n)
			\right)
		\right)
		\\
		& =
		p\left(
			\frac{B_n^* t^n}{n \cdot n!}
			(\ad_{\xi_{n,n}})^n(\eta_n)
		\right)
		\\
		& >
		\frac{|B_n|^* t^n \varepsilon^n}{n \cdot n!}.
	\end{align*}
	But we know that this series does not converge to zero: It has a divergent 
	subsequence, if we choose for example $t = \frac{3 \pi}{\varepsilon}$, which is 
	allowed. Hence we have a contradiction.
\end{proof}


\subsection{Uniform Quasi-nilpotency is sufficient for projective grouplike 
Topologies}

The reverse direction of the proof is constructive and uses the 
$\Tensor_R$-topology. If $\mathcal{N}_1(n)$ decreases fast enough, one can
give a proof using the $\Tensor_{1^-}$-topology, which will be done in Proposition
\ref{prop:mainthm_rev_1}. If the characteristic series does not decrease fast 
enough, one needs to modify this a little, which we do in Proposition
\ref{prop:mainthm_rev_2}.

\begin{proposition}
	\label{prop:mainthm_rev_1}
	Let $\lie{g}$ be a uniformly quasi-nilpotent Banach-Lie algebra and assume that 
	there is a $p \geq 1$, such that $\lim_{n \rightarrow \infty} 
	\frac{\mathcal{N}_1(n)}{\sqrt[p]{n!}} = 0$. Then the $\Tensor_{1^-}$-topology is 
	projective grouplike.
\end{proposition}
\begin{proof}
	The continuity of the projection $\pi$ is clear from the construction and the 
	fact that it is grouplike is can be seen from Equation 
	\eqref{eq:TR-completion-explicit}. the only problem is the question of 
	continuity of $\star_z$. We look at the situation in the symmetric tensor 
	algebra and see that the product is continuous in $\Sym_{1^-}^{\bullet}
	(\lie{g})$ by the following argumentation. Denote the norm on $\lie{g}$ 
	for simplicity by $\halbnorm{p}$ again. Since $\lim_{n \rightarrow \infty} 
	\frac{\mathcal{N}_1(n)}{\sqrt[p]{n}} = 0$, we know there is a $p \geq 1$ and a 
	$c > 0$, such that $\mathcal{S}(n) \leq \frac{c}{\sqrt[p]{n}}$. Take $\xi_1, 
	\ldots, \xi_k, \eta \in \lie{g}$, then we find for some $\varepsilon > 0$ with 
	$R > 1 - \frac{1}{p} + \varepsilon$
	\begin{align*}
		\halbnorm{p}_R \big(
		&
			\xi_1 \tensor \cdots \tensor \xi_k
			\star
			\eta
		\big)
		\\
		&=
		\sum\limits_{n = 0}^k
		\frac{|B_n^*|}{k!} \binom{k}{n}
		(k + 1 - n)!^R
		\halbnorm{p}^{k+1-n}
		\left(
			\sum	\limits_{\sigma \in S_k}
			\xi_{\sigma(1)} \cdots \xi_{\sigma(k-n)}
			\cdot
			\left( 
				\ad_{\xi_{\sigma(k-n+1)}} 
				\circ \cdots \circ
				\ad_{\xi_{\sigma(k)}}
			\right)
			(\eta)
		\right)
		\\
		& \leq
		(k + 1)^R
		\sum\limits_{n = 0}^k
		\frac{|B_n^*|}{n!}
		\frac{1}{(k-n)!^{1 - R}}
		\halbnorm{p}^{k+1-n}
		\Bigg(
			\sum\limits_{
				I = \{i_1, \ldots, i_n\} \subset \{1, \ldots, k\}
			}
			(n-k)!
			\xi_1 \cdots \widehat{\xi_I} \cdots \xi_k
		\\
		& \quad \cdot
			\sum	\limits_{\sigma \in S_n}
			\left( 
				\ad_{\xi_{\sigma(1)}} 
				\circ \cdots \circ
				\ad_{\xi_{\sigma(n)}}
			\right)
			(\eta)
		\Bigg)
		\\
		& \leq
		(k + 1)^R
		\sum\limits_{n = 0}^k
		\frac{|B_n^*|}{n!}
		\frac{(k-n)!}{(k-n)!^{1 - R}}
		\sum\limits_{
			I = \{i_1, \ldots, i_n\} \subset \{1, \ldots, k\}
		}
		\halbnorm{p}(\xi_1) 
		\cdots \widehat{\halbnorm{p}(\xi_I)} \cdots 
		\halbnorm{p}(\xi_k)
		\\
		& \quad \cdot
		\halbnorm{p} \left(
			\sum	\limits_{\sigma \in S_n}
			\left( 
				\ad_{\xi_{\sigma(1)}} 
				\circ \cdots \circ
				\ad_{\xi_{\sigma(n)}}
			\right)
			(\eta)
		\right)
		\\
		& \leq
		(k + 1)^R
		\sum\limits_{n = 0}^k
		\frac{|B_n^*|}{n!}
		\frac{(k-n)!}{(k-n)!^{1 - R}}
		\binom{k}{n}
		\frac{c^n n!}{n!^{\frac{1}{p}}}
		\halbnorm{p}(\xi_1) \cdots \halbnorm{p}(\xi_k)
		\halbnorm{p}(\eta)
		\\
		& \leq
		(k + 1)^R
		\sum\limits_{n = 0}^k
		\frac{c^n |B_n^*|}{n!^{1 + \varepsilon}}
		\binom{k}{n}^{1-R}
		k!^R
		\halbnorm{p}(\xi_1) \cdots \halbnorm{p}(\xi_k) 
		\halbnorm{p}(\eta)
		\\
		& \leq
		(k + 1)^R
		\underbrace{
			2^{k (1-R)}
		}_{
			=\kappa_1 k!^{\frac{1-R}{2}}
		}
		\underbrace{
			\sum\limits_{n = 0}^k
			\frac{c^n |B_n^*|}{n!^{1 + \varepsilon}}
		}_{
			= \kappa_2
		}
		k!^R
		\halbnorm{p}^k(\xi_1 \tensor \cdots \tensor \xi_k) 
		\halbnorm{p}(\eta)
		\\
		& \leq
		(k-1)^R
		\kappa
		\halbnorm{p}_{R + \frac{1-R}{2}}
		(\xi_1 \tensor \cdots \tensor \xi_k)
		\halbnorm{p}(\eta).
	\end{align*}
	In the last step, we took $\kappa = \kappa_1 \kappa_2$. We can extend this 
	estimate to any arbitrary tensor of degree at most $k$ on the left hand side.
	Now we do the next step and take $\xi_1, \ldots, \xi_k, \eta_1, \ldots, 
	\eta_\ell \in \lie{g}$. We can iterate this estimate to get
	\begin{align*}
		\halbnorm{p}_R \big(
			\xi_1 \tensor \cdots \tensor \xi_k
		&
			\star
			\eta_1 \tensor \cdots \tensor \eta_\ell
		\big)
		\\
		& \leq
		(k + \ell)^R (k + \ell - 1)^{1 - \frac{1 - R}{2}}
		\cdots (k+1)^{{1 - \frac{1 - R}{2^{\ell - 1}}}}
		\kappa^\ell
		\halbnorm{p}_{1 - \frac{1-R}{2^\ell}}
		\left( \xi_1 \tensor \cdots \tensor \xi_k \right)
		\halbnorm{p}(\eta_1) \cdots p(\eta_\ell)
		\\
		& \leq
		\left( \frac{(k+\ell)!}{\ell!} \right)^{\frac{1 + R}{2}}
		\kappa^\ell
		\halbnorm{p}_1 \left( \xi_1 \tensor \cdots \tensor \xi_k \right)
		\halbnorm{p}^\ell \left( \eta_1 \tensor \cdots \tensor \eta_\ell \right)
		\\
		& \leq
		2^{k + \ell}
		\ell!^{\frac{1 + R}{2}}
		\kappa^\ell
		\halbnorm{p}_1 \left( \xi_1 \tensor \cdots \tensor \xi_k \right)
		\halbnorm{p}^\ell \left( \eta_1 \tensor \cdots \tensor \eta_\ell \right)
		\\
		&=
		(2 \halbnorm{p})_1 \left( \xi_1 \tensor \cdots \tensor \xi_k \right)
		(2 \kappa \halbnorm{p})_{\frac{1 + R}{2}} 
		\left( \eta_1 \tensor \cdots \tensor \eta_\ell \right).
	\end{align*}
	Finally we can assume without loss of generality that $\ell \geq k$, since
	we could do all estimates also for the right hand side in the same way.
	By setting $R' = \frac{3}{4}(1-R)$, we find
	\begin{equation*}
		p_R \big(
			\xi_1 \tensor \cdots \tensor \xi_k
			\star
			\eta_1 \tensor \cdots \tensor \eta_\ell
		\big)
		\leq
		(2p)_{R'}
		\left( \xi_1 \tensor \cdots \tensor \xi_k \right)
		(2 \kappa p)_{R'}
		\left( \eta_1 \tensor \cdots \tensor \eta_\ell \right)
	\end{equation*}
	and the proposition is proven.
\end{proof}

Now we need to cover the last possible case: A uniformly quasi-nilpotent Banach-Lie 
algebra, which has a characteristic sequence, which decreases slower than any power 
of $n!$. An example for something like this would be a Banach-Lie algebra fulfilling 
$\mathcal{N}_1(n) = (\log(n) + 1)^{-n}$ for $n \in \mathbb{N}$, for example. To 
control this case, we need a to adapt the $\Tensor_{1^-}$-topology. If we have a 
monotonously increasing sequence $(\alpha) = (\alpha_n)_{n \in \mathbb{B}}$, such 
that for every $c > 0$ we have
\begin{equation}
	\lim\limits_{n \rightarrow \infty}
	\frac{\alpha_n}{c^n}
	=
	\infty
	\quad \text{ and } \quad
	\lim\limits_{n \rightarrow \infty}
	\frac{\alpha_n}{n!}
	=
	0,
\end{equation}
then we can use it as a counterweight to the factorials and define a seminorm
\begin{equation}
	\halbnorm{p}_\alpha
	=
	\sum\limits_{n=0}^\infty
	\frac{n!}{\alpha_n}
	\halbnorm{p}^n
\end{equation}
on $\Tensor^{\bullet}(\lie{g})$ from a seminorm $\halbnorm{p}$ on $\lie{g}$. Since 
we do not want to introduce a new notation, we just call the norm of the Banach-Lie 
algebra $\halbnorm{p}$. In this way, we get a Fr\'echet-topology on 
$\Tensor^{\bullet}(\lie{g})$ using all powers $(\alpha_n^{1/r})_{n \in \mathbb{N}}$, 
$r \in \mathbb{N}$ of the sequence $(\alpha)$. Such a sequence is always there for 
every uniformly quasi-nilpotent Banach-Lie algebra: it is given by 
$\mathcal{N}_1(n)$.
\begin{proposition}
	\label{prop:mainthm_rev_2}
	Let $\lie{g}$ be a uniformly quasi-nilpotent Banach-Lie algebra and assume that 
	\begin{equation*}
		\lim_{n \rightarrow \infty} 
		\left( \frac{1}{n}\mathcal{N}_1(n) \right)
		= 
		0.
	\end{equation*}
	Then there is a projective grouplike topology on $\algebra{U}(\lie{g})$.
\end{proposition}
\begin{proof}
	Define a sequence $(\widetilde{\omega})$ with
	\begin{equation*}
		\widetilde{\omega}_n
		=
		\log \left(
			\mathcal{N}_1(n)^{-1}
		\right).
	\end{equation*}
	Since $\mathcal{N}_1(n)$ is monotonously decreasing as $\norm{[\xi, \eta]} \leq 
	\norm{\xi} \norm{\eta}$, $(\widetilde{\omega})$ will be monotonously increasing.
	We can construct a new series $(\omega)$, which fulfils 
	$\omega_n \leq \widetilde{\omega}_n$ for all $n \in \mathbb{N}$ and which is 
	``convex'' meaning that for all $n, m \in \mathbb{N}$, 
	$n \leq m$ and $t \in \mathbb{N}$ with $n \leq t \leq m$, the estimate
	\begin{equation*}
		\omega_t 
		\leq 
		\omega_n \left( 1 - \frac{t - n}{m - n} \right) + 
		\omega_m \frac{t - n}{m - n}
	\end{equation*}
	holds. From this, we can define a convex, smooth and monotonously increasing 
	function
	\begin{equation*}
		f \colon
		\mathbb{R}_0^+
		\longrightarrow
		\mathbb{R}^+
		, \quad
		\forall_{n \in \mathbb{N}}
		\colon
		f(n) 
		=
		\omega_n,
	\end{equation*}
	which also fulfils
	\begin{equation*}
		\lim_{x \rightarrow \infty}
		\frac{f(x)}{x \log(x)}
		=
		0
		\quad \text{ and } \quad
		\lim_{x \rightarrow \infty}
		\frac{f(x)}{x}
		=
		\infty
	\end{equation*}
	since $\mathcal{N}_1(n)$ decreases more slowly than any power of the factorials, 
	but $\lie{g}$ is still uniformly quasi-nilpotent. We want an analogue of the 
	estimate $\binom{n}{m} \leq 2^n$ for the series $\omega$, which is done by the 
	following lemma.
	\begin{lemma}
		Let $(\alpha_n)_{n \in \mathbb{N}}$ be a monotonously increasing, convex 
		sequence, such that
		\begin{equation*}
			\lim_{n \rightarrow \infty}
			\frac{\alpha_n}{n \log(n)}
			=
			0
			\quad \text{ and } \quad
			\lim_{n \rightarrow \infty}
			\frac{\alpha_n}{n}
			=
			\infty.
		\end{equation*}
		Then there is a constant $c \geq 0$, such that for every $n,m \in 
		\mathbb{N}$ with $m \leq n$ the following inequality holds:
		\begin{equation}
		\label{Lemma:LogBinomialEstimate}
			\alpha_n - \alpha_m - \alpha_{n-m}
			\leq
			c n.
		\end{equation}
	\end{lemma}
	\begin{subproof}
		This is a logarithmic version of the estimate for the binomial coefficient.
		First note that we can assign to the sequence $\alpha$ a smooth, 
		monotonously increasing and convex function $f$ as we did before.
		Note that $\lim_{x \rightarrow \infty} f'(x) = \infty$, since $f$ is convex 
		and $\lim_{x \rightarrow \infty} \frac{f(x)}{x} = \infty$. Now we can use
		the Theorem of de l'H\^{o}spital to get
		\begin{equation*}
			0 
			=
			\lim_{x \rightarrow \infty}
			\frac{f(x)}{x \log(x)}
			=
			\lim_{x \rightarrow \infty}
			\frac{f'(x)}{\log(x) + 1}
			=
			\lim_{x \rightarrow \infty}
			x f''(x)
		\end{equation*}
		and hence there is a $c \geq 0$ with $f''(x) \leq \frac{c}{x}$ by the 
		continuity of $f''$. Now assume $z = x + y$ for positive real numbers 
		$x,y,z$. By Taylor's Theorem, we find
		\begin{equation}
			\label{eq:TaylorEstimate}
			f(z)
			=
			f(x) + y f'(x)
			+ \int\limits_{x}^{z} (z - t) f''(t) dt.
		\end{equation}
		For reasons of convexity, the function
		\begin{equation*}
			g(z,x)
			=
			g(z) - g(x) - g(z-x)
		\end{equation*}
		is maximal for fixed $z \in \mathbb{R}^+$ if $2x = z$. So we have proven
		Equation \eqref{Lemma:LogBinomialEstimate}, if we can show 
		$f(2x) - 2f(x) \leq \widetilde{c} x$ for some $\widetilde{c} > 0$. Using 
		Equation \eqref{eq:TaylorEstimate} for $x = y$, we find
		\begin{equation*}
			f(2x)
			=
			f(x) - x f'(x) +
			\int\limits_x^{2x} (2x-t) f''(t) dt
			\quad
			\Longleftrightarrow
			\quad
			f(2x) - 2 f(x)
			=
			x f'(x) - f(x) +
			\int\limits_x^{2x} (2x-t) f''(t) dt.
		\end{equation*}
		Now we use a backward Taylor development to find
		\begin{equation*}
			f(0)
			=
			f(x) - xf'(x) + \int_x^0 (0-t) f''(t) dt
		\end{equation*}
		\begin{equation*}
			\Longleftrightarrow
			xf'(x) - f(x)
			=
			\int_0^x t f''(t) dt - f(0)
			\leq
			\int_0^x t f''(t) dt
			\leq
			\int_0^x t \frac{c}{t} dt
			=
			cx.
		\end{equation*}
		We then get
		\begin{align*}
			f(2x) - 2 f(x)
			& \leq
			cx + \int\limits_x^{2x} (2x-t) f''(t) dt
			\\
			& \leq
			cx + 
			\int\limits_x^{2x} (2x-t) f\frac{c}{t} dt
			\\
			&=
			cx + cx (\log(4) - 1)
			\\
			&=
			c \log(4) x
		\end{align*}
	\end{subproof}
	Now we can start doing the estimation using the sequence $(\omega_n)$. 
	Define a series $(\alpha)$ by $\alpha_n = \sqrt{\omega_n}$. As pointed out 
	before, we will use $\alpha^{\frac{1}{2^m}}$ as counterweights.
	By doing analogous computations to the first part, we arrive at
	\begin{align*}
		p_\alpha \big(
		&
			\xi_1 \tensor \cdots \tensor \xi_k
			\star_z
			\eta
		\big)
		=
		(k+1) 
		p^k(\xi_1 \tensor \cdots \tensor \xi_k)
		p(\eta)
		\sum\limits_{n=0}^k
		\frac{|B_n^*| |z|^n}{n!}
		\frac{1}{\alpha_{k+1-n} \omega_n}
		\\
		& \leq
		(k+1) 
		p^k(\xi_1 \tensor \cdots \tensor \xi_k)
		p(\eta)
		\frac{1}{\alpha_{k+1}}
		\sum\limits_{n=0}^k
		\frac{|B_n^*| |z|^n}{n! \alpha_n}
		\frac{\alpha_{k+1}}{\alpha_{k+1-n} \alpha_n}
		\\
		& \leq
		(k+1) 
		p^k(\xi_1 \tensor \cdots \tensor \xi_k)
		p(\eta)
		\frac{1}{\alpha_{k+1}}
		\left(
			\sum\limits_{n=0}^k
			\frac{|B_n^*| |z|^n}{n! \alpha_n}
		\right)
		\left(
			\sum\limits_{n=0}^k
			\frac{\alpha_{k+1}}{\alpha_{k+1-n} \alpha_n}
		\right)
		\\
		& \leq
		(k+1) 
		p^k(\xi_1 \tensor \cdots \tensor \xi_k)
		p(\eta)
		\frac{1}{\alpha_{k+1}}
		\kappa_1
		c^{k+1}
		\\
		& \leq
		(k+1) 
		p_{\alpha'}(\xi_1 \tensor \cdots \tensor \xi_k)
		p(\eta)
		\frac{\sqrt{\alpha_k}}{\alpha_{k+1}^{3/4}}
		\kappa_1
		\kappa_2
		\\
		&=
		\kappa
		(k+1)
		\frac{1}{\sqrt[4]{\alpha_{k+1}}}
		\frac{\sqrt{\alpha_k}}{\sqrt{\alpha_{k+1}}}
		p_{\alpha'}(\xi_1 \tensor \cdots \tensor \xi_k)
		p(\eta).
	\end{align*}
	We use the notation $\alpha' = \sqrt{\alpha}$ and $\alpha^{(n)} = 
	\alpha^{\frac{1}{2^n}}$ to keep things readable. We find the estimate on all 
	tensors of degree at most $k$ by the infimum argument and can iterate to higher 
	tensors on the right hand side now:
	\begin{align*}
		p_\alpha \big(
		&
			\xi_1 \tensor \cdots \tensor \xi_k
			\star_z
			\eta_1 \tensor \cdots \tensor \eta_\ell
		\big)
		\\
		& \leq
		\kappa^\ell
		\frac{(k + \ell)!}{k!}
		\frac{\alpha_{k + \ell - 1}'}{\alpha_{k + \ell}' \alpha_{k+\ell}''}
		\frac{\alpha_{k + \ell - 2}''}
		{\alpha_{k + \ell - 1}'' \alpha_{k+\ell-1}^{(3)}}
		\cdots
		\frac{\alpha_{k}^{(\ell)}}
		{\alpha_{k + 1}^{(\ell)} \alpha_{k+1}^{(\ell + 1)}}
		p_{\alpha^{(\ell)}} (\xi_1 \tensor \cdots \tensor \xi_k)
		p^\ell ( \eta_1 \tensor \cdots \tensor \eta_\ell )
		\\
		& \leq
		\kappa^\ell
		2^{k + \ell}
		\ell!
		\frac{1}{\alpha_{k + \ell}'}
		k!
		p^k (\xi_1 \tensor \cdots \tensor \xi_k)
		p^\ell ( \eta_1 \tensor \cdots \tensor \eta_\ell )
		\\
		& \leq
		\kappa^\ell
		2^{k + \ell}
		\frac{k! \ell!}{\alpha_k' \alpha_\ell '}
		p^k (\xi_1 \tensor \cdots \tensor \xi_k)
		p^\ell ( \eta_1 \tensor \cdots \tensor \eta_\ell )
		\\
		&=
		(2p)_{\alpha'} (\xi_1 \tensor \cdots \tensor \xi_k)
		(2 \kappa p)_{\alpha'} ( \eta_1 \tensor \cdots \tensor \eta_\ell ).
	\end{align*}
	This, together with the infimum argument, finishes the proof, since it is clear 
	from the way we can describe the completion of $\left( \Sym^{\bullet}(\lie{g}), 
	\star_z \right)$, that exponential functions are part of it. The continuity of 
	the projections from the grading is clear from the construction and so the 
	topology defined by those $\halbnorm{p}_{\alpha^{m}}$ is projective grouplike.
\end{proof}



\section{Conclusions and further ideas}

\subsection{Banach-Lie Algebras}
The results from the former section immediately lead to the following result.
\begin{corollary}
	Every uniformly quasi-nilpotent Banach-Lie algebra is integrable.
\end{corollary}
Theoretically, we could construct a multiplication on \emph{every quasi-nilpotent}
Banach-Lie algebra, since the Baker-Campbell-Hausdorff series converges globally 
then. However, this does not give a Lie group yet, since we would also want the 
multiplication to be continuous and Proposition \ref{prop:mainthm_part_1} shows that 
this does not need to be the case for only quasi-nilpotent Banach-Lie algebras. In 
the modified $\Tensor_{1^-}$-topology, the continuity of the multiplication follows 
from Proposition \ref{prop:mainthm_rev_1} and Proposition \ref{prop:mainthm_rev_2}. 
Moreover, let $\xi, \eta \in \lie{g}$, then for all $z \in \mathbb{K}$, we have
\begin{equation}
	\exp(\xi) \star_z \exp(\eta)
	=
	\exp\left(
		\frac{1}{z}
		\bch{\xi}{\eta}
	\right)
\end{equation}
as an equality of convergent series, in complete analogy to 
\cite[Corollary 4.3]{esposito.stapor.waldmann:2015a:pre}. From this, it follows
\begin{equation}
	\exp(t \xi) \star_z \exp(s \eta)
	=
	\exp((t + s) \xi)
\end{equation}
for all $t,s \in \mathbb{K}$, $\xi \in \lie{g}$ and hence the continuity of the map
\begin{equation}
	\mathbb{K}
	\ni
	t
	\longmapsto
	\exp(t \xi)
	\in
	\widehat{\algebra{U}}(\lie{g}).
\end{equation}
Clearly, this is a result which we would like to have for an integrable Lie algebra.
\begin{remark}
	It is clear that if a Banach-Lie algebra is not uniformly quasi-nilpotent,
	not all of the three points from Definition \ref{def:grouplike-topology} can be
	fulfilled. Yet, it is interesting to see, that each combination of two of them
	can be fulfilled for at least \emph{every finite-dimensional} Banach-Lie 
	algebra.
	\begin{remarklist}
		\item
		If we drop $(i.)$, the continuity of the multiplication, then we just do not
		have a topological algebra any more, so this is maybe the least interesting
		case. However, this can be fulfilled by taking the $\Tensor_R$-topology for 
		$R = 0$. We even get a very big completion then and can do this even for
		every locally convex Lie algebra.
		
		\item
		If we drop $(ii.)$ and just want a topology which has a continuous 
		projection onto $\lie{g}$, then we can use the $\Tensor_R$-topology for 
		$R = 1$ for every $AE$-Lie algebra.

		\item
		Maybe the most interesting case is rejecting the existence of the 
		projection, hence point $(iii.)$. One can choose 
		the coarsest topology on $\algebra{U}(\lie{g})$, such that all 
		finite-dimensional representations of $\algebra{U}(\lie{g}) 
		\longrightarrow V$ are continuous. This topology was investigated by
		Schottenloher and Pflaum in \cite{pflaum.schottenloher:1998a}
		Every representation gives a seminorm on $\algebra{U}(\lie{g})$, using e.g.
		the spectral norm on the representation space $V$, which is in addition
		submultiplicative. So we get a locally multiplicatively convex topology
		on $\algebra{U}(\lie{g})$ and hence a very big completion with exponentials.
		The problem is, this only works for finite-dimensional Lie algebras. We also 
		know that it can not be extended to general infinite-dimensional Banach-Lie 
		algebras: if we could, every Banach-Lie algebra would be integrable, which 
		is known to be false \cite[Remark 1.14.15]{duistermaat.kolk:2000a}. 
		Furthermore, we know that the projection $\pi \colon
		\algebra{U}(\mathfrak{g}) \longrightarrow \lie{g}$ from
		\eqref{eq:projection} would be discontinuous, although it projects on a
		finite-dimensional subspace.
	\end{remarklist}
\end{remark}
Using the work of Wojty\`nski \cite{wojtynski:1998a}, Galindo and Palacios 
\cite{galindo.palacios:2012a} and Theorem \ref{thm:main}, we can somehow draw a 
little scheme to classify quasi-nilpotency in Banach-Lie algebras:
\begin{equation}
	\label{classification}
	\begin{array}{lccl}
		& \text{nilpotent} &
		\\
		& \Downarrow &
		\\
		\text{topol. nilpotent } &
		(iii) & \Longleftrightarrow &
		\text{Many results}
		\\
		& \Downarrow &
		\\
		\text{unif. quasi-nilpotent } &
		(ii) & \Longleftrightarrow &
		\text{Projective grouplike topologies exist on }
		\algebra{U}(\lie{g})
		\\
		& \Downarrow &
		\\
		\text{quasi-nilpotent } &
		(i) & \Longleftrightarrow &
		\text{Baker-Campbell-Hausdorff series converges globally}
	\end{array}
\end{equation}
A final question in this case, which is a conjecture for associative Banach-Algebras 
since a long time, deals with finite quasi-nilpotency. We can state the same 
conjecture for Lie-case as well. For an integer $k \in \mathbb{N}$, we define
\begin{equation}
	\mathcal{N}_k(n)
	=
	\sup \left\{ 
	\left.
		\norm{ \ad_{\xi_1} \circ \cdots \circ \ad_{\xi_n} }^{\frac{1}{n}} 
	\ \right| \
		\xi_1, \ldots, \xi_n 
		\in 
		\{\eta_1, \ldots, \eta_k \}
		\subset 
		\mathbb{B}_1(0)
	\right\}.
\end{equation}
and say that $\lie{g}$ is finitely quasi-nilpotent, if for every $k \in \mathbb{N}$
\begin{equation}
	\lim\limits_{n \rightarrow \infty}
	\mathcal{N}_k(n)
	=
	0.
\end{equation}
\begin{conjecture}
	A Banach-Lie algebra $\lie{g}$ is quasi-nilpotent if and only if it is finitely 
	quasi-nilpotent.
\end{conjecture}


\subsection{Locally convex Lie Algebras}
One can also think of extending these results to locally convex Lie algebras. The 
first problem is, that there is no big theory on quasi-nilpotency for those Lie 
algebras. It would be possible, of course, to take the notions from Definition 
\ref{def:nilpotencies} and bind them to a pair of seminorms $\halbnorm{p}$ and 
$\halbnorm{q}$.
\begin{definition}[Quasi-nilpotency in locally convex Lie algebras]
	Let $\lie{g}$ be an AE-Lie algebra, then we call it quasi-nilpotent, if for 
	every $\xi, \eta \in \lie{g}$ and every continuous seminorm $\halbnorm{p}$,
	we have
	\begin{equation}
		\lim\limits_{n \rightarrow \infty}
		\left(
			\halbnorm{p}
			\left(
				\ad_\xi^n(\eta)
			\right)
		\right)^{1/n}
		=
		0.
	\end{equation}
	We call it uniformly quasi-nilpotent, if for every continuous seminorm 
	$\halbnorm{p}$, there is another continuous seminorm $\halbnorm{q}$ such that 
	\begin{equation}
		\lim\limits_{n \rightarrow \infty}
		\mathcal{N}_{1; \halbnorm{p}. \halbnorm{q}}(n)
		=
		0
		\quad \text{ with } \quad
		\mathcal{N}_{1; \halbnorm{p}. \halbnorm{q}}(n)
		=
		\sup
		\left\{
		\left.
			\left(
				\halbnorm{q}
				\left( \ad_\xi^n (\eta) \right)	
			\right)^{1/n}
		\ \right| \
			\halbnorm{p}(\xi) \leq 1,
			\halbnorm{p}(\eta) \leq 1
		\right\}
	\end{equation}
	and uniformly m-quasi-nilpotent, if there is a defining set of submultiplicative 
	seminorms and for every continuous seminorm 
	$\halbnorm{p}$
	\begin{equation}
		\lim\limits_{n \rightarrow \infty}
		\mathcal{N}_{1; \halbnorm{p}. \halbnorm{p}}(n)
		=
		0
	\end{equation}
	holds.
\end{definition}
In an analogous way, one could define topological quasi-nilpotency. However, many of 
the proofs we did would not work a priori for a locally convex Lie algebra 
$\lie{g}$. Also the theorems we used on the way are not true for general locally 
convex Lie algebras, maybe not even for AE-Lie algebras. One could think of redoing 
the proof of Proposition \ref{prop:top-nilpotency} for a Fr\'echet-Lie algebra, 
since the construction \eqref{eq:metric-on-X} would still give a metric space $X$, 
when we have a metric on $\lie{g}$ itself. But already to reconstruct the argument 
with the dense and open balls $X_{k, \delta}$, one would probably need 
submultiplicativity of the seminorms. Even then, it is not clear whether all 
arguments would really work.

To redo the arguments of the Propositions \ref{prop:mainthm_rev_1} and 
\ref{prop:mainthm_rev_2}, submultiplicativity of the seminorms would definitely be 
needed, but this is the only restriction. Those results should hence hold for every 
uniformly m-quasi-nilpotent Lie algebra.



\begin{thebibliography}{1}

\footnotesize
\bibitem {beltita.nicolae:2015a}
\textsc{Beltita, D. Nicolae, M.: }\newblock \emph{On universal enveloping algebras in 
  a topological setting}.
\newblock Studia Math. \textbf{230} (2015), 1--29.

\bibitem {bogfjellmo.dahmen.schmedig:2015a}
\textsc{Bogfjellmo, G., Dahmen, R., Schmedig, A.: }\newblock \emph{Character
  groups of Hopf algebras as infinite-dimensional Lie groups}.
\newblock Preprint  \textbf{arXiv: 1410.6468 [math.DG]} (2015).

\bibitem {boseck.czichowski.rudolph:1981a}
\textsc{Boseck, H., Czichowski, G., Rudolph, K.-P.: }\newblock \emph{Analysis
  on topological groups - General Lie Theory}, vol.~37 in \emph{Teubner-Texte
  zur Mathematik [Teubner Texts in Mathematics]}.
\newblock BSB B. G. Teubner Verlagsgesellschaft, 1981.

\bibitem {dixmier:1977a}
\textsc{Dixmier, J.: }\newblock \emph{Enveloping Algebras}.
\newblock \emph{North-Holland Mathematical Library} no. \textbf{14}.
\newblock North-Holland Publishing Co., Amsterdam, New York, Oxford, 1977.

\bibitem {dixon:1991a}
\textsc{Dixon, P.: }\newblock \emph{Topologically nilpotent Banach algebras and 
factorization}.
\newblock Proc. Roy. Soc. Edinburgh Sect. A \textbf{119} (1991), 329--341.

\bibitem {dixon.mueller:1992a}
\textsc{Dixon, P., M\"uller, V.: }\newblock \emph{A note on topologically 
nilpotent Banach algebras}.
\newblock Studia Math. \textbf{102} (1992), 269--275.

\bibitem {duistermaat.kolk:2000a}
\textsc{Duistermaat, J. J., Kolk, J. A. C.: }\newblock \emph{Lie Groups}, Universitext.
\newblock Springer-Verlag, 2000.

\bibitem {esposito.stapor.waldmann:2015a:pre}
\textsc{Esposito, C., Stapor, P., Waldmann, S.: }\newblock \emph{Convergence 
  of the Gutt Star Product}.
\newblock Preprint  \textbf{arXiv: 1590.09160 [math.QA]} (2015).

\bibitem {galindo.palacios:2012a}
\textsc{Galindo, A.M., Palacios, A.R.: }\newblock \emph{Topologically nilpotent 
normed algebras}.
\newblock J. Algebra.  \textbf{368} (2012),126--168.

\bibitem {gloeckner.neeb:2012a}
\textsc{Gl{\"o}ckner, H., Neeb, K.-H.: }\newblock \emph{When unit groups of
  continuous inverse algebras are regular Lie groups}.
\newblock Studia Math.  \textbf{211} (2012), 95--109.

\bibitem {goodman:1971a}
\textsc{Goodman, R.: }\newblock \emph{Differential Operators of Infinite Order
  on a Lie Group, II}.
\newblock Indiana Math. J.  \textbf{21} (1971), 383--409.

\bibitem {gutt:1983a}
\textsc{Gutt, S.: }\newblock \emph{An Explicit $*$-Product on the Cotangent
  Bundle of a Lie Group}.
\newblock Lett. Math. Phys.  \textbf{7} (1983), 249--258.

\bibitem {mitiagin.rolewicz.zelazko:1962a}
\textsc{Mitiagin, B.~S., Rolewicz, S., {\.{Z}}elazko, W.: }\newblock
  \emph{Entire functions in $B_0$-algebras}.
\newblock Studia Math.  \textbf{21} (1962), 291--306.

\bibitem {mueller:1997a}
\textsc{M{\"u}ller, V.: }\newblock \emph{On the joint spectral radius}.
\newblock Annales Polonici Mathematici \textbf{66} (1997), 173--182.

\bibitem {mueller:1994a}
\textsc{M{\"u}ller, V.: }\newblock \emph{Nil, Nilpotent and PI-Algebras}.
\newblock Func. Anal. Op. Theory  \textbf{30} (1994), 259--265.

\bibitem {pflaum.schottenloher:1998a}
\textsc{Pflaum, M.~J., Schottenloher, M.: }\newblock \emph{Holomorphic
  deformation of Hopf algebras and applications to quantum groups}.
\newblock J. Geom. Phys.  \textbf{28} (1998), 31--44.

\bibitem {rasevskii:1966a}
\textsc{Ra{\v{s}}evski{\u{i}}, P.~K.: }\newblock \emph{Associative
  hyper-envelopes of Lie algebras, their regular representations and ideals}.
\newblock Trans. Mosc. Math. Soc.  \textbf{15} (1966), 3--54.

\bibitem {rota.strang:1960a}
\textsc{Rota, G. C., Strang, W. G.: }\newblock \emph{A note on the joint spectral 
  radius}.
\newblock Indag. Math. \textbf{22} (1966), 379--381.

\bibitem {shulman.turovskii:2000a}
\textsc{Shulman, V. S., Turovski\u i, Y. V.: }\newblock \emph{Joint Spectral 
  Radius, Operator Semigroups, and a Problem of W. Wojty\'nski}.
\newblock J. Func. Anal. \textbf{177} (1966), 383--441.

\bibitem {stapor:2015a}
\textsc{Stapor, P.: }\newblock \emph{Convergence 
  of the Gutt Star Product}.
\newblock Master Thesis, University of W\"urzburg (2015).

\bibitem {waldmann:2014a}
\textsc{Waldmann, S.: }\newblock \emph{A nuclear Weyl algebra}.
\newblock J. Geom. Phys.  \textbf{81} (2014), 10--46.

\bibitem {wojtynski:1998a}
\textsc{Wojtynski, W.: }\newblock \emph{Quasi-nilpotent 
{B}anach-{L}ie-Algebras are {B}aker-{C}ampbell-{H}ausdorff}.
\newblock J. Func. Anal.  \textbf{153} (1998), 405--413.

\end{thebibliography}


% shulman.turovskii

% galindo.palacios

\end{document}

%%% Local Variables:
%%% mode: latex
%%% TeX-master: t
%%% End:
