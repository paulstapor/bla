%
% A new paper...
% Title: Convergence of the Gutt Star Product
% git-repository is gstar
%
% From now on, we proudly use the chairx style file for
% everything. All stuff in there is soo useful!
%


%
% Before we start, let's nag a bit about old latex constructs
% just to learn... This will be removed in the final version
%

\RequirePackage[l2tabu, orthodox]{nag}


%
% we always start with 11pt, draft mode on for easier editing and
% english as default language
%

\documentclass[
11pt,                          % standard font size
%draft,                         % draft or final?
english                        % standard language
]{article}


%
% some macro packages
%

\usepackage[english]{babel}    % with explicit language
\usepackage{amsmath}           % ams mathematical stuff
\usepackage[utf8]{inputenc}    % smart input of funny chars
\usepackage[T1]{fontenc}       % also for the font encoding
\usepackage{longtable}         % tables longer than one page
\usepackage{exscale}           % large summation signs in 11pt
\usepackage[final]{graphicx}   % to include pdf pictures
\usepackage[sort]{cite}        % nicer citations
\usepackage{array}             % nice tables
\usepackage{wasysym}           % smiley symbols
\usepackage[a4paper]{geometry} % geometry of page layout
%\usepackage{gitinfo2}          % include git info: Version 2
%\usepackage[multiuser]{fixme}  % correction notes, warnings etc.
\usepackage{xspace}            % better spacing after macros
\usepackage{tikz}              % for commutative diagrams and stuff
\usepackage{ifdraft}           % to determine whether draft mode
\usepackage{chairx}            % the Chair X style file
\usepackage[expansion=false    % no font expansion
           ]{microtype}        % only protrusion
\usepackage[nottoc]{tocbibind} % refs and index in the toc
\usepackage[backref=page,      % backrefs in the bibliography
           final=true,         % always treat as final
           pdfpagelabels       % use pdf page labels
           ]{hyperref}         % hyperrefs are cool!

%
% Some own macros...
%

%\usepackage{mnsymbol}

\newcommand{\ostar}{\star}
\newcommand{\coproduct}{\Delta}
\newcommand{\ocoproduct}{\Delta}

\newcommand{\Eta}{\mathrm{H}}
\newcommand{\Rho}{\operatorname{P}}
\newcommand{\bch}[2]{\mathrm{BCH}\left(#1, #2\right)}
\newcommand{\bchpart}[3]{\mathrm{BCH}_{#1}\left(#2, #3\right)}
\newcommand{\bchparts}[4]{\mathrm{BCH}_{#1, #2}\left(#3, #4\right)}
\newcommand{\bchtilde}[4]{\widetilde{\mathrm{BCH}}_{#1, #2}\left(#3; #4\right)}
\newcommand\ot[2]{\stackrel{\mathclap{#1}}{#2}}
%
%\mathrel{\overset{\makebox[0pt]
%	{\mbox{\normalfont\footnotesize\sffamily #1}}}{#2}}}


%
% pdf files for graphics in the following directory:
%

\graphicspath{{../tikz/}}


%
% tikz libraries to be loaded, feel free to add more...
%

\usetikzlibrary{matrix}
\usetikzlibrary{arrows}
\usetikzlibrary{patterns}
\usetikzlibrary{decorations.pathreplacing}


%
% page dimensions, scaling etc. Not final yet
%


\geometry{bindingoffset=0cm}
\geometry{hcentering=true}
\geometry{hscale=0.8}
\geometry{vscale=0.8}


%
% check whether draft or not: synctex is soo cool
%

%\ifdraft{\synctex=1}{}


%
% fix me settings
%

%\fxusetheme{color}


%
% Get the authors from external file
% This used in all documents of the paper project
%

%\input{../authors/authors}


%
% own local math macros follow here
%


%
% title page for Convergence of the Gutt Star Product
% authors are included from the authors file
% This has to be adapted in the final version
%

\title{Convergence of the Gutt Star Product}

\author{
  \textbf{Chiara Esposito}\thanks{\texttt{chiara.esposito@mathematik.uni-wuerzburg.de}},
  \addtocounter{footnote}{2}
  \textbf{Paul Stapor}\thanks{\texttt{paul.stapor@stud-mail.uni-wuerzburg.de}},
  \addtocounter{footnote}{2}
  \textbf{Stefan Waldmann}\thanks{\texttt{stefan.waldmann@mathematik.uni-wuerzburg.de}}\\[0.5cm]
  \chairXaddress
}

%\date{Current Version of gstar: \gitAuthorIsoDate\\[0.2cm]
%  {\small
%    Last changes by \gitAuthorName{} on \gitAuthorDate \\
%    Git revision of gstar: \texttt{\gitAbbrevHash{}} \gitReferences
%  }
%}

\date{September 2015}


%
% the text starts here
%

\begin{document}

%
% title page
%

\maketitle

%
% abstract
%

\begin{abstract}
    In this work we consider the Gutt star product viewed as an
    associative deformation of the symmetric algebra
    $\Sym^\bullet(\lie{g})$ over a Lie algebra $\lie{g}$ and discuss
    its continuity properties: we establish a locally convex topology
    on $\Sym^\bullet(\lie{g})$ such that the Gutt star product becomes
    continuous. Here we have to assume a mild technical condition on
    $\lie{g}$: it has to be an \emph{Asymptotic Estimate} Lie
    algebra. This condition is e.g. fulfilled automatically for all
    finite-dimensional Lie algebras.  The resulting completion of the
    symmetric algebra can be described explicitly and yields not only
    a locally convex algebra but also the Hopf algebra structure maps
    inherited from the universal enveloping algebra are continuous.
    We show that all Hopf algebra structure maps depend analytically
    on the deformation parameter. The construction enjoys good
    functorial properties.
\end{abstract}

\newpage


%
% table of contents
%

\tableofcontents
\newpage


%
% Introduction
%

\section{Introduction}
\label{sec:Introduction}


\section{Quasi-Nilpotency in Banach-Lie Algebras}
\label{sec:QuasiNilpotency}

\subsection{Classes of Quasi-Nilpotent Banach-Lie Algebras}
First, we want to classify, how we can ``measure'' the nilpotency of a Banach-Lie 
algebra. In an associative Banach algebra $\algebra{A}$, the usual way is looking 
at the spectral radii of the elements $a \in \algebra{A}$ or at the joint spectral 
radii of bounded subsets of $\algebra{A}$. Unfortunately, for a Banach-Lie algebra, 
a good notion of the spectral radius of elements is not so obvious. In 
\cite{galindo.palacios:}, Galindo and Palacios constructed something that they 
called the ``Lie-spectral radius''. We want to go a different way. Let $\lie{g}$ be 
a Banach-Lie algebra, then we know that $\lie{g}$ itself is not an associative 
Banach algebra, but the bounded linear operators on it $\Bounded(\lie{g})$ surely 
form such an algebra using the spectral norm. Since the map $\ad \colon \lie{g} 
\longrightarrow \Bounded(\lie{g})$ is a continuous homomorphism of Lie algebras, we 
know that its image forms a closed (and hence complete) subspace of 
$\Bounded(\lie{g})$ by the closed graph theorem. The subalgebra generated by 
$\Im(\ad)$ is now a subalgebra of $\Bounded(\lie{g})$ and hence a normed (but not 
necessarily complete) associative algebra. However, we can now use the notions from 
the associative theory by applying them to this subalgebra of $\Bounded(\lie{g})$.
\begin{definition}
	\label{Def:Nilpotencies1}
	Let $\lie{g}$ be a Banach-Lie algebra in which the Lie bracket fulfils the 
	estimate
	\begin{equation*}
		\norm{ [\xi, \eta] }
		\leq
		\norm{\xi}
		\norm{\eta}.
	\end{equation*}
	Denote by $\mathbb{B}_1(0)$ all elements $\xi \in \lie{g}$ with 
	$\norm{\xi} = 1$. We say that
	\begin{definitionlist}
		\item
		$\lie{g}$ is topologically nil (or radical, or quasi-nilpotent, or 
		(quasi-nilpotent) of type $(i)$), if every $\xi \in \lie{g}$ is 
		quasi-nilpotent, i.e.
		\begin{equation*}
			\lim_{n \longrightarrow \infty}
			\norm{\ad_{\xi}^n}^{\frac{1}{n}}
			=
			0.
		\end{equation*}
		
		\item
		$\lie{g}$ is uniformly topologically nil or (quasi-nilpotent) of type 
		$(ii)$, if
		\begin{equation*}
			\lim_{n \longrightarrow \infty}
			\mathcal{N}_1(n)
			=
			0
		\end{equation*}
		for
		\begin{equation}
			\mathcal{N}_1(n)
			=
			\sup \left\{ 
			\left.
				\norm{ \ad_{\xi}^n}^{\frac{1}{n}} 
			\right|
				\xi \in \mathbb{B}_1(0)
			\right\}.
		\end{equation}
		
		\item
		$\lie{g}$ is topologically nilpotent or (quasi-nilpotent) of type 
		$(iii)$, if for every sequence
		$(\xi_n)_{n \in \mathbb{N}} \subset \mathbb{B}_1(0)$ we have
		\begin{equation*}
			\lim_{n \longrightarrow \infty}
			\norm{ 
				\ad_{\xi_1} \circ \ldots \circ \ad_{\xi_n}
			}^{\frac{1}{n}}
			=
			0.
		\end{equation*}
		
		\item
		$\lie{g}$ is uniformly topologically nilpotent or (quasi-nilpotent) of 
		type $(iv)$, if
		\begin{equation*}
			\lim_{n \longrightarrow \infty}
			\mathcal{N}(n)
			=
			0
		\end{equation*}
		for
		\begin{equation}
			\mathcal{N}(n)
			=
			\sup \left\{ 
			\left.
				\norm{ 
					\ad_{\xi_1} \circ \ldots \circ \ad_{\xi_n}
				}^{\frac{1}{n}} 
			\right|
				\xi_1, \ldots, \xi_n \in \mathbb{B}_1(0)
			\right\}.
		\end{equation}
	\end{definitionlist}
\end{definition}
This definition is directly taken from \cite{muller}, just using the little adaptations we explained before. We will need one more notion, which will be 
turn out to be very useful later.
\begin{definition}[Symmetrically topologically nil Banach-Lie algebras]
	\label{Def:Nilpotencies2}
	Let $\lie{g}$ be a Banach-Lie algebra like in Definition 
	\ref{Def:Nilpotencies1}, then we say that $\lie{g}$ is symmetrically
	topologically nil or (quasi-nilpotent) of type $(ii')$, if
	\begin{equation*}
		\lim_{n \longrightarrow \infty}
		\mathcal{S}(n)
		=
		0
	\end{equation*}
	for
	\begin{equation}
		\mathcal{S}(n)
		=
		\sup \left\{ 
		\left.
			\left\Vert
				\frac{1}{n!}
				\sum\limits_{\sigma \in S_n}
				\ad_{\xi_{\sigma(1)}} \circ \ldots \circ \ad_{\xi_{\sigma(n)}}
			\right\Vert^{\frac{1}{n}} 
		\right|
			\xi_1, \ldots, \xi_n \in \mathbb{B}_1(0)
		\right\}.
	\end{equation}
\end{definition}
It is immediate to see the following implications:
\begin{equation*}
	(iv) \Longrightarrow 
	(ii') \Longrightarrow 
	(ii) \Longrightarrow 
	(i)
	\quad \text{ and } \quad
	(iv) \Longrightarrow
	(iii) \Longrightarrow 
	(i),
\end{equation*}
whereas the relations between $(iii)$, $(ii)$ and $(ii')$ are not clear. In the 
associative case, we find $(iii) \Longrightarrow (iv)$. It is also fair to ask 
whether $(ii) \Longrightarrow (ii')$.

\subsection{Examples}


\section{Topologies for the Universal Enveloping Algebra}
\label{sec:Topologies}

\subsection{The $\Tensor_R$-Topology}

\subsection{$\E$-like Topologies}
We are particularly interested in the question, when a Banach-Lie algebra $\lie{g}$ allows a locally convex topology on $\algebra{U}(\lie{g})$, which has such a big completion that exponential functions and hence group-like elements are part of the completion. This would the case for every locally multiplicatively convex structure, for example and hence especially for a topology induced by a norm. Unfortunately, this would lead to the fact, that we do not find the original topology on $\lie{g}$ in $\algebra{U}(\lie{g})$.
\begin{remark}
	\label{Rem:BCHDesaster}
	Assume for every $\xi, \eta \in \lie{g}$ we would have $\exp(\xi)$ and 
	$\exp(\eta)$ in $\widehat{\algebra{U}(\lie{g})}$. Then of course their product 
	would be also part of the completion. But from 
	\cite{esposito.stapor.waldmann:2015a:pre} we see that in this case
	\begin{equation*}
		\exp(\xi) \cdot \exp(\eta)
		=
		\exp\left(
			\bch{\xi}{\eta}
		\right),
	\end{equation*}
	so for all $\xi, \eta \in \lie{g}$, we could give a sense to the element 
	$\exp\left( \bch{\xi}{\eta} \right)$, although we know that $\bch{\xi}{\eta}$ 
	does not need to exist at all. 
\end{remark}
We now propose a good criterion, which forbids this kind of effects.
\begin{definition}
	Let $\lie{g}$ be a Banach-Lie algebra and $\tau$ a metrizable topology on 
	$\algebra{U}(\lie{g})$. Then we say $\tau$ is an $\E$-like topology, if the 
	three following things hold:
	\begin{definitionlist}
		\item
		The multiplication in $(\algebra{U}(\lie{g}), \tau)$ is continuous.
		
		\item
		For every $\xi \in \lie{g}$, the exponential series $\sum_{n=0}^\infty 
		\frac{\xi^n}{n!}$ converges in the completion of $(\algebra{U}(\lie{g}), 
		\tau)$.
		
		\item
		The topology of $\lie{g}$ as a subspace of  the completion of $(\algebra{U}
		(\lie{g}), \tau)$ is equivalent to the norm topology of $\lie{g}$ as 
		Banach-Lie algebra.
	\end{definitionlist}
	We say that a Banach-Lie algebra is $\E$-like, when its universal enveloping 
	algebra has an $\E$-like topology.
\end{definition}
We now want to see that if a locally convex topology is $\E$-like, then effects 
like in Remark \ref{Rem:BCHDesaster} can not occur. 
\begin{proposition}
	Let $\lie{g}$ be a Banach-Lie algebra and $\algebra{U}(\lie{g})$ has an $\E$-
	like topology. Then the Baker-Campbell-Hausdorff series converges globally on 
	$\lie{g}$.
\end{proposition}
\begin{proof}
	We use the $\E$-like topology on $\algebra{U}(\lie{g})$.
	Since $\widehat{\algebra{U}(\lie{g})}$ is a Fr\'echet space and $\lie{g}$ is a 
	closed subspace, the can look at the quotient 
	\begin{equation*}
		\algebra{U}'(\lie{g})
		=
		\frac{\widehat{\algebra{U}(\lie{g})}}{\lie{g}},
	\end{equation*}
	which will be a Fr\'echet space again, since $\lie{g}$ is closed and the 
	topology on $\widehat{\algebra{U}(\lie{g})}$ is second countable. We can see 
	$\algebra{U}'(\lie{g})$ as a subspace of $\widehat{\algebra{U}(\lie{g})}$ an 
	get two continuous projections
	\begin{equation*}
		\pi' \colon
		\widehat{\algebra{U}(\lie{g})}
		\longrightarrow
		\algebra{U}'(\lie{g})
		\quad \text{ and }
		\pi \colon
		\widehat{\algebra{U}(\lie{g})}
		\longrightarrow
		\lie{g}
	\end{equation*}
	fulfilling $\pi + \pi' = \id$. So in an $\E$-like topology, we have for $\xi, 
	\eta \in \lie{g}$ and $t,s \in \mathbb{K}$
	\begin{align*}
		\pi \left( \exp(t \xi) \cdot \exp(s \eta) \right)
		& =
		\pi
		\left(
			\lim_{N \rightarrow \infty}
			\left(
				\sum\limits_{n=0}^N
				\frac{t^n \xi^n}{n!}
			\right)
			\cdot
			\lim_{M \rightarrow \infty}
			\left(
				\sum\limits_{m=0}^M
				\frac{s^m \eta^m}{m!}
			\right)
		\right)
		\\
		& \ot{(a)}{=}
		\pi
		\left(
			\lim_{N \rightarrow \infty}
			\lim_{M \rightarrow \infty}
			\left(
				\sum\limits_{n=0}^N
				\frac{t^n \xi^n}{n!}
			\right)
			\cdot
			\left(
				\sum\limits_{m=0}^M
				\frac{s^m \eta^m}{m!}
			\right)
		\right)
		\\
		& \ot{(b)}{=}
		\lim_{N \rightarrow \infty}
		\lim_{M \rightarrow \infty}
		\pi
		\left(	
			\left(
				\sum\limits_{n=0}^N
				\frac{t^n \xi^n}{n!}
			\right)
			\cdot
			\left(
				\sum\limits_{m=0}^M
				\frac{s^m \eta^m}{m!}
			\right)
		\right)
		\\
		& \ot{(c)}{=}
		\lim_{N \rightarrow \infty}
		\lim_{M \rightarrow \infty}
		\sum\limits_{n=0}^{N}
		\sum\limits_{m=0}^{M}
		\bchparts{n}{m}{t \xi}{s \eta}
		\\
		& \ot{(d)}{=}
		\lim_{L \rightarrow \infty}
		\sum\limits_{\ell=0}^{L}
		\sum\limits_{a + b = \ell}
		\bchparts{a}{b}{t \xi}{s \eta}{n}{m}
		\\
		& =
		\lim_{L \rightarrow \infty}
		\sum\limits_{\ell=0}^{L}
		\bchpart{\ell}{t \xi}{s \eta}{n}{m},
	\end{align*}
	where we used the continuity of multiplication in $(a)$ and the one of $\pi$ in 
	$(b)$. In $(c)$ we just evaluated $\pi$ and used the fact that the 
	Baker-Campbell-Hausdorff series is a power series in $(d)$ and hence converges 
	absolutely. So it must converge globally on $\lie{g}$, when the latter is seen 
	as a subspace of $\widehat{\algebra{U}(\lie{g})}$. But since the topology is 
	$\E$-like and therefore equivalent to the one of $\lie{g}$ as Banach-Lie 
	algebra, $\lie{g}$ must be globally BCH.
\end{proof}
From \cite{woj} we see immediately, that being quasi-nilpotent is a necessary 
condition for being $\E$-like for a Banach-Lie algebra. For a finite-dimensional 
Lie algebra, this means that it must be nilpotent, since there quasi-nilpotency 
implies nilpotency. Furthermore, in \cite{stapor:2015a}, it is shown that being 
topologically uniformly nilpotent Banach-Lie algebras are $\E$-like. There and in 
\cite{esposito.stapor.waldmann:2015a:pre}, the so called $\Sym_{1^-}$ was shown to 
be $\E$-like for nilpotent, locally convex Lie algebras.



\section{A sufficient and necessary condition for $\E$-like topologies}
\label{sec:MainProof}

We have everything at the hand to state or main Theorem now:
\begin{theorem}
	\label{Thm:Main}
	Let $\lie{g}$ be a Banach-Lie algebra. The $\lie{g}$ is $\E$-like if and only 
	if it is of type $(ii')$.
\end{theorem}
Since it is rather long and technical, we will split up the proof into two parts.


\subsection{Necessity of Type $II'$-Nilpotency}
It is rather easy to show, that being of type $(ii')$ is really necessary to be 
$\E$-like.
\begin{proposition}
	Let $\lie{g}$ be a Banach-Lie algebra, which is not of type $(ii')$. Then it is 
	not $\E$-like.
\end{proposition}
\begin{proof}
	The idea is to construct two sequences which converge to zero, but whose 
	product does not converge: Being not of type $(ii')$ means, that there is an
	$\varepsilon > 0$, such that $\lim_{n \rightarrow \infty} \mathcal{S}(n) > 
	\varepsilon$. This means that for every $n \in \mathbb{N}$, there is a sequence 
	$\left(\xi_1^k, \ldots, \xi_1^k\right)_{k \in \mathbb{N}} \subset 
	\mathbb{B}_1(0)^n$, 
	such that
	\begin{equation*}
		\lim_{k \longrightarrow \infty}
		\left\Vert
			\frac{1}{n!}
			\sum\limits_{\sigma \in S_n}
			\ad_{\xi_{\sigma(1)}^k}
			\circ \cdots \circ
			\ad_{\xi_{\sigma(n)}^k}
		\right\Vert^{\frac{1}{n}}
		=
		\mathcal{S}(n).
	\end{equation*}
	Let's fix an $\varepsilon' < \varepsilon$, then for every $n$, there is a $k 
	\in \mathbb{N}$, such that
	\begin{equation*}
		\left\Vert
			\frac{1}{n!}
			\sum\limits_{\sigma \in S_n}
			\ad_{\xi_{\sigma(1)}^k}
			\circ \cdots \circ
			\ad_{\xi_{\sigma(n)}^k}
		\right\Vert^{\frac{1}{n}}
		> \varepsilon'.
	\end{equation*}
	We denote those $\xi_1^k, \ldots, \xi_n^k$ as $\eta_1^n, \ldots, \eta_n^n$ and 
	define $\eta^n = \eta_1^n \vee \cdots \vee \eta_n^n \in \Sym^n(\lie{g})$.
\end{proof}


\subsection{Sufficiency of Type $II'$-Nilpotency}

\subsubsection{Part I}

\subsubsection{Part II}


\section{Module Structures in the $\Tensor_R$-Topology}


\section{An Outlook to possible Generalizations and open Questions}


\end{document}

%%% Local Variables:
%%% mode: latex
%%% TeX-master: t
%%% End:
