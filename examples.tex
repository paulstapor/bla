
\RequirePackage[l2tabu, orthodox]{nag}

%
% we always start with 11pt, draft mode on for easier editing and
% english as default language
%

\documentclass[
11pt,                          % standard font size
english                        % standard language
]{article}


%
% some macro packages

\usepackage[english]{babel}    % with explicit language
\usepackage{amsmath}           % ams mathematical stuff
\usepackage[utf8]{inputenc}    % smart input of funny chars
\usepackage[T1]{fontenc}       % also for the font encoding
\usepackage{longtable}         % tables longer than one page
\usepackage{exscale}           % large summation signs in 11pt
\usepackage[final]{graphicx}   % to include pdf pictures
\usepackage[sort]{cite}        % nicer citations
\usepackage{array}             % nice tables
\usepackage{wasysym}           % smiley symbols
\usepackage[a4paper]{geometry} % geometry of page layout
\usepackage{xspace}            % better spacing after macros
\usepackage{tikz}              % for commutative diagrams and stuff
\usepackage{ifdraft}           % to determine whether draft mode
\usepackage{chairx}            % the Chair X style file
\usepackage[expansion=false    % no font expansion
           ]{microtype}        % only protrusion
\usepackage[nottoc]{tocbibind} % refs and index in the toc
\usepackage[backref=page,      % backrefs in the bibliography
           final=true,         % always treat as final
           pdfpagelabels       % use pdf page labels
           ]{hyperref}         % hyperrefs are cool!

%
% Some own macros...

\newcommand{\bch}[2]{\mathrm{BCH}\left(#1, #2\right)}
\newcommand{\bchpart}[3]{\mathrm{BCH}_{#1}\left(#2, #3\right)}
\newcommand{\bchparts}[4]{\mathrm{BCH}_{#1, #2}\left(#3, #4\right)}
\newcommand\ot[2]{\stackrel{\mathclap{#1}}{#2}}
\newcommand{\diff}[2]{\frac{d#1}{d#2}}
\newcommand{\Tangent}{\mathrm{T}}

%
% tikz libraries to be loaded, feel free to add more...

\usetikzlibrary{matrix}
\usetikzlibrary{arrows}
\usetikzlibrary{patterns}
\usetikzlibrary{decorations.pathreplacing}



%
% page dimensions, scaling etc. Not final yet

\geometry{bindingoffset=0cm}
\geometry{hcentering=true}
\geometry{hscale=0.8}
\geometry{vscale=0.8}


%
% header, title, etc.

\title{Note1 - Banach-Lie}

\author{
  \textbf{Raimond Abt},
  \textbf{Paul Stapor}
  \\[0.5cm]
}

\date{November 2015}


%
% The real fun starts...

\begin{document}


%
% title page

\maketitle

\begin{abstract}
    Indeed, very much so...
\end{abstract}

\tableofcontents

\newpage

\section{Quasi-nilpotent Banach-Lie algebras}

As we know, there are different types of quasi-nilpotency of Banach-Lie algebras.
We are interested in seeing some examples, so we will construct two explicit ones:
the first will be topologically uniformly nil, the second topologically nil.
The goal is to show, that in both classes of Banach-Lie algebras, there are objects
which allow an explicit, locally convex topology on their universal enveloping 
algebras.



\subsection{Example 1}

Take the free associative $\mathbb{C}$-algebra in the generators $\{x_n\}_{n \in 
\mathbb{N}}$ together with the relation $x_i x_j = \delta_{i+1}^j$. Let $p \in [1, 
\infty)$and use the $\ell^p$-norm on this space to complete it. An arbitrary 
element in this algebra will have the form
\begin{equation*}
	a =
	\sum\limits_{i < j}
	a_{i, j} x_i x_{i + 1} \cdots x_{j - 1}
	, \
	a_{i,j} \in \mathbb{C}
	\quad \text{ with } \quad
	\norm{a}_p
	=
	\left(
		\sum\limits_{i < j}
		|a_{i, j}|^p
	\right)^{1/p}.
\end{equation*}
We want to call this associative algebra $\algebra{A}_p$. Since we will need it for 
our calculations, we state the following lemma.
\begin{lemma}
	Let $p \in [1, \infty)$, $n \in \mathbb{N}$ and $a^{(1)}, \ldots a^{(n)}$. 
	Denote each element as
	\begin{equation*}
		a^{(k)}
		=
		\sum\limits_{i < j} a_{i, j}^{(k)} x_i \cdots x_{j-1}.
	\end{equation*}
	Then we have the following estimates.
	\begin{lemmalist}
		\item
		The symmetrized product fulfils the inequality
		\begin{equation}
			\label{note1:lemma:Ex1ProductEstimateAssociative}
			\left\Vert
				\sum\limits_{\sigma \in S_n}
				\frac{1}{n!}
				a^{(\sigma(1))} \cdots a^{(\sigma(n))}
			\right\Vert_p
			\leq
			\frac{1}{\sqrt[p]{n!}}
			\norm{a^{(1)}}_p \cdots \norm{a^{(n)}}_p.
		\end{equation}
		
		\item
		$\algebra{A}_p$ is not topologically nilpotent but topologically uniformly 
		nil as associative algebra with characteristic series fulfilling
		\begin{equation}
			\label{note1:lemma:Ex1NilpotencyAssociative}
			\sqrt[p]{\frac{1}{n}} 
			\leq 
			\mathcal{N}_1(n) 
			\leq 
			\sqrt[p]{\frac{\E}{n}}.
		\end{equation}	
	\end{lemmalist}
\end{lemma}
\begin{proof}
	We just need to compute
	\begin{align*}
		\left\Vert
			\sum\limits_{\sigma \in S_n}
			a^{(\sigma(1))} \cdots a^{(\sigma(n))}
		\right\Vert_p^p
		&=
		\sum\limits_{\sigma \in S_n}
		\sum\limits_{i_0 < i_1 < \cdots < i_n}
		\left|
			a_{i_0, i_1}^{(\sigma(1))}
		\right|^p
			\cdots
		\left|
			a_{i_{n-1}, i_n}^{(\sigma(n))}
		\right|^p
		\\
		& \leq
		\Bigg(
			\sum\limits_{i < j}
			\left|
				a_{i,j}^{(1)}
			\right|^p
		\Bigg)
		\cdots
		\Bigg(
			\sum\limits_{i < j}
			\left|
				a_{i,j}^{(n)}
			\right|^p
		\Bigg)
		\\
		& \leq
		\left(
			\sum\limits_{\sigma \in S_n}
			\Bigg(
				\sum\limits_{i < j}
				\left|
					a_{i,j}^{(1)}
				\right|^p
			\Bigg)
			\cdots
			\Bigg(
				\sum\limits_{i < j}
				\left|
					a_{i,j}^{(n)}
				\right|^p
			\Bigg)
		\right)^{1/p}
		\\
		&=
		\sqrt[p]{n!}
		\norm{a^{(1)}}_p \cdots \norm{a^{(n)}}_p.
	\end{align*}
	For the second part, it is clear that the symmetrized product coincides with 
	the usual product, if $a^{(1)} = \cdots = a^{(n)} = a$. Taking $\norm{a}_p = 1$
	and using $\frac{1}{n!} \leq \frac{\E^n}{n^n}$ we get the estimate of the right 
	hand side for $\mathcal{N}_1(n)$ by taking the $n$-th root.
	For the left hand side, we take
	\begin{equation*}
		a_n 
		=
		\frac{1}{\sqrt[p]{n}}
		\sum\limits_{i=1}^n
		x_i.
	\end{equation*}
	Note that $\norm{a_n}_p = 1$ for all $n$ and we get
	\begin{equation*}
		\norm{a_n^n}_p
		=
		\left\Vert
			\left(
				\frac{1}{\sqrt[p]{n}}
				\sum\limits_{i=1}^n
				x_i
			\right)^n
		\right\Vert_p
		=
		\sqrt[p]{\frac{1}{n^n}}
		\norm{x_1 \cdots x_n}_p
		=
		\sqrt[p]{\frac{1}{n^n}},
	\end{equation*}
	yielding $\mathcal{N}_1(n) \geq \sqrt[p]{\frac{1}{n}}$.
	Hence $\algebra{A}_p$ is at least uniformly topologically nil. But since
	\begin{equation*}
		\norm{a_1 \cdots a_n}_p
		=
		\norm{a_1}_p \cdots \norm{a_n}_p
	\end{equation*}
	for $a_i = x_i$ and $i = 1, \ldots, n$, we see that $\algebra{A}_p$ is 
	\emph{not} topologically nilpotent, hence the statement is true.
\end{proof}
Now we want to pass to the Lie algebra $\algebra{A}'_p$ which comes from this 
associative algebra. Note that we get almost the same properties concerning the 
nilpotency.
\begin{lemma}
	The Banach-Lie algebra $\algebra{A}'_p$ has the following properties.
	\begin{lemmalist}
		\item
		The symmetrized iterated Lie brackets fulfil the inequality
		\begin{equation}
			\label{note1:lemma:Ex1ProductEstimateLie}
			\left\Vert
				\sum\limits_{\sigma \in S_n}
				\frac{1}{n!}
				\left[a^{(\sigma(1))}, 
					\left[
						\ldots 
						\left[a^{(\sigma(n))}, b \right]
						\ldots 
					\right]
				\right]
			\right\Vert_p
			\leq
			\sqrt[p]{\frac{2^n}{n!}}
			\norm{a^{(1)}}_p \cdots \norm{a^{(n)}}_p \norm{b}_p.
		\end{equation}
		
		\item
		$\algebra{A}_p$ is not topologically nilpotent but topologically uniformly 
		nil as Lie algebra with characteristic series fulfilling
		\begin{equation}
			\label{note1:lemma:Ex1NilpotencyLie}
			\sqrt[p]{\frac{1}{n}} 
			\leq 
			\mathcal{N}_1(n) 
			\leq 
			\sqrt[p]{\frac{2 \E}{n}}.
		\end{equation}	
	\end{lemmalist}
\end{lemma}
\begin{proof}
	We just have to compute explicitly the norm of the iterated brackets in 
	Equation \eqref{note1:lemma:Ex1ProductEstimateLie} and use Estimate 
	\eqref{note1:lemma:Ex1ProductEstimateAssociative} for the associative algebra.
	Let $a^{(1)}, \ldots, a^{(n)}, b \in \algebra{A}'_p$ and we have
	\begin{align*}
		\Bigg\Vert
			\sum\limits_{\sigma \in S_n}
			\frac{1}{n!}
		&
			\left[a^{(\sigma(1))}, 
				\left[
					\ldots 
					\left[a^{(\sigma(n))}, b \right]
					\ldots 
				\right]
			\right]
		\Bigg\Vert_p
		\\
		&=
		\left\Vert
			\sum\limits_{\sigma \in S_n}
			\frac{1}{n!}
			\sum\limits_{k = 0}^n
			\sum\limits_{1 \leq \ell_1 < \cdots < \ell_k \leq n}
			(-1)^k
			\binom{n}{k}
			a^{(\sigma(\ell_1))}
			\cdots
			a^{(\sigma(\ell_k))}
			\cdot b \cdot
			a^{(\sigma(1))}
			\cdots
			\widehat{a^{(\sigma(\ell))}}
			\cdots
			a^{(\sigma(1))}
		\right\Vert_p
		\\
		& \leq
		\frac{1}{n!}
	\end{align*}
\end{proof}



\subsection{Example 2}

The idea is to combine all algebras $\algebra{A}_p$ to one: Take the free  
$\mathbb{C}$-algebra in the generators $\{x_n^m\}_{n,m \in \mathbb{N}}$, now with 
the relation $x_n^m \cdot x_{n'}^{m'} = \delta_m^{m'} \delta_{n+1}^{n'}$. Now an 
arbitrary element will have the form
\begin{equation*}
	a =
	\sum\limits_{m \in \mathbb{N}}
	\sum\limits_{i < j}
	a_{i, j}^m x_i^m x_{i + 1}^m \cdots x_{j - 1}^m
	, \
	a_{i,j}^m \in \mathbb{C}
	\quad \text{ with } \quad
	\norm{a}
	=
	\sum\limits_{m \in \mathbb{N}}
	\left(
		\sum\limits_{i < j}
		|a_{i, j}|^m
	\right)^{1/m}.
\end{equation*}
One can see from what preceded, that this algebra is not uniformly topologically 
nil any more, but still topologically nil. The idea will now be to endow the 
universal enveloping algebra with weights like $\frac{n!}{(\log n + 1)^n}$ and 
hope, that things work well.


\end{document}