
\RequirePackage[l2tabu, orthodox]{nag}

%
% we always start with 11pt, draft mode on for easier editing and
% english as default language
%

\documentclass[
11pt,                          % standard font size
english                        % standard language
]{article}


%
% some macro packages

\usepackage[english]{babel}    % with explicit language
\usepackage{amsmath}           % ams mathematical stuff
\usepackage[utf8]{inputenc}    % smart input of funny chars
\usepackage[T1]{fontenc}       % also for the font encoding
\usepackage{longtable}         % tables longer than one page
\usepackage{exscale}           % large summation signs in 11pt
\usepackage[final]{graphicx}   % to include pdf pictures
\usepackage[sort]{cite}        % nicer citations
\usepackage{array}             % nice tables
\usepackage{wasysym}           % smiley symbols
\usepackage[a4paper]{geometry} % geometry of page layout
\usepackage{xspace}            % better spacing after macros
\usepackage{tikz}              % for commutative diagrams and stuff
\usepackage{ifdraft}           % to determine whether draft mode
\usepackage{chairx}            % the Chair X style file
\usepackage[expansion=false    % no font expansion
           ]{microtype}        % only protrusion
\usepackage[nottoc]{tocbibind} % refs and index in the toc
\usepackage[backref=page,      % backrefs in the bibliography
           final=true,         % always treat as final
           pdfpagelabels       % use pdf page labels
           ]{hyperref}         % hyperrefs are cool!

%
% Some own macros...

\newcommand{\bch}[2]{\mathrm{BCH}\left(#1, #2\right)}
\newcommand{\bchpart}[3]{\mathrm{BCH}_{#1}\left(#2, #3\right)}
\newcommand{\bchparts}[4]{\mathrm{BCH}_{#1, #2}\left(#3, #4\right)}
\newcommand\ot[2]{\stackrel{\mathclap{#1}}{#2}}
\newcommand{\diff}[2]{\frac{d#1}{d#2}}
\newcommand{\Tangent}{\mathrm{T}}

%
% tikz libraries to be loaded, feel free to add more...

\usetikzlibrary{matrix}
\usetikzlibrary{arrows}
\usetikzlibrary{patterns}
\usetikzlibrary{decorations.pathreplacing}



%
% page dimensions, scaling etc. Not final yet

\geometry{bindingoffset=0cm}
\geometry{hcentering=true}
\geometry{hscale=0.8}
\geometry{vscale=0.8}


%
% header, title, etc.

\title{Note1 - Banach-Lie}

\author{
  \textbf{Raimond Abt},
  \textbf{Paul Stapor}
  \\[0.5cm]
}

\date{November 2015}


%
% The real fun starts...

\begin{document}


%
% title page

\maketitle

\begin{abstract}
    Indeed, very much so...
\end{abstract}

\tableofcontents

\newpage

\section{Quasi-nilpotent Banach-Lie algebras}

As we know, there are different types of quasi-nilpotency of Banach-Lie algebras.
We are interested in seeing some examples, so we will construct two explicit ones:
the first will be topologically uniformly nil, the second topologically nil.
The goal is to show, that in both classes of Banach-Lie algebras, there are objects
which allow an explicit, locally convex topology on their universal enveloping 
algebras.



\subsection{Example 1}

Take the free $\mathbb{C}$-algebra in the generators $\{x_n\}_{n \in \mathbb{N}}$ 
together with the relation $x_i x_j = \delta_{i+1}^j$. Let $p \in [1, \infty)$
and use the $\ell^p$-norm on this space to complete it. An arbitrary element in this algebra will have the form
\begin{equation*}
	a =
	\sum\limits_{i < j}
	a_{i, j} x_i x_{i + 1} \cdots x_{j - 1}
	, \
	a_{i,j} \in \mathbb{C}
	\quad \text{ with } \quad
	\norm{a}_p
	=
	\left(
		\sum\limits_{i < j}
		|a_{i, j}|^p
	\right)^{1/p}.
\end{equation*}



\subsection{Example 2}

The idea is to combine all algebras $\algebra{A}_p$ to one: Take the free  
$\mathbb{C}$-algebra in the generators $\{x_n^m\}_{n,m \in \mathbb{N}}$, now with 
the relation $x_n^m \cdot x_{n'}^{m'} = \delta_m^{m'} \delta_{n+1}^{n'}$. Now an 
arbitrary element will have the form
\begin{equation*}
	a =
	\sum\limits_{m \in \mathbb{N}}
	\sum\limits_{i < j}
	a_{i, j}^m x_i^m x_{i + 1}^m \cdots x_{j - 1}^m
	, \
	a_{i,j}^m \in \mathbb{C}
	\quad \text{ with } \quad
	\norm{a}
	=
	\sum\limits_{m \in \mathbb{N}}
	\left(
		\sum\limits_{i < j}
		|a_{i, j}|^m
	\right)^{1/m}.
\end{equation*}
One can see from what preceded, that this algebra is not uniformly topologically 
nil any more, but still topologically nil. The idea will now be to endow the 
universal enveloping algebra with weights like $\frac{n!}{(\log n + 1)^n}$ and 
hope, that things work well.


\end{document}